\textbf{The Monte Carlo simulation estimates that the projection angle to
be 41.7\degree, with the CI = 22.7\degree, 61.14\degree.} 
\begin{itemize}
\item explains that there hasn't been quantitative constraints on the angle
for which double radio relic can be observed, even though that many studies
have suggested that the detection of radio relic should imply that $\alpha$
should be small. From this simulation we have shown that it is possible to detect
the double radio relics with $\alpha$ being as big as $61.14\degree$. 

\item Lindner et al. provided constraint of $\alpha > 7.8 \degree$ based 
on the dynamics.
\item James' paper did mention how the mass estimation depends on $\alpha$,
with the estimated mass being a lot smaller if $\alpha \ge 65\degree$. 
However, since we did use the larger mass estimate as the input of this
simulation, we can only say that the inferred $\alpha$ is consistent with
the mass estimation. 
% angle dependence of observation of radio relic,  this is the first estimate 
% that provides an upper bound on the angle
% for which we can observe double radio relic  
\item discussion of the different scenarios mention in M11:
1) we are viewing after core passage, but before first turn around, and
the merger speed is low"\\
2) the merger speed is high, but we are viewing after the first turn
around as the two components come together for a second core passage
\item discuss the inclination angle estimate from M11
\item Dave: explain where the limits of the projection angle comes
from. what observational evidence contradicts the low velocity
scenario the most
\end{itemize}

\textbf{With this new piece of evidence, we find that the absence of an
X-ray shock feature from El Gordo, may not be due to the merger speed being
low, as suggested by J13.} 
In particular, taking into account that the estimated projection angle of 
$\sim 41.7\degree$, we estimate the projected relative velocity to be 597 \kilo \meter~\second$^{-1}$, which is consistent with the estimated line-of-sight velocity differences of $586 \pm  96~\kilo \meter ~\second^{-1}$ in M11. 

Furthermore, the study from \cite{L13} Lindner et al. has come up
with an estimation of the shock velocity of the radio relic of El Gordo as 
$\sim 4000~\kilo \meter~\second^{-1}$. While this shock velocity is not the
same as the merger velocity, they should be of similar magnitude. Indeed
our simulation found that a merger velocity of $4000~\kilo
\meter~\second^{-1}$ is within the 95\% credible interval. 
