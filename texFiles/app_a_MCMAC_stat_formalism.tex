The default prior probabilities that we employed can be summarized as
follows for most of the output variables: 
\begin{equation}
	P(TSC_0) = 
	\begin{cases}
		& \text{const}~\text{if }TSC_0 < \text{age of universe at } z=0.87	\\
		& 0~\text{otherwise}
	\end{cases}
\end{equation}

In addition, we apply the following prior on $TSC_1$ only when evaluating the
statistics of $TSC_1$ and $T$, thus allowing realiziations with a valid
outgoing TSC but an invalid returning $TSC_1$.  

\begin{equation}
	P(TSC_1) = 
	\begin{cases}
		& \text{const}~\text{if }TSC_1 < \text{age of universe at } z=0.87	\\
		& 0~\text{otherwise}
	\end{cases}
\end{equation}

To correct for observational limitations, we further convolve the posterior probabilities of the different
realizations with 
\begin{equation}
	P(TSC_0 | T) = 2 \frac{TSC_0}{T}
\end{equation}
to account for how the subclusters move faster at lower $TSC$ and thus it
is more probable to observe the subclusters at a stage with a larger $TSC$.






%\begin{table*} 
%\begin{minipage}{180mm} 
%\caption{Table of comparison of PDFs of the data in the Monte Carlo
%simulation with different applied priors
%\label{tab:predictiveposteriorchecking}} 
%\begin{tabular}{@{}lccccccc@{}}
%\toprule 
%&&& Original PDF & & & PDF of data after applying polarization priors \\ 
%\hline
%%\multicolumn{3-5}{c}{Default Priors} \multicolumn{6-8}{c}{Default + radio prior} \\
%%\cmidrule(r){1-3} \cmidrule(r){4-6}
%Data & Units & Location & 68$\%$ CI$^{\dagger}$ & 95$\%$ 
%CI & Location & 68$\%$ CI  & 95$\%$ CI \\
%\hline 
%$M_{200c_{NW}}$ &$10^{14}$ M$_{\odot}$&&&&&&\\
%$M_{200c_{SE}}$ &$10^{14}$ M$_{\odot}$&&&&&&\\
%$z_{NW}$ &/&&&&&&\\
%$z_{SE}$ &/&&&&&&\\
%$d_{proj}$ &Mpc&&&&&&\\
%\bottomrule 
%\end{tabular} 
%\footnotesize{\\$\dagger$ CI stands for credible interval } \\ 
%\end{minipage} 
%\end{table*} 


