The prior probabilities that we employed can be summarized as: 


%\begin{table*} 
%\begin{minipage}{180mm} 
%\caption{Table of comparison of PDFs of the data in the Monte Carlo
%simulation with different applied priors
%\label{tab:predictiveposteriorchecking}} 
%\begin{tabular}{@{}lccccccc@{}}
%\toprule 
%&&& Original PDF & & & PDF of data after applying polarization priors \\ 
%\hline
%%\multicolumn{3-5}{c}{Default Priors} \multicolumn{6-8}{c}{Default + radio prior} \\
%%\cmidrule(r){1-3} \cmidrule(r){4-6}
%Data & Units & Location & 68$\%$ CI$^{\dagger}$ & 95$\%$ 
%CI & Location & 68$\%$ CI  & 95$\%$ CI \\
%\hline 
%$M_{200c_{NW}}$ &$10^{14}$ M$_{\odot}$&&&&&&\\
%$M_{200c_{SE}}$ &$10^{14}$ M$_{\odot}$&&&&&&\\
%$z_{NW}$ &/&&&&&&\\
%$z_{SE}$ &/&&&&&&\\
%$d_{proj}$ &Mpc&&&&&&\\
%\bottomrule 
%\end{tabular} 
%\footnotesize{\\$\dagger$ CI stands for credible interval } \\ 
%\end{minipage} 
%\end{table*} 
