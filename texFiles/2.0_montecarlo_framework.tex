For this analysis, we made use of the collisionless 
dark-matter-only Monte Carlo modeling code written by Dawson (2013),
hereafter \citepalias{D13}.  
In the code, the time evolution of the head-on merger was computed
based on an analytical model assuming that the only dominant force is the gravitational attraction from
the masses of two truncated Naverro-Frenk-White (hereafter NFW) DM halos.
Other major assumptions for modeling systems with this code include
negligible impact parameter and no self-interaction of dark matter.\par

In the Monte Carlo simulation, many realizations of the collision is
computed from the inputs of each realization, including
the data ($\vec{D}$) and the model variable ($\alpha$). In particular,
the standard required data, which were in the form of samples of the probability density
functions (PDFs), included the masses ($M_{200_{NW}},M_{200_{SE}}$) the
redshifts ($z_{NW}, z_{SE}$) and the projected separation of the two
subclusters ($d_{proj}$).  
%For the $j-$th realization, w
In each realization, we randomly drew the samples of the PDFs.
%
%\begin{equation}
%	D_i^{j} \sim \mathcal{L}(\vec{\theta}|D_i) =  P(D_i | \vec{\theta})
%\end{equation}
%and we also draw the model variable $\alpha$ from the prior:
%\begin{equation}
%	\alpha^j \sim P(\alpha)
%\end{equation}
These inputs are then used for computing the output variables
($\vec{\theta}^\prime$) by making use of conservation of energy to describe
their collision due to the mutual gravitational attraction.
%\begin{equation}
%(\vec{\theta}^\prime)^{(j)} = f(\vec{D_i}^j, \alpha^j) 
%\end{equation}
%where $f$ are some suitable functions expressing the conservation of energy.
(See Table \ref{tab:inputs}
for quantitative descriptions of the sample PDFs and we outline how those
PDFs are obtained in the following subsections.) 
To ensure convergence of the output PDFs, in total, 2 million (to be
confirmed) realizations were computed. The results, however, are
consistent up to a fraction of a percent just from 20 000 runs
\citepalias{D13}.\par    
We note that the Monte Carlo simulation is described from a Bayesian
point of view but the analysis differs from conventional Bayesian inference. The Bayes
chain rule underlies the simulation can be written as:
\begin{equation}
    P(\vec{\theta}|\vec{D}) \propto P(\vec{D}|\vec{\theta})P(\vec{\theta})
\end{equation}
where the likelihood is defined to be the PDF of $\vec{D}$ given $\vec{\theta}$,
i.e. the input variables, not statistical parameters, and the priors are
defined to be the probabilities due to prior knowledge of the estimated values of
$\vec{\theta}$. The output variables $\vec{\theta}^\prime$, on the other
hand, were computed according to the conservation of energy, which is
represented by a suitable functional form $f$ below. For example,the
calculation of the output variables of the $j$-th realization can be denoted as: 
\begin{equation}
    (\vec{\theta}^\prime)^{(j)} = f(\vec{\theta}^{(j)}, \vec{D}) 
\end{equation}    
and computed over all $j$ realizations. Finally, we took the physical
constraints on $\vec{\theta}$ and $\vec{\theta}^\prime$ into account by
examining the resulting physical variables against the physical limits and
excluding realizations that would produce impossible values. We refer to this
process of excluding realizations as ``applying prior probability''. 

%To model projection effects, we randomly draw a projection
%angle in each realization. We throw out realizations with unphysical
%outputs.  \textbf{what are the inputs}
