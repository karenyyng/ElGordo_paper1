For this analysis, we used the collisionless 
dark-matter-only Monte Carlo modeling code written by Dawson (2013),
hereafter \citepalias{D13}.  
In the D13 code, the time evolution of the head-on merger was computed
based on an analytical model assuming that the only dominant force is the gravitational attraction from
the masses of two truncated Naverro-Frenk-White (hereafter NFW) DM halos.
Other major assumptions for modeling systems with this code include
negligible impact parameter (cite M11???) and no self-interaction of dark matter.
No mass accretion during the merger. 
\par

In the Monte Carlo simulation, many realizations of the collision is
computed from the inputs of each realization, including
the data ($\vec{D}$) and the model variable ($\alpha$). In particular,
the standard required data, which were in the form of samples of the
probability density functions (PDFs), included the masses ($M_{200_{NW}},M_{200_{SE}}$) the
redshifts ($z_{NW}, z_{SE}$) and the projected separation of the two
subclusters ($d_{proj}$).  
%For the $j-$th realization, w
In each realization, we randomly drew samples of the PDFs.
%
%\begin{equation}
%	D_i^{j} \sim \mathcal{L}(\vec{\theta}|D_i) =  P(D_i | \vec{\theta})
%\end{equation}
%and we also draw the model variable $\alpha$ from the prior:
%\begin{equation}
%	\alpha^j \sim P(\alpha)
%\end{equation}
These inputs are then used for computing the output variables
($\vec{\theta}^\prime$) by making use of conservation of energy to describe
their collision due to the mutual gravitational attraction.
%\begin{equation}
%(\vec{\theta}^\prime)^{(j)} = f(\vec{D_i}^j, \alpha^j) 
%\end{equation}
%where $f$ are some suitable functions expressing the conservation of energy.
(See Table \ref{tab:inputs}
for quantitative descriptions of the sample PDFs and we outline how those
PDFs are obtained in the following subsections.) 
To ensure convergence of the output PDFs, in total, 2 million (to be
confirmed) realizations were computed. The results, however, are
consistent up to a fraction of a percent just from 20 000 runs
\citepalias{D13}.\par    
We adapt a Bayesian interpretation of the outputs of the Monte Carlo
simulation. The Bayes chain rule underlies the simulation can be written as:
\begin{equation}
    P(M|\vec{D}) \propto P(\vec{D}|M)P(M)
\end{equation}
where the likelihood is defined to be the PDF of $\vec{D}$ given our
physical model $M$ which we parametrize using variables in Table 1
($\vec{\theta}$) by
%i.e. the input variables, not statistical parameters, 
%. The output variables $\vec{\theta}^\prime$, on the other
%hand, were computed according to  
assuming conservation of energy, which is
represented by a suitable functional form $f$ below. For example,the
calculation of the output variables of the $j$-th realization can be denoted as: 
\begin{equation}
    (\vec{\theta}^\prime)^{(j)} = M(\vec{\theta}^{(j)}, \vec{D}) 
\end{equation}    
and computed over all $j$ realizations. Finally, we took the physical
constraints on the dynamical variables into account by
examining the resulting physical variables against the physical limits and
excluding realizations that would produce impossible values. We refer to this
process of excluding unphysical realizations as applying priors. 
Even though we denote the priors for one dimension at a time (See Appendix~\ref{app:priors}), 
the correlations between different variables are properly taken into account
 due to how we throw away all the variables of problematic
realizations. 
%To model projection effects, we randomly draw a projection
%angle in each realization. We throw out realizations with unphysical
%outputs.  \textbf{what are the inputs}
