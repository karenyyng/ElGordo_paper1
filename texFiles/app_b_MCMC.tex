The reduced shear generated by each NFW halo is determined by its
mass ($m_{200c}$) and the position of its center ($\vec{s}$). 
%This parametrization
%is possible since we make use of the mass-concentration relationship from 
%\citet{Duffy2008} to express the concentration of the halo in terms of the
%$m_{200c}$.
\par
% aligning multiline equations:
% http://tex.stackexchange.com/questions/44450/how-to-align-a-set-of-multiline-equations
%\begin{align}
%    &\begin{aligned}    
%    P(m_{a},  \vec{s_{a}}, &m_{2}, \vec{s_{2}} | \vec{e}) \propto \qquad\\
%    &P(\vec{e} | m_{1}, \vec{s_{1}}, m_{2}, \vec{s_{2}}) P(m_1) 
%    P(\vec{s_{1}}) P(m_2) P(\vec{s_{2}})  
%    \end{aligned}
%\end{align}
We consider the joint posterior as the fit to the ellipticity data:
%\begin{align}
%    &\begin{aligned}
%			&\log(P( m_{1}, m_{2}, \vec_{s_1}, \vec_{s_2} | \vec{e} )) \propto\\
%			&-\left[\frac{(\hat{e_1}(m_1, m_2, \vec{s_1}, \vec{s_2}) - e_1)^2
%    }{\sigma_{e_1}^2+\sigma_{SN}^2 }+ 
%    \frac{(\hat{e_2}(m_1, m_2) - e_2)^2
%    }{\sigma_{e_2}^2 + \sigma_{SN}^2 }\right] \label{eqn:jointposterior} 
%    \end{aligned}
%\end{align} 
%where we have fixed the centers of the halos so $\vec{s}_1$ and $\vec{s}_2$
%so $\vec{s}_1$ and $\vec{s}_2$ are left out of the joint posterior.

\begin{align}
    &\begin{aligned}
			&\log(P( m_{1}, \vec{s}_{1}, m_{2}, \vec{s}_{2} | \vec{e} )) \propto\\
%    &-\left[\frac{(\hat{e_1}(m_1, \vec{s_1}, m_2, \vec{s_2}) - e_1)^2
%    }{\sigma_{e_1}^2+\sigma_{SN}^2 }+ 
%    \frac{(\hat{e_2}(m_1, \vec{s_1}, m_2, \vec{s_2}) - e_2)^2
%    }{\sigma_{e_2}^2+\sigma_{SN}^2 }\right] \label{eqn:jointposterior} 
    \end{aligned}
\end{align} 

Gaussian shape noise of the background galaxies are represented by $\sigma_{SN} = 0.25$(to be checked) represents Gaussian shape noise
of the background galaxies, this is also the form of uncertainty made used
of by Umetsu et. al. on a similar analysis; 


The reduced shear due to the NFW halos can be
decomposed into two components, $\hat{e_1}$ and $\hat{e_2}$  
%The ellipticities generated by a NFW halo can be summarized as:
%\begin{align}
%    \hat{e}_1(m, \vec{s_1}) &=\\
%    \hat{e}_2(m, \vec{s_2}) &= 
%\end{align}
%\textbf{where did we use Duffy et al?}
%To reduce the number of model variables, we also made use of the
%mass-concentration relationship for NFW halos from Duffy et al. (2008). 

We apply a uniform (or log flat) prior and only drew starting mass values between
$10^{13} M_\odot$ and $10^{15} M_\odot$ for our MCMC
chains, as informed by previous published mass estimates. (M12, J13, Zitrin
et al. 2013). For each of the subsequent MCMC step, we 
draw a random pair of mass value with the values of
previous step as the means of the distributions, and two pairs of
coordinates for the centroids, i.e. we can write down the
proposal functions at each MCMC step as: 

\begin{align}
	&\Delta m = N(0, \sigma_m) \hspace{1pc} \text{for each halo} \\
	&\Delta s = U(s_{min}, s_{max}) \hspace{1pc} \text{for each RA, DEC for
	each halo}
\end{align}






We may use blocking. 
(THERE SHOULD BE A TRACE PLOT HERE REPORTING THE ACCEPTANCE RATE.)

We vary the step size of our MCMC  before the burn-in period such that an
optimal acceptance rate of the Metropolis algorithm of $\sim0.234$ \citep{Roberts97} is achieved.

In addition to using the Rubin-Gelman R statistic to check for convergence,
we remove autocorrelation of our chains by estimating the effective sample
size of our chains.
 
