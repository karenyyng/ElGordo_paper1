The reduced shear generated by each NFW halo is determined by its
mass ($m_{200c}$) and the position of its center ($\vec{s}$). 
%This parametrization
%is possible since we make use of the mass-concentration relationship from 
%\citet{Duffy2008} to express the concentration of the halo in terms of the
%$m_{200c}$.
\par
% aligning multiline equations:
% http://tex.stackexchange.com/questions/44450/how-to-align-a-set-of-multiline-equations
%\begin{align}
%    &\begin{aligned}    
%    P(m_{a},  \vec{s_{a}}, &m_{2}, \vec{s_{2}} | \vec{e}) \propto \qquad\\
%    &P(\vec{e} | m_{1}, \vec{s_{1}}, m_{2}, \vec{s_{2}}) P(m_1) 
%    P(\vec{s_{1}}) P(m_2) P(\vec{s_{2}})  
%    \end{aligned}
%\end{align}
We consider the joint posterior as the fit to the ellipticity data:
\begin{align}
    &\begin{aligned}
    &\log(P( m_{1}, \vec{s_{1}}, m_{2}, \vec{s_{2}} | \vec{e} )) \propto\\
    &-\left[\frac{(\hat{e_1}(m_1, \vec{s_1}, m_2, \vec{s_2}) - e_1)^2
    }{\sigma_{e_1}^2+\sigma_{SN}^2 }+ 
    \frac{(\hat{e_2}(m_1, \vec{s_1}, m_2, \vec{s_2}) - e_2)^2
    }{\sigma_{e_2}^2+\sigma_{SN}^2 }\right] \label{eqn:jointposterior} 
    \end{aligned}
\end{align} 
where $\sigma_{SN} = 0.25$(to be checked) represents Gaussian shape noise
of the background galaxies; The reduced shear due to the NFW halos can be
decomposed into two components, $\hat{e_1}$ and $\hat{e_2}$  

%\textbf{where did we use Duffy et al?}
%To reduce the number of model variables, we also made use of the
%mass-concentration relationship for NFW halos from Duffy et al. (2008). 
We only drew starting mass values between
$10^{13} M_\odot$ and $10^{15} M_\odot$ for our MCMC
chains, as informed by previous published mass estimates. (M11, J13, Zitrin et al. 2013). Each of the subsequent MCMC step then
draws a random pairs of mass value with the values of
previous step as the means of the distributions. 
The ellipticities generated by a NFW halo can be summarized as:
\begin{align}
    \hat{e}_1(m, \vec{s_1}) &=\\
    \hat{e}_2(m, \vec{s_2}) &= 
\end{align}

 
