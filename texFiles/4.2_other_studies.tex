Compare to \citet{L13}.
Compare to \citet{Donnert13} for their best fit scenario.


Initial velocity is higher for the hydrodynamical simulations than our
Monte Carlo simulations.

This is reflected by the low probability for our Monte Carlo simulations
to reach $d_max$. This is an aspect that should be further investigated in
future cosmological simulations.
 







%See Menanteau 2012 v1 last section to incorporate discussion of sound
%crossing time of $\sim $ 1~Gyr. 

%\begin{itemize}
%\item I want to compare the radio relic speed $~4300 km~s^{-1}$ estimated in  Lindner et al. to our
%$v_{3D}{(t_{obs})}$ estimate but I am not sure if they are quoting the speed
%in the same reference frame. I need to double check
%%Also we note that from Lindner et al. in press, it is found that the
%%projected speed of the radio relic is estimated to be $4300 \pm ^{800}_{500} \kilo
%%\meter~\second{-1}$. Since the radio relic does not experience
%%gravitational effects to slow it down and is generated from the merger, it
%%can be used to approximate the collision speed.  
%\item I can also discuss the TSC constraint from the observation of the
%depression in X-ray (the wake) using the argument that sound-crossing time is $\sim 1$
%Gyr. This should set an upper limit to the $TSC$, but $TSC_0 = 0.62$ Gyr
%and $TSC_1 = 1.01$ Gyr, I do not think it helps break the degeneracy.   
%\end{itemize}
%
