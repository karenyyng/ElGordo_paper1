% mn2esample.tex
%
% v2.1 released 22nd May 2002 (G. Hutton)
%
% The mnsample.tex file has been amended to highlight the proper use of
% LaTeX2e code with the class file and using natbib cross-referencing.
% These changes do not reflect the original paper by A. V. Raveendran.
%
% Previous versions of this sample document were compatible with the LaTeX
% 2.09 style file mn.sty v1.2 released 5th September 1994 (M. Reed) v1.1
% released 18th July 1994 v1.0 released 28th January 1994

\documentclass[letterpaper,useAMS,usenatbib]{"mn2e"}
%\documentclass[letterpaper,useAMS]{"mn2e"}
%LINUX version of the path
%\documentclass[useAMS,usenatbib,letterpaper]{"/media/blank/Macintosh
%HD/Users/karenyng/Library/texmf/tex/latex/commonstuff/mn2e"}

% If your system does not have the AMS fonts version 2.0 installed, then
% remove the useAMS option.
%
% useAMS allows you to obtain upright Greek characters.  e.g. \umu, \upi
% etc.  See the section on "Upright Greek characters" in this guide for
% further information.
%
% If you are using AMS 2.0 fonts, bold math letters/symbols are available
% at a larger range of sizes for NFSS release 1 and 2 (using \boldmath or
% preferably \bmath).
%
% The usenatbib command allows the use of Patrick Daly's natbib.sty for
% cross-referencing.
%
% If you wish to typeset the paper in Times font (if you do not have the
% PostScript Type 1 Computer Modern fonts you will need to do this to get
% smoother fonts in a PDF file) then uncomment the next line 
%\usepackage{Times}

%%%%% AUTHORS - PLACE YOUR OWN MACROS HERE %%%%%
\usepackage{hyperref}
\usepackage{amssymb}
\usepackage{graphicx}
\usepackage{amsmath}
\usepackage[amssymb]{SIunits} 
\usepackage{booktabs}
\usepackage{hhline}
\usepackage{breqn}
\usepackage{standalone}
\usepackage{dcolumn}
	\newcolumntype{d}[1]{D{.}{.}{#1}}
\usepackage{tabularx}
\usepackage{booktabs}
\usepackage{microtype}
\graphicspath{{graphics/}}
\newcommand{\mc}[1]{\multicolumn{1}{c}{#1}} % handy shortcut macro
%-----------------------------------------------------------------------

\defcitealias{D13}{D13}
\defcitealias{Jee13}{J13}
\defcitealias{M12}{M12}
\defcitealias{Sifon13}{Sif\'{o}n 2013}
\def\apjl{ApJL }
\def\aj{AJ }
\def\apj{ApJ }
\def\pasp{PASP }
\def\spie{SPIE }
\def\apjs{ApJS }
\def\araa{ARAA }
\def\aap{A\&A }
\def\nat{Nature }
\def\mnras{MNRAS }
\def\mnrasl{MNRASL }
\providecommand{\eprint}[1]{\href{http://arxiv.org/abs/#1}{#1}}
\providecommand{\adsurl}[1]{\href{#1}{ADS}}
\providecommand{\ISBN}[1]{\href{http://cosmologist.info/ISBN/#1}{ISBN: #1}} 

\title[The dynamics and merging scenario of ACT-CL J0102-4915, El
Gordo]{The dynamics and the inbound merging scenario of the galaxy cluster ACT-CL J0102-4915, El Gordo}
\author[K. Y. Ng et al.]{K. Y. Ng$^{1}$, W. A. Dawson$^{2}$, D. Wittman$^{1}$, J.
Jee$^{1}$, J. Hughes$^{3}$, F. Menanteau$^{3}$, C. Sif\'{o}n$^{4}$\\
(temporary order)\\
$^{1}$Department of Physics, University of California Davis, One Shields
Avenue, Davis, CA 95616, USA\\ 
$^{2}$Lawrence Livermore National Laboratory, P.O. Box 808, Livermore, CA 94551-0808, USA \\
$^3$Department of Physics \& Astronomy,
Rutgers University, 136 Frelinghysen Rd., Piscataway, NJ 08854, USA\\
$^{4}$Leiden Observatory, Leiden University, PO Box 9513, NL-2300 RA
Leiden, Netherlands\\}
\begin{document}
\date{arXiV 666} \pagerange{\pageref{firstpage}--\pageref{lastpage}}
\pubyear{2014} \maketitle \label{firstpage}
\begin{abstract} 
Merging galaxy clusters with radio relics provide rare insights to the merger
dynamics as the relics are created by the violent merger process. 
We demonstrate one of the first uses of the
properties of the radio relic to reduce the uncertainties of the dynamical variables 
and determine 3D configuration of a cluster merger, ACT-CL J0102-4915,named El Gordo. 
From the double radio relic observation and the X-ray observation of a
comet-like gas morphology induced by motion of the cool core, 
it is believed that El Gordo is observed shortly after the first
core-passage of the subclusters.
%At a redshift of 0.87, El Gordo (M$_{200c} = 
%2.75\times10^{15} \pm^{7.4}_{1.5}$ M$_{\sun}$) is one of the most massive
%clusters discovered in the early universe. The two subclusters of El
%Gordo has a mass ratio of around 2:1. 
%The X-ray and weak-lensing data of El Gordo show an offset of X kpc between
%the intracluster gas and the dark matter (DM) at $\sim$4 $\sigma$ level.
%All these features of El Gordo make it part of a valuable class of
%dissociative mergers that can probe the self-interaction of dark matter.
We employ a Monte Carlo simulation to investigate the three-dimensional (3D)
configuration and dynamics of El Gordo. 
%and provide a summary of the probability
%density functions of the inferred variables.
By making use the polarization
fraction of the radio relic, we are able to constrain the estimate of the
angle between the sky and the merger axis to be $\alpha = 21\degree
\pm^9_{11}$. We find the relative 3D merger speed of El Gordo to be
$2400\pm^{400}_{200} ~\kilo\meter~\second^{-1}$. We put our estimates of
the time-since-collision into context by showing that if the time-averaged
shock velocities $< 1100~ \kilo\meter~\second^{-1}$ in the center of mass
frame, the two subclusters are more likely to be moving towards, rather
than away from each other, after the apocenter. 
We compare and contrast the merger scenario of El Gordo with that of the Bullet
Cluster, and show that this late-stage merging scenario might help
explain why the southeast dark matter lensing peak of El Gordo is
closer to the merger center than the southeast cool core. 
Finally, we provide our insight on what information from simulations and
observations would help us better constrain the merger scenarios for other
bimodal merging clusters. 
\end{abstract}
\begin{keywords}
gravitational lensing -- dark matter -- cosmology: observations -- galaxies: clusters: individual (ACT-CL J0102-4915) --
galaxies: high redshift -- methods: statistical 
\end{keywords}
\section{Introduction} 
Mergers of dark-matter-dominated galaxy clusters probes properties
of the cluster components like no other systems. 
Clusters of galaxies are made up of $\sim80\%$ of dark matter in mass content, 
with a smaller  portion of intercluster gas($\sim15\%$ in mass content), and
sparsely spaced galaxies ($\sim2\%$ in mass content). During a merger of
clusters, the subclusters are accelerated to high speeds of several
thousand \kilo \meter~\second$^{-1}$. The offsets of
different components of the subclusters reflect the differences in the
strengths of interactions between various components. Galaxies are
expected to lead the gas due to its negligible interaction cross
sections with other components. The intracluster medium (ICM) is expected to lose
momentum through electromagnetic interactions. On the other hand, offsets
between dark matter and galaxies may suggest dark matter self-interaction
(\citealt{Kahlhoefer14}, \citealt{Randall2008d}).  
\par
El Gordo possesses a range of noteworthy features that allows us to constrain
the merger dynamics in multiple ways.  Ever since the discovery of El Gordo
in the Atacama Camera Telescope survey (ACT; \citealt{Marriage11}), there is an ongoing effort for
collecting comprehensive data for El Gordo.
From the spectroscopy and Dressler-Schecter test for the member galaxies
in \cite{Sifon13}, El Gordo is confirmed to be a binary merger 
without significant substructures. This picture is further supported by the
weak lensing analysis by \cite{Jee13}. The weak lensing analysis shows
a mass ratio of $\sim$2:1  between the two main subclusters, named according to their location as the northeast (NW) and southeast (SE) subclusters respectively. 
(See Figure~\ref{fig:config}). El Gordo has interesting intracluster medium morphology as shown in the X-ray. In the northwest, it shows a wake feature, i.e.,
turbulent flow due to object of higher density moving through fluids, while in the southeast, it shows
highest X-ray emissivity indicative of a cool gas core southeast of the
wake. The cool gas core may have passed from the northwest to the southeast
to have caused this morphology (\citealt{M12}, hereafter M12). 
The extended mass distribution of El Gordo also makes it a good
gravitational lens. \cite{Zitrin13} have found multiple strong
gravitationally lensed images around the center region of El Gordo. 
On the outskirt, strong radio emission is detected in
the NW and the SE respectively. These radio emission has steep spectral
index gradient and are identified as radio relic associated with shockwaves
created from a merger \citep{L13}. 
\begin{figure}
	\includegraphics[width=.95\linewidth]{ElGordo.pdf}
	\caption{Configuration of El Gordo showing overlay of dark
		matter distribution in blue, and X-ray emissivity in red. 
		(Image credit: NASA, ESA and \citealt{Jee13}). 
		The cross markers show the positions of the northwest (NW) and
		southeast (SE) dark matter density peaks, and the center of mass (CM)
		locations respectively. Note that the mass ratio of the NW subcluster
		to the SE subcluster is $\sim 2:1$ \citep{Jee13}. 
		The dashed white lines indicate the approximate location and extent of the northwest radio relic (NW relic), the east radio relic (E relic) and the
		southeast radio relic (SE relic) \citep{L13}.
		\label{fig:config}}
\end{figure}
El Gordo is one of the small samples of galaxy clusters ($\sim50$) that have
been associated with a radio relic and show dissociation between the X-ray
gas and the DM subclusters. Even fewer of them have been studied in
great details, making El Gordo a valuable candidate for further analysis.\par 
In this paper, we combined most of the available information of El Gordo
with the main goal of giving estimates of
the dynamical parameters after taking into account all
constraints and uncertainties due to the missing variables.
Determining the time-since-collision of mergers of similar clusters helps
us reconstruct different stages of a cluster merger.
Mergers of clusters proceed on the time-scale of millions of year,
observations of each cluster only provides a snapshot of a particular type
of merger. In order to understand the merger process observationally, 
we need to identify different stages of similar dissociative mergers and
gather statistics to understand the physics of the mergers.  
Another crucial piece of missing information is the 3D
configuration, i.e.\ the angle between the plane of the sky and the merger
axis called the projection angle $\alpha$. Since most of the dynamical
observables are projected quantities while the modelling of physics
requires 3D
variables, the deprojection based on $\alpha$ contributes the
largest amount of uncertainties to the dynamical variables \citepalias{D13}.
From the morphology of the double relic of El Gordo, it is believed that
$\alpha$ should be small. For mergers with a
large projection angle, the radio emission would be projected towards the
center of the merger, which is hard to be detected \citep{Vazza11}.
However, the only quantitative constraints on $\alpha$ for El Gordo is from
\cite{L13} with a lower bound of $\alpha \geq 11.6 \degree$. A tighter
constraint on $\alpha$ is needed for us to reduce uncertainty of the
dynamical variables. 
\par 
We employed a data-driven approach that thoroughly probes parameter
space by directly drawing samples from the probability density functions
(PDFs) of
the observables. 
This work based on Monte Carlo simulation is particularly important since
it is forbiddingly expensive to simulate and analyze clusters similar to El
Gordo in high resolution. Previous attempts at modeling El Gordo with hydrodynamical
simulations such as \cite{Donnert13} and \cite{Molnar14} provided only in
total a dozen possible configurations of El Gordo, which do not
reflect the input uncertainties. Another approach for
estimating dynamical parameters would be to look for multiple analogs of El Gordo in cosmological
simulations.  However, under the hierarchical picture
of structure formation in the $\Lambda$CDM model, there is a rare chance
for massive clusters like El Gordo to have formed at a redshift of $z = 0.87$.  
The number density of analogs with mass comparable to El Gordo in a
cosmological simulation is as low as $10^{-11} \mega\text{pc}^{-3}$ \citepalias{M12}.  
\par
In the following sections, we adopt the following conventions: (1) we
assume the standard $\Lambda$CDM cosmology with $\Omega_{m} = 0.3$, $\Omega_{\Lambda} = 0.7$. (2) All confidence intervals are quoted at the 68\% level unless otherwise stated. 
(3) All credible intervals (a.k.a. Bayesian confidence intervals) are also
quoted at the 68\% level unless otherwise stated and are central credible
intervals. We adopt this terminology to remind readers of our Bayesian
interpretation of the probability density functions (PDFs), and that we obtained
the intervals / regions are estimated by integrating the posterior
probability densities.  
(4) All quoted masses ($M_{200c}$) are based on mass contained
within $r_{200}$ where the mass density is 200 times the critical density
of the universe at the redshift of $z = 0.87$. 
\section{DATA} 
We gathered and analyzed data from multiple sources for different
purposes. For assigning membership of galaxies to the two identified subclusters, we
examined the spectroscopic data obtained from the Very Large Telescope (VLT) and
Germini South as described in \citetalias{M12} and \citet{Sifon13}.
For the weak-lensing mass estimation, we used the published
Monte Carlo Markov Chains (MCMC) mass estimates from \citetalias{Jee13}.
See Table~\ref{tab:inputs} for descriptions of the PDFs of the input
variables. \par 
%and these galaxies were discussed as the population in Region A in
%\citetalias{Jee13}.  
In order to further constrain our parameter space, we referred to the properties of
the radio relics from \citet{L13}. El Gordo shows radio emission on the
periphery of both subclusters \citepalias{M12}. The two radio relics, the
northwest (NW) relic and the southeast (SE) relic, of El Gordo were first
discovered in the Sydney University Molonglo Sky Survey (SUMSS) data in low
resolution at 843 MHz \citep{Mauch03} as shown in M12. A higher
resolution radio observation conducted by \cite{L13} at 610 \mega Hz and
2.1 \giga Hz later confirmed the identities of the NW and the SE relic, and
found another extended source of radio relic in the east (E) (See Fig.~\ref{fig:config}). Among the radio relics, the NW relic possesses the most extended geometry
(0.56 Mpc in length), and its physics, including the
polarization and Mach number were studied in the greatest detail. Such
information allows us to constrain the $\alpha$ and the merger scenario. The E relic
was also reported to have a resolved length of 0.27 Mpc, while the SE relic
was found to overlap with a point source \citep{L13}. Both the E and SE
relic are located closer to the SE DM subcluster, we therefore considered them to
originate from the same merger shock in the following work.

\section{METHOD -- Monte Carlo simulation} 
We used the collisionless 
dark-matter-only Monte Carlo modeling code written by \citetalias{D13}, to compute the physics of between
the first and second core-passage of the DM subclusters. 
In the D13 code, the time evolution of the
head-on merger was computed based on an analytical, dissipationless model
assuming that the only dominant force is the gravitational attraction from
the masses of two truncated Naverro-Frenk-White (hereafter NFW) DM halos. 
In the simulation, many realizations of the collision is
computed by drawing random realizations of the PDFs of the inputs, including
the data ($\vec{D}$) and one model variable, the projection angle between
the plane of the sky and the merger axis, $\alpha$. In particular,
the required data, included the masses ($M_{200_{NW}},M_{200_{SE}}$) the
redshifts ($z_{NW}, z_{SE}$) and the projected separation of the two
subclusters ($d_{proj}$).  (See Table~\ref{tab:inputs}
for quantitative descriptions of the sample PDFs) 
Each set of inputs is then used for computing the output variables
($\vec{\theta}^\prime$) by making use of conservation of energy to describe
their collision due to the mutual gravitational attraction.
To ensure convergence of the output PDFs, in total, 2 million realizations
were computed. 
The results, however, are consistent up to a fraction of a percent just
from 20 000 runs \citepalias{D13}. The random sampling allows us to
throughly explore the multidimensional input parameter space and account
for the uncertainties of the inputs at the same time.
\par    
We adopt a Bayesian interpretation of the PDFs of the Monte Carlo
simulation. The Bayes chain rule that underlies the simulation can be written as:
\begin{equation}
    P(M|\vec{D}) \propto P(\vec{D}|M)P(M),
\end{equation}
where the likelihood is defined to be the PDF of $\vec{D}$ given our
physical model $M$ which we parametrize using variables in Table 1
($\vec{\theta}$).
For example,the
calculation of the output variables of the $j$-th realization can be denoted as: 
\begin{equation}
    (\vec{\theta}^\prime)^{(j)} = f(\vec{\theta}^{(j)}, \vec{D}), 
\end{equation}    
and computed over all $j$ realizations. Finally, we took the physical
constraints on the dynamical variables into account by
examining the resulting physical variables against the physical limits and
excluding realizations that would produce impossible values. We refer to this
process of excluding unphysical realizations as applying priors. 
Even though we denoted the priors for one dimension at a time (See Appendix~\ref{app:priors}), 
the correlations between different variables are properly taken into account
since we discarded all the variables of the problematic
realizations.\par 
The system of El Gordo satisfies several major assumptions in the Monte Carlo
simulation.
One of the strongest assumptions is that all realizations correspond to
gravitationally bound systems. The simulation excludes all realizations
that would result in relative collisional velocities of the subclusters
higher than the free-fall velocity. We justify our assumption of only
modeling gravitationally bound system by noting that the relative escape
velocity of the subclusters for El Gordo is
$4500~\kilo\meter~\second^{-1}$ (based on the mass estimates of
\cite{Jee13}). Studies from cosmological simulations giving the PDFs of the pairwise velocities of massive merging clusters ($>
10^{15} M_{\sun}$) indicate that it is highly unlikely that the pairwise
velocities would be $> 3000~\kilo \meter~\second^{-1}$ under $\Lambda$CDM.
(\citealt{Thompson12}, \citealt{Lee2010}).  Other major assumptions for
modeling systems with this code include negligible impact parameter. There
is a study indicating that the impact parameter of El Gordo
may be as large as $300$ kpc $ \approx 40\%~r_s$ \citep{Molnar14},
where $r_s$ is the characteristic core
radius of the NFW halo with the mass of the SE subcluster. According to
\cite{Ricker98}, the resulting remnant of bimodal cluster mergers would
have drastic differences only when the impact parameter $> 10~ r_{\text{core}}$.
\cite{Mastropietro2008a} also reported that an impact parameter of $0.1~
r_{200}$ affected merger dynamics only at the $\sim$10\% level.   
Other assumptions in this simulation include negligible dynamical friction
during the merger, negligible mass accretion and negligible self-interaction
of dark matter. Discussion of the effects of each of these assumptions are
included in \citetalias{D13}.  
\par
\subsection{Inputs of the Monte Carlo simulation} \label{sec: inputs}
\setcounter{table}{0} 
\setcounter{table}{0}
\begin{table}
\caption{Properties of the sampling PDFs of the Monte Carlo simulation} 
\begin{center} 
\begin{tabular}{@{}lcccc}
\hline \hline Data & Units & Location & Scale & Ref\\ \hline
$M_{200c_{\mathrm{NW}}}$ & $10^{14} h_{70}^{-1}$ M$_{\odot}$ &10.0&1.6& \citetalias{Jee13}\\ 
c$_{\mathrm{NW}}$ &  & 2.50& 0.02& \citetalias{{Jee13}} \\ 
$M_{200c_{\mathrm{SE}}}$ & $10^{14} h_{70}^{-1}$ M$_{\odot}$ &8.0&1.2 & \citetalias{Jee13}\\ 
$c_{\mathrm{SE}}$ &  & 2.70 & 0.04& \citetalias{Jee13}\\ 
$z_{\mathrm{NW}}$ &  & 0.86842 & 0.00109& \citetalias{M11}, \citetalias{Sifon13}\\ 
$z_{\mathrm{SE}}$ &  & 0.87110 & 0.00117& \citetalias{M11}, \citetalias{Sifon13}\\ 
d$_{\mathrm{proj}}$ & Mpc & 0.75 &0.07 & \citetalias{Jee13} \\ 
\hline 
\end{tabular} 
\end{center} 
\label{tab:inputs} 
\end{table} 

\subsubsection{Membership selection and redshift estimation of subclusters}
We adopted the identification of galaxy membership of El Gordo given by
\citetalias{M12} with a total count of 89 galaxies.
To further distinguish member galaxies of each subcluster, we first
converted the coordinates (01:03:22.0,
-49:12:32.9) and (01:02:35.1, -49:18:09.8) to pixel space to
avoid anarmorphic distortion. Then we performed a spatial cut using the
aforementioned two points in pixel space as the two ends of the cut. The
spatial cut is approximately perpendicular to the 2D merger axis and is consistent with
the bimodal number density contours (See Figure~\ref{fig:membership}). 
There are 51 members identified in the NW subclusters and 35 members in the SE
subclusters. 
After identifying members of each subcluster, we performed 10, 000 bootstrap realizations to estimate the biweight
locations of the spectroscopic redshifts of the respective members in order
to obtain the samples of the PDFs of the redshifts of each subcluster. 
The spectroscopic redshift of the subclusters were
determined to be 
%$z_{total} = \pm $ 
$z_{\mathrm{NW}} = 0.86842 \pm 0.0011$ and 
$z_{\mathrm{SE}} = 0.87131 \pm 0.0012$, where the quoted numbers represent the
biweight location and 1$\sigma$ bias-corrected confidence level
respectively \citep{Beers90}.  
Both the estimated redshifts of the subclusters and the uncertainties are
consistent with estimates of $z=0.8701 \pm 0.0009$ for El Gordo given by \citealt{Sifon13}, and the fact that the
member galaxies of El
Gordo shows large velocity dispersion, i.e. the largest velocity
dispersion among all the ACT galaxy clusters, as reported by
\citetalias{M12}.

We estimated the radial velocity differences of the
subclusters by first calculating the velocity of each subcluster with
respect to us, using  
\begin{equation}
	v_i = \left[ \frac{(1+z_i)^2 - 1 }{(1+z_i)^2 + 1 }\right]c,
\end{equation}
where $i=1, 2$ represents the two subclusters, and $c$ is the speed of
light. The radial velocity was calculated by: 
\begin{equation}
	\Delta v_{rad}(t_{obs}) = \frac{|v_2 - v_1|}{1-\frac{v_1 v_2}{c^2}}.
\end{equation}
Due to the estimates of the subcluster redshifts are close to
one another with overlapping confidence intervals, we obtained a low 
radial velocity difference of the two subclusters to be
$476~\pm~242~\kilo\meter~\second^{-1}$ (See Fig. \ref{fig:
bootstrap_redshift}). 
The radial velocity difference of $586~\kilo \meter~\second^{-1}$ reported by \citetalias{M12} 
is higher than our estimates but within the 68\% bias-corrected
confidence interval. 
%Limitations and possible improvements of this analysis
%of $v_{rad}$ are provided in the discussion. 
\begin{figure}
	\includegraphics[width = \linewidth]{confirmed_member_divide.png}
	\caption{\label{fig:membership} Points showing the locations of the
	member galaxies and the division of the member galaxies among the two subclusters of El Gordo by a spatial cut
(black line). The color of the points shows the corresponding spectroscopic
redshift of the member galaxies (see color bar for matching of
spectroscopic values), with the redder end indicating higher
redshift. The background number density contours in green indicate a bimodal
distribution.} 
\end{figure}
\begin{figure}
	\includegraphics[width = \linewidth]{bootstrapped_redshift.png}
	\caption{\label{fig: bootstrap_redshift} Bootstrapped location of the
	redshift estimates and $v_{rad}$ estimates for each subcluster using the
	selected spectroscopic members. The shaded histograms approximate the
	PDFs from the bootstrapping procedure.
} 
\end{figure}
\subsubsection{Weak lensing mass estimation} 
We obtained 40, 000 samples of the joint PDFs of the masses of the two dark
matter halos as the outputs of the Monte Carlo Markov Chain (MCMC)
procedure from \citealt{Jee13}. Discussion of the handling of the weak
lensing source galaxies and the details the MCMC procedure for mass
estimation can be found in \citealt{Jee13}. 
\subsubsection{Estimation of projected separation ($d_{proj}$)} 
To be consistent with our MCMC mass inference, our Monte Carlo simulation takes 
the projected separation of the NFW halos to be those of the inferred
DM centroid locations in \citealt{Jee13}. We draw random samples
 of the location of centroids from two 2D Gaussians centered at
 RA$=01:02:50.601$, Decl.=$-$49:15:04.48 for the NW subcluster and RA =
 01:02:56.312, Decl.=$-$49:16:23.15 for the SE
subcluster, with a 1'' standard deviation each as estimated from the
convergence map of \citet{Jee13}. Or equivalently, the
the inferred centroid locations correspond to a mean projected separation
($d_{proj}$) of $0.74\pm {0.007}$ Mpc. 
\subsection{Outputs of the Monte Carlo simulation}\label{sec: outputs}
%We outline the outputs of the simulation here to facilitate the discussion
of the design of the priors used in the simulation. The simulation
provides PDF estimates for many of the output variables. Variables
of the most interest include the time dependence and $\alpha$, which is
defined to be the projection angle between the plane of the sky and the merger axis. Other output variables are dependent on $\alpha$ and the time
dependence. Specifically, the simulation denotes the time dependence by
providing several characteristic time-scales, including the time
elapsed between the collision and when the subclusters first reach apoapsis
($T$) and the time-since-collision.  

The two versions of the time-since-collision variables $TSC_0$ and
$TSC_1$ denotes different possible merger scenarios. 1) We call the scenario for which the subclusters are
moving apart after collision to be ``outgoing" and it corresponds to the
smaller $TSC_0$ value, and 2) we call the alternative scenario 
``incoming" for which the subclusters are approaching each other after turning
around from the apoapsis for the first time and it corresponds to $TSC_1$.
We describe how we use to break the degeneracies of the two scenarios in
section \ref{sec: positionprior}. 
 
The simulation also output estimates of variables that characterize
the dynamics of the merger. The 3D velocities, both at the time of the
collision ($v_{3D}(t_{col})$) and at the time of observation
($v_{3D}(t_{obs})$) are provided. The maximum 3D separation ($d_{max}$),
which is defined to be the distance between the position of collision to
the apoapsis, is also part of
the outputs. (See the lower half of Table \ref{tab:outputs} for all the outputs).
%Here we present results based on:\\  %1) a flat radio prior\\
%2) a uniform prior over a range of most likely 3D separations\\
%3) a Gaussian prior  
%We discuss in subsection \ref{sec:priors}  on the use of the default filters
%and two new filters designed according to the observed data and the physics of the radio relic.
%
%\textbf{While the underlying formalism of the Monte Carlo simulation is
%    based on the Bayes theorem, we caution the reader that this simulation
%    does not correspond to a conventional Bayesian parameter estimation but
%    more similar to the Bayesian uncertainty estimation method mentioned in Saltelli 
%    et al. (2004). (See appendix \ref{} for a more in-depth discussion)}



We outline the outputs of the simulation here to facilitate the discussion
of the design of the priors used in the simulation. The simulation
provides PDF estimates for 8 output variables. Variables
of highest interest include the time dependence and the angle $\alpha$, which is
defined to be the projection angle between the plane of the sky and the
merger axis. Other output variables are dependent on $\alpha$ and time. Specifically, the simulation denotes the time dependence by
providing several characteristic time-scales, including the time
elapsed between consecutive collisions
($T$) and the time-since-collision of the observed state ($TSC$).  

We provide two versions of the time-since-collision variables $TSC_0$ and
$TSC_1$ to denote different possible merger scenarios. 1) We call the scenario for which the subclusters are
moving apart after collision to be ``outgoing" and it corresponds to the
smaller $TSC_0$ value, and 2) we call the alternative scenario 
``returning" for which the subclusters are approaching each other after turning
around from the apocenter for the first time and it corresponds to $TSC_1$.
We describe how we make use of properties of the radio relic to evalute
which scenario is more likely in
section~\ref{sec:positionprior}. Evolution of the merger after the second
passage is not considered. Outputs from our dissipationless simulation for
a ``second'' passage will not differ from the first passage.
 
The simulation also output estimates of variables that describe
the dynamics and the characteristic distances of the merger. The relative
3D velocities of the subclusters, both at the time of the
collision ($v_{3D}(t_{col})$) and at the time of observation
($v_{3D}(t_{obs})$) are provided. The characteristic
distances included in the outputs are the maximum 3D separation ($d_{max}$),
which is the distance between the subclusters at
the apocenter and the 3D separation of the subclusters at observation
($d_{3D}$). 
\subsection{Design and application of priors} 
\label{sec:priors}
%The strength of the Monte Carlo simulation by \citetalias{D13} is its ability
to detect and rule out extreme input values that would result in
unphysical realizations via the application of prior probability. 
Our default Monte Carlo priors are described in D13 and in Appendix
\ref{app: results}. We also examine the effects of applying 
two priors derived based on the position and the integrated polarization
fraction of the radio relic of El Gordo respectively. 
We considered other properties of the radio relic to be used as prior
information, such as the physical location. Due to large uncertainty in the
projection angle that affects the prior information derived from the prior
information, we have not included the corresponding numerical results.
\par 
%(Not sure if the following fits best here but it is definitely an intro to
%the priors) 
El Gordo shows radio relics on the periphery of both subclusters
\citepalias{M11}. The
radio relic  of El Gordo was first mentioned in the Sydney University
Molonglo Sky Survey (SUMSS) data in low resolution at 843 MHz
\citep{Mauch03} as shown in M11. The higher resolution radio observation
conducted by \cite{L13} at 610 \mega Hz and 2.1 \giga Hz confirms that the identity of the radio relic
after removing effects of radio point sources. 
Three main sources of radio relic were identified, including the NW, SE and the
E relic. The NW radio relic possesses the most extended geometry among all
the identified relic source. We do not refer to the The SE nor the E radio
relic in our calculation since we do not have an estimation of the shock
speed of the SE relic nor the E relic from \citet{L13} for comparison.    
%Radio relics have been suggested to be able to constrain the mass ratios,
%the projection and the merger configuration \citep{vanWeeren10}. 

%%%%%%%%%% WAIT WHAT AM I TRYING TO SAY????
%Ever since the first detection of radio relic, cosmological hydrodynamical simulations of
%merging clusters have been used to model their emission spectrum and
%geometry. (\citealt{Vazza11}, \citealt{VanWeerenRJ2011}, Bonafede
%et al. 2013, \citealt{E98}, Br\"{u}ggen et al., Skillman et al.
%2013) While such cosmological simulations have provided valuable insights
%to verifying the physical models, they are expensive in terms of
%computational power and novel techniques have to be invented in order to
%analyze the large amount of simulated data so progress has been slow. 
%Our Monte Carlo simulation can make use of known physics combined with the
%preliminary results from such cosmological simulations to use properties of
%the radio relic to constrain merger dynamics. 

%Compared to hydrodynamical simulations or cosmological simulations, this
%    Monte Carlo simulation is not demanding in terms of CPU time, therefore, we
%    can run many realizations in order to probe how the input variables
%affect the output variables. 

%\begin{itemize}
%\item talks about the observable, which is the comoving kinetic power through each shock surface
%\item refer to diffusive shock acceleration (DSA) mechanism?
%\item Kang \& Jones treatment of Mach number-dependent efficiency
%considering the possibility of having an non-isotropic magnetic field  
%$KJ_BparallelRadial$ model
%\end{itemize}
%\begin{figure}
%	\includegraphics[width=\linewidth]{d_3d_prior1.png}
%	\caption{The marginalized output PDFs of the observed 3D separation
%		($d_{3D}$) 
%		with and without the radio prior applied. 
%		(maybe I should replot this more nicely without too many
%		distracting lines but only the lines showing the location)
%		%The vertical
%		%lines denote, dashed line: biweight location, dash-dot
%		%line: 68\% credible limit, dotted line: 95\% credible
%		%limit.
%	\label{fig:radioprior}}
%\end{figure}


%\subsubsection{Weighting function based on the observed position of the radio relic}\label{sec:relic} 
%-----------------------------------------------------------------------
%Among the known galaxy cluster mergers that are associated with radio
%relics, \cite{Vazza12} noted that most of them have radio relic located
%more than 800 \kilo pc away from the merger center. \cite{Vazza12} then
%conducted hydrodynamical simulations of twenty of known galaxy mergers with 
%radio relic to investigate this observed trend. They found a radial
%trend of kinetic power dissipation increasing up to around half the virial
%radius (r$_{vir}$) of the cluster. Summarizing the results from the proposed model for energy dissipation of the radio relic, \cite{Vazza12} gives the range of highest kinetic power emission in a range of 
%$.2 ~r_{\mathrm{vir}} < d_{\mathrm{3D}} < .5~r_{\mathrm{vir}}$.
%
%
%% In particular, Vazza et al. (2011) showed dependence of observed
%%location of radio relic: when the clusters are at small separation, the
%%Mach number is too high for a radio shock to form and the steep fall off of
%%the emission power of the radio relic as a function of separation makes it
%%difficult to observe a radio relic when it has propagated beyond a certain
%%separation.  
%
%%\textbf{We take into account the uncertainties of their modeling and 
%% construct prior probability on a range of 3D separation for which the kinetic
%%power dissipation of the radio relic is more than 10\% of the peak value.} 
%Since we do not have information on how the probability of
%being able to observe the relic would fall off as a function of emission power, we adopt a conservative approach and designed a uniform prior. 
%%and contrast that to a flat prior to test the effect of the prior on the output variables. 
%We also take into account the uncertainties of the different proposed power
%emission model and come up with a prior of:
%\[
% \text{P}({d_{3D}}) = 
%\begin{cases} 
%\text{constant,} & \text{for 1.0  Mpc} < d_{3D}(t_{\mathrm{obs}}) < 3.0 \text{ Mpc}\\
%0, & \text{otherwise}
%\end{cases}
%\]
% for El Gordo.\par 
%%\begin{equation}  
%%P(d_{3D}(t_{\mathrm{obs}})) = 
%%\begin{cases}
%%1/C, \text{ if }1.0 \text{ Mpc } < d_{3D}(t_{\mathrm{obs}}) < 3.0 \text{ Mpc} \\ 0, \text{ otherwise}
%%\end{cases}
%%\end{equation}
%
%\begin{figure}
%	\includegraphics[width=\linewidth]{alpha_pdf_prior_diff.png}
%	\caption{The projection angle with and without the radio relic
%prior applied. (Needs to update and label the figure better)} 
%\end{figure}


One of the biggest strengths of the Monte Carlo simulation by \citetalias{D13} is its ability
to detect and rule out extreme input values that would result in
unphysical realizations via the application of prior probability. 
Our default priors are described in D13 and we include them in
Appendix~\ref{app:results} for the convenience of the readers. 
In addition, we have devised a new prior on the projection angle $\alpha$
based on the polarization fraction of the radio relic.

%---------------------------------------------------------------------------
\subsubsection{Monte Carlo prior based on the polarization fraction of the radio relic}
%We can relate the polarization fraction of the radio relic to the
projection angle by examining the
generating mechanism of the radio relic.
The observed radio relic is due to synchrotron emission of free electrons in a
magnetic field. If the magnetic field is uniform, the observed
polarization fraction of the synchrotron emission of the electrons depends on the
viewing angle (or equivalently the projection angle) with respect to the alignment of the magnetic field. 
Synchrotron emission from electrons inside unorganized magnetic field are
randomly polarized. The high reported integrated polarization fraction from
\citet{L13} can be explained by a highly aligned magnetic field,
created by the compressed intracluster medium during a merger
(\citealt{E98}, \citealt{vanWeeren10}, \citealt{Feretti12}).
This picture is consistent with a high polarization fraction perpendicular
to this magnetic field along the relic. 
\par
We designed our prior to reflect how $\alpha$ decreases monotonically as the
maximum observable integrated polarization fraction. 
\begin{figure}
	\includegraphics[width=\linewidth]{Ensslin_polar_fig.pdf}
	\caption{Predictions of polarization percentage of the radio relic at a
		given projection angle from different models, reproduced from
		\citealt{E98}. Each model assumes electrons producing the radio emission
		to be accelerated inside
		uniform magnetic field of various strengths ({\it strong} or {\it weak}). The curves are plotted with spectral index of the radio emission
		($\alpha_{radio}$), spectral index of the electrons ($\gamma$) and
		compression ratio of the magnetic field ($R$) corresponding to the
		estimated values from \citet{L13}.
		We highlight the observed polarization percentage of the main NW radio relic
		of El Gordo by the dotted vertical line with the greyed out region
		indicating the uncertainty \citep{L13}.\label{fig:Ensslin_fig}}
\end{figure}
This assumption is based on the class of models given by \cite{E98}(See
Figure~\ref{fig:Ensslin_fig}). In particular, we refer to a model from \cite{E98} that would give the most
conservative estimate on the upper bound of $\alpha$:
\begin{align}
\alpha &= 90 \degree -
\arcsin
\left(
\sqrt{
\frac{
	\frac{2}{15} \frac{13R - 7}{R - 1} \frac{\gamma + 7/3}{\gamma + 1}
	\langle P_{strong} \rangle}{
	1 + \frac{\gamma + 7/3}{ \gamma +1} \langle P_{strong} \rangle }}\right)
\end{align}


This model
corresponds to the strong field case with the relic being supported by
magnetic pressure only, with $\alpha_{radio} = 0.86$, compression ratio
$R=2.7$ and $\gamma = 2.7$. 
This model predicts a maximum integrated polarization fraction of
$\sim60\%$ when $\alpha \rightarrow 0$. From this model, the observed integrated
polarization fraction of $33\%\pm1\%$ corresponds to an estimated value
of $\hat{\alpha}
 = 35\degree$. 
%We consider 39\degree as an upper bound on the projection angle since this idealized model assume isotropic distribution of magnetic field and
%electrons. 
This  polarization fraction of $\sim 60\%$ predicted by \citep{E98} is
consistent with the upper bound of relic polarization fraction in cosmological
simulations \citep{S13}. No other model of the magnetic field should predict a higher polarization fraction, thus it is highly unlikely that we see 33\%
integrated polarization at $\alpha > 35\degree$.  
\par

We cannot rule out $\alpha \leq 35\degree$ as a result of possible
variations in the magnetic field. 
\cite{E98} assumes an isotropic distribution of electrons in an isotropic magnetic field. Cosmological
simulations of radio relics from \cite{S13} show varying polarization
fraction across and along the relic assuming $\alpha = 0$, resulting in a
lower integrated polarization fraction. For example, it is possible to see a edge-on radio relic ($\alpha = 0$) with integrated polarization fraction of 33\%. 
Furthermore, \cite{S13} shows that after convolving the
simulated polarization signal with a Gaussian kernel of 4\arcmin~to
illustrate effects of non-zero beam size, the polarization fraction drops to between 30\% to
65\% even when $\alpha = 0$. 
Other uncertainties come from the fact that the inferred spectral indices
differ between the two observed frequencies and vary between the three
identified relic sources \citep{L13}. We examine the effects  of changing
the cutoff value of this prior to ensure the uncertainties do not
introduce significant bias in the estimated output variables and we
present the results in Appendix \ref{app: results}.
To summarize, we adopt a conservative uniform prior to encapsulate the
information from the polarization fraction of the radio relic as:
\begin{equation}
P(\alpha) = 
	\begin{cases}
	& \text{const. $>$ 0 for  }\alpha < 35 \degree \\ 
	& 0 \text{ otherwise}
	\end{cases}
\end{equation}


%Due to these likely variations in the true magnetic field, the true observable integrated polarization values at a given $\alpha$ can be lower than what is predicted by \cite{E98}. 
%For example, it is possible that the radio relic of El Gordo has a lower maximum face-on polarization fraction than 75\%, but if we are viewing the relic at a smaller $\alpha$, the integrated
%polarization fraction can still comes out to be 33\%.

%With simplifying assumptions, \cite{E98} have derived the integrated polarization fraction of a radio relic as a function of the viewing
%angle ($\delta = 90\degree - \alpha$).
% and the compression
%\~{R} of the magnetized region where the relic is generated. 
%. The simplifying assumptions, such as having an
%isotropic distribution of unshocked magnetic fields and electrons etc.,
%represents an idealized case showing maximum possible polarization fraction at a given $\alpha$.  
%% Cosmological simulations of radio
%relic \citep{S13} show a maximum integrated polarization fraction $\sim75\%$ at
%$\alpha = 0$ as predicted by \cite{E98}. 
%After accounting for different spectral
%indices and magnetic field strength, 
%The simplifying
%assumptions, such as having an isotropic distribution of unshocked fields
%and an isotropic distribution of electrons etc. \citep{E98}, gives
%polarization fraction as high as $\sim$ 75\% when $\alpha = 0$. 

%For an
%actual merger, the magnetic field can be less isotropic,  and the resulting polarization fraction at a given $\alpha$ would be lower. This postulate is backed up by the edge-on view of polarization fraction of simulated relics, such as the top left hand panel of figure 9 from Skillman et al. 2013.
%%This model, however, assumes an isotropic distribution of electrons in an isotropic magnetic field. \cite{E98}
%%These mathematical relationships underlies the design of this prior based on observed polarization fraction. The different cases that \cite{E98} considered have different magnetic field strengths and various spectral indices.
%We note that power of polarized synchrotron emission from relativistic electrons has a ratio of 7:1 between parallel polarization and perpendicular polarization. 
%Therefore,   
%
%\par
%\textbf{We pick a form of uniform prior, to represent
%the uncertainties in both the modeling (\citealt{E98}, \citealt{S13}) and the interpretation of the data from \cite{L13}.} 
%Following previous discussion, we pick a value of $\mu_\alpha =39\degree +
%2 \degree$ to filter realizations, i.e. we do not draw values of $\alpha >
%41\degree$. The extra $2 \degree$ in the prior is included to account for the uncertainty of the integrated polarization fraction reported by \cite{L13}. 
%
%For the width of fall off of the sigmoidal function, we pick
%$\sigma_\alpha = 1\degree$ that corresponds to the uncertainty of the
%integrated polarization fraction reported by \cite{L13}.    

%\begin{itemize}
%\item spectral index of ...
%\item During the merger process, the hot intracluster is cluster merger compresses the magnetic field and orders the polarization.    
%\item \cite{L13} reported that the polarization can constrain viewing angle to be $> 18 \degree $-- check if this viewing angle is defined the same way 
%\item Ensslin 's work which is an application of the theory of
%plane-parallel shock acceleration, which can be justified by the large
%radius of the shock sphere
%\item we consider the most conservative constraint that can be recovered
%from this model, which is strong/weak field case with a spectral index of
%$\alpha_{\text{spectral}}\sim 2$ combined with the observed mean
%polarization fraction of $P \sim 33.3\%$, we recover a  
%\item
%\end{itemize}
%\begin{equation}
%P(\alpha) = 
%\frac{1}{2} - \frac{1}{2} \text{erf}\left(\frac{1}{\sqrt{2}}\frac{\alpha -
%(\mu_\alpha+3\degree)}{\sigma_\alpha}\right)  
%\label{eqn:prior}
%\end{equation}
%
%\noindent See Appendix \ref{app:priors} for a plot of (\ref{eqn:prior}).

%The polarization information has larger constraining power than the .   
%To test the effects of applying the prior on the aforementioned range of
%separation,  we have come up two priors and applied them separately
%%Therefore, the distance between the subclusters, which has to be less than twice the 3D distance between the radio relic from the center of the cluster, is taken conservatively to be $1.0~\mega$pc $<$ d$_{\mathrm{3D}}
%%(t_{\mathrm{obs}}) < 3.0~\mega$pc. 
% to the 3D separation of the subclusters at the time of observation 
%($d_{\mathrm{3D}}(t_{\mathrm{obs}})$):
%
%The effect of the uniform prior is shown in Figure \ref{fig:radioprior}.
%
%%\textbf{description of the radio observation} 
%
%\textbf{how the distances were determined - overview of previous work}
%


We can relate the polarization fraction of the radio relic to the
projection angle by examining the
generating mechanism of the radio relic.
The observed radio relic is due to synchrotron emission of free electrons in a
magnetic field. If the magnetic field is uniform, the observed
polarization fraction of the synchrotron emission of the electrons depends on the
viewing angle (or equivalently the projection angle) with respect to the alignment of the magnetic field. 
Synchrotron emission from electrons inside unorganized magnetic field are
randomly polarized. The high reported integrated polarization fraction from
\citet{L13} can be explained by a highly aligned magnetic field,
compressing the ICM during a merger
(\citealt{E98}, \citealt{vanWeeren10}, \citealt{Feretti12}).
This picture is consistent with a high polarization fraction perpendicular
to this magnetic field along the relic. 
\par
We designed our prior to reflect how $\alpha$ decreases monotonically as the
maximum observable integrated polarization fraction. 
\begin{figure}
	\includegraphics[width=\linewidth]{Ensslin_polar_fig.png}
	\caption{Predictions of polarization percentage of the radio relic at a
		given projection angle from different models, reproduced from
		\citealt{E98}. Each model assumes electrons producing the radio emission
		to be accelerated inside uniform magnetic field of various strengths ({\it strong} or 
		{\it weak}). The curves are plotted with spectral index of the radio emission
		($\alpha_{radio}$), spectral index of the electrons ($\gamma$) and
		compression ratio of the magnetic field ($R$) corresponding to the
		estimated values from \citet{L13}.
		We highlight the observed polarization percentage of the main NW radio relic
		of El Gordo by the dotted vertical line with the greyed out region
		indicating the uncertainty \citep{L13}.\label{fig:Ensslin_fig}}
\end{figure}
This assumption is based on the class of models given by \cite{E98}(See
Figure~\ref{fig:Ensslin_fig}). In particular, we refer to a model from \cite{E98} 
that would give the most
conservative estimate on the upper bound of $\alpha$:
\begin{align}
\alpha &= 90 \degree -
\arcsin
\left(
\sqrt{
\frac{
	\frac{2}{15} \frac{13R - 7}{R - 1} \frac{\gamma + 7/3}{\gamma + 1}
	\langle P_{strong} \rangle}{
	1 + \frac{\gamma + 7/3}{ \gamma +1} \langle P_{strong} \rangle }}\right),
\end{align}

This model corresponds to the strong field case with the relic being supported by
magnetic pressure only, with the spectral index of the radio
emission being $\alpha_{radio} = 0.86$, the compression ratio of the
magnetic field being
$R=2.7$ and the spectral index of the electrons being $\gamma = 2.7$. 
This model predicts a maximum integrated polarization fraction of
$\sim60\%$ when $\alpha \rightarrow 0$. 
%We consider 39\degree as an upper bound on the projection angle since this 
% idealized model assume isotropic distribution of magnetic field and
%electrons. 
This  polarization fraction of $\sim60\%$ predicted by \citep{E98} is
consistent with the upper bound of relic polarization fraction in cosmological
simulations \citep{S13}. From this model, the
observed integrated polarization fraction of $33\%\pm1\%$ corresponds to an estimated value
of $\hat{\alpha}  = 35\degree$. 
No other model of the magnetic field should predict 
a higher polarization fraction, thus it is highly unlikely that we see 33\%
integrated polarization at $\alpha > 35\degree$.  
\par

We cannot rule out $\alpha \leq 35\degree$ because magnetic field
nonuniformities can lower the polarization below the Ensslin model value.
\cite{E98} assumes an isotropic distribution of electrons in an isotropic magnetic field. Cosmological
simulations of radio relics from \cite{S13} show varying polarization
fraction across and along the relic assuming $\alpha = 0$, resulting in a
lower integrated polarization fraction. For example, it is possible to see 
a edge-on radio relic ($\alpha = 0$) with integrated polarization fraction of 33\%. 
Furthermore, \cite{S13} shows that after convolving the
simulated polarization signal with a Gaussian kernel of 4\arcmin~to
illustrate effects of non-zero beam size, the polarization fraction drops
to between 30\% to 65\% even when $\alpha = 0$. We examined the effects of perturbing
the cutoff value of this prior to ensure the uncertainties do not
introduce significant bias in the estimated output variables in
section~\ref{sec:sensitivityTests}.
To summarize, we used a conservative uniform prior to encapsulate the
information from the polarization fraction of the radio relic as:
\begin{equation}
P(\alpha) = 
	\begin{cases}
	& \text{const. $>$ 0 for  }\alpha < 35 \degree \\ 
	& 0 \text{ otherwise}\label{eqn:polarprior}.
	\end{cases}
\end{equation}
We refer to equation \ref{eqn:polarprior} as the polarization prior. Unless
otherwise stated, the main results of the paper are obtained after applying
this polarization prior in addition to the default priors.

%%---------------------------------------------------------------------------
\begin{table*} 
\begin{minipage}{170mm} 
\caption{Table of the output PDF properties of the model variables and output variables from Monte Carlo simulation
\label{tab:outputs}}
\begin{tabularx}{\textwidth}{@{\extracolsep{\fill}}lccccccccc@{}}
\hline
\hline
&&&&Default priors & & & & Default + position priors  \\ 
\cmidrule{4-6} \cmidrule{8-10} 
Variables & Units && Location & 68$\%$ CI $^{\dagger}$ &95$\%$ CI && Location & 68$\%$ CI  & 95$\%$ CI \\ 
\hline 
$\alpha$ &(degree)&&43&19-69&6-80&&21&10-30&3-34\\
$d_{proj}$ &Mpc&&0.74&0.74-0.75&0.73-0.76&&0.74&0.74-0.75&0.73-0.76\\
$d_{max}$ &Mpc&&1.2&0.9-2.2&0.77-4.6&&0.93&0.81-1.2&0.75-1.9\\
$d_{3D}$ &Mpc&&1&0.79-2.1&0.75-4.3&&0.8&0.76-0.88&0.74-0.91\\
$TSC_0$&Gyr&&0.61&0.4-0.95&0.26-2.4&&0.46&0.3-0.55&0.21-0.64\\
$TSC_1$&Gyr&&1&0.77-1.7&0.63-4.4&&0.91&0.69-1.3&0.59-2.3\\
$T$&Gyr&&1.6&1.3-2.6&1.2-7.1&&1.4&1.2-1.6&1.2-2.4\\
$v_{3D}(t_{obs})$ & \kilo \meter~\second$^{-1}$ &&580&260-1200&59-2400&&940&360-1800&62-2900\\
$v_{rad}(t_{obs})$ & \kilo \meter~\second$^{-1}$ &&360&140-630&27-880&&310&110-590&8-840\\
$v_{3D}(t_{col})$ & \kilo \meter~\second$^{-1}$ &&2800&2400-3700&2100-4200&&2400&2200-2800&2100-3500\\
\bottomrule
\end{tabularx}\\
\footnotesize{$\dagger$ CI stands for credible interval}\\
\end{minipage}
\end{table*}

\subsection{Extension to the Monte Carlo simulation - Determining merger
scenario with radio relic position by model comparison}

One of the biggest questions involving the merger is whether El Gordo was
observed to be in a returning or outgoing phase. We compared the two merger
scenarios by making use of the observed projected separation of the relic from the
center of mass.
Simulations of cluster mergers such as the work of \citet{Paul2011b},
\citet{VanWeerenRJ2011}, and \citet{Springel2007} showed that, merger shock
fronts that may correspond to the radio relics 1) are generated near the
center of mass of the subclusters close to the time of the first
core-passage, 2) propagate outwards with the shock speed decreasing only slightly.
The propagation speed of the shock wave {\it with respect to the
center-of-mass} is reported to drop between 10\% to
30\% from private communication with Paul S. and $\sim 10\%$ from
\citet{Springel2007}. \par 
To capture the monotonically decreasing trend of the
propagation speed of the the shock fronts with respect to the center of
mass, we expressed the possible shock speeds as a factor of the inferred
collisional speed of the corresponding subcluster in the center of mass
(momentum) frame. 
Then we calculated how far the shock would have propagated for our inferred
$TSC_0$ and $TSC_1$ values. We worked in the center of mass frame where the
shock speed is expected to drop slightly with TSC. 
The projected separation of the shock is approximated as:
\begin{equation}
	%s_{proj} = \langle v_{relic} \rangle (\hat{t}_{obs} - \hat{t}_{col})
	s^j_{proj} \approx \langle v_{relic} \rangle^j (t^j_{obs} - t^j_{col})
	\cos(\alpha^j),
	\label{eq:proj_s_model}
\end{equation}
where the superscript $j$ of any variable denotes that the value of the
variable from the $j-th$ realization of the simulation, and $s_{proj}$ is the estimated projected
separation. We estimated the upper and lower bounds of the time-averaged velocity
$\langle v_{relic} \rangle$ of the shock between
the collision of the subclusters and the observed time as:  
\begin{align}
	\label{eqn:NW_speed}
	\langle v_{NW relic} \rangle^j &= \beta~v^j_{3D, NW}(t_{col}) \\
	&= \beta~v^j_{3D}(t_{col}) \frac{m^j_{SE}}{m^j_{SE} + m^j_{NW}}, 
\end{align}
where $0.7 \leq \beta \leq 1.5$ is a factor that we introduce to represent the
uncertainty of the velocity of the relic shockwave, and $v_{3D, NW}(t_{col})$ refers to the collisional velocity of
the NW subcluster in the center-of-mass frame as a comparison. 
Likewise, we have also computed the expected projected separation of the SE
relic using:  
\begin{equation}
	\label{eqn:SE_speed}
	\langle v_{SE relic} \rangle^j = \beta~v^j_{3D}(t_{col}) \frac{m^j_{NW}}{m^j_{SE} + m^j_{NW}}. 
\end{equation}
\par 
We examined the projected separation for a large range of $0.7 <\beta <
1.5$. This range of $\beta \approx 1$ allows us to use  
equations \ref{eqn:NW_speed} and \ref{eqn:SE_speed} to reflect that the
shock is driven by the merger. We note that the propagation speed of the
shock is also determined by the temperature, density and other details of
the gas medium (\citealt{Prokhorov2007}, \citealt{Springel2007},
\citealt{Milosavljevic07}), so the shock may propagate with $\beta > 1$ without violating any physics
laws. However, we note that merger shocks from cosmological
simulations are reported to show a low Mach number between 1 and 3
\citep{Bruggen2011}, with an even tighter upper limit on the Mach number of
$\sim 1$ reported for mergers of comparable masses \citep{Markevitch2007}.    
Simulation of the Bullet Cluster by \cite{Springel2007} also indicates that the
propagation velocity of the shock evolves such that $\beta \approx 0.95$ within
$\sim 0.4~\giga$yr after the collision. For the analysis of El Gordo,
we suggest $\beta \approx 0.9$ to be the most likely value given that the $TSC$ of
El Gordo is longer. 
\label{sec:positionprior}

\section{RESULTS} 
We found that the two subclusters collided with a relative velocity of $2400\pm^{900}_{400}~\kilo\meter~\second^{-1}$, at an estimated projection
angle of $\alpha = 21\degree\pm^{9}_{11}$. From our analysis of the two
scenarios, we found that El Gordo is more likely to be observed at a returning
phase with a estimate of $TSC_1 = 0.91\pm^{0.22}_{0.39}$ Gyr. We present an
overview of all the estimated variables in table~\ref{tab:outputs}, with
results only applying the default priors on the left hand side of the table
and those also applied with the polarization prior on the right hand side.
Furthermore, we include the plots of all the marginalized PDFs with the
polarization prior in Appendix~\ref{app:results}. \par 
Our estimates of $v_{3D}(t_{col}) = 2400\pm^{900}_{400}~\kilo\meter~\second^{-1}$ 
at the time of collision is compatible with the independent estimate from \citealt{L13}. 
By making use of the Mach number of the NW radio relic, \cite{L13}
reported an estimate of the upper bound of the relative collisional
velocity to be $2500
\pm^{400}_{300}\kilo\meter~\second^{-1}$. 
From the simulation of the Bullet Cluster, Springel \& Farrar (2007) showed
that pre-shock gas could travel at as high as $\sim
1100~\kilo\meter~\second^{-1}$ from the outer edge of the subclusters to
the merger center.  It is therefore reasonable to assume the reported speed
from \cite{L13} to be an estimate of the {\it relative collisional speed
between the subclusters}, instead of the collision speed of the NW
subcluster {\it relative to the center-of-mass}.
Magnitude of the relative $v_{3D}$ of the subclusters dropped as the
subclusters climbed out of the gravitational potential of each other, and
reduced to $v_{3D}(t_{obs}) = 940~\kilo\meter~\second^{-1}$ at
the time of observation.\par 
%The projection angle of the El Gordo is estimated to be
%$21\degree \pm^{13}_{17}$. Without the polarization priors, the Monte Carlo
%simulation gives the estimate of the projection angle as $43\degree^{26}_{24}$.
%Many previous studies (van Weeren et al. CITATIONS) have
%suggested that $\alpha$ should be small for the detection of double radio
%relics to be possible but did not provide quantitative constraints.    

\subsection{Time-since-collision (TSC) and the merger scenario}
The simulation gives two estimates for
the time-since-collision, with $TSC_0 = 0.46\pm^{0.9}_{0.16}~\giga \text{yr}$
and $TSC_1 =0.91\pm^{0.39}_{0.22}~\giga\text{yr}$. Both the estimates of
$TSC_0$ and $TSC_1$ are compatible physical time-scales of observable
features of El Gordo. Both estimates are lower than the approximate
observable time-scale of the wake feature in the X-ray, i.e.\ the sound
crossing time of $\sim2~\giga$yr. The observable time scale of the radio relics is also on the scale of
$\sim1~\giga$yr.\par 
Based on section~\ref{sec:positionprior}, we present the most likely
value of $\beta = 0.9$ in Fig.~\ref{fig:our_guessed_scenario} to show that
the returning case is preferred for both the calculations of the NW and the
SE relic. This conclusion favoring the returning case holds true for $\beta
< 1.1$, which corresponds to the time-averaged velocity of the relics at
$\langle v_{NW relic} \rangle < 1000~\kilo\meter~\second^{-1}$ and $\langle
v_{SE relic} \rangle < 1800~\kilo\meter~\second^{-1}$ in the center of
mass frame. For comparison purpose, we found that an extreme, and unlikely
range of $\beta > 1.5$ would be needed for the outgoing scenario to be
preferred. (See appendix~\ref{app:Bayes_factor} for plots of all the range
of $\beta$ that we examined). We marginalized $\beta$ to compute a Bayes
factor to compare the ratio between the likelihood of the returning case
and the outgoing case, and obtained a Bayes factor to be $\approx 1.6$ from
the NW relic and a Bayes factor of $\approx 460$ for the SE relic, showing
that our results favor the returning scenario despite the
uncertainties.
(See appendix~\ref{app:Bayes_factor}). Finally, we note that the estimate of NW shock velocity at $2500
\pm^{400}_{300}~\kilo\meter~\second^{-1}$ by \cite{L13} was reported with respect to
the turbulent ICM, not the propogation velocity with respect to center of
mass, so we have not made use of the estimate of \cite{L13} in this
calculation. \par
\begin{figure}
	\includegraphics[width=\linewidth]{our_guess.png}
	\caption{Comparison of the PDFs of the observed position of the NW relic (red bar
		includes the 95\% confidence interval of location of the NW radio relic in the center of mass frame) with the predicted position from the two simulated merger
		scenarios (blue for outgoing and green for the returning scenario).
	We made use of the polarization prior for producing this figure.} 
	\label{fig:our_guessed_scenario}
\end{figure}
\begin{figure}
	\includegraphics[width=\linewidth]{our_guess_SE.png}
	\caption{Comparison of the PDFs of the observed position of the SE relic (red bar
	includes the 95\% confidence interval of location of the radio relic in
the center of mass frame)}
\end{figure}
\begin{figure}
	\includegraphics[width=\linewidth]{TwoMnWBSG_2contour2d.png}
	\caption{The marginalized output PDF of the returning time-since-collision
($TSC_1$) vs. the 3D velocity at the time of collision for El Gordo. The
grey set of contours show the credible regions before applying the
polarization prior and the colored contours correspond to the credible
regions after applying the priors. The contours represents the 95\% and
68\% credible regions respectively. }
	\label{fig:TSC_v3D}
\end{figure}
\subsection{Sensitivity analysis of the polarization prior}
%We perform tests of how sensitive our parameters are to the design of our
prior filters.

\label{sec:sensitivityTests}
We performed tests of how each of the output variables vary according to the
choice of the cutoff of the polarization prior between
$\alpha_{\text{cutoff}} =
29 \degree$ to $49\degree$ instead of $35 \degree$.  
We found that in the most extreme case, choosing the cutoff values as $29
\degree$ ($-6 \degree$), the location of the $v_{3D}(t_{obs})$, is
increased by $ 16 \%$. While the $95\%$ CI of $d_{max}$ is
the most sensitive to the prior and it changes by
$\sim20 \%$ when $\alpha_{\text{cutoff}} = 49 \degree$. 
This shows that the exact choice of the cut off value for $\alpha$ for the
polarization prior does not change our estimates drastically.


%Both estimates are consistent with the estimate of $\alpha > 11.6 \degree$ from
%\citet{L13} based on the dynamics of the radio relic. 

%More work from N-body magentodynamical simulations is needed to better understand about the physics and the
%observable contraints of
%radio relics on merger dynamics. from cosmological simulations such as,  
%More observational constraints  
%
%(More speculative stuff should be put at the end.)
%From this simulation we have shown that it is possible to detect
%the double radio relics with $\alpha$ being as big as $61.14\degree$. 
%More concrete conclusions can only be drawn with better understanding of
%the radio relic properties from simulations and observations. 
%
%Low projection angle are rejected due to the assumption of a bound system
%
%\begin{itemize}
%\item explains that there hasn't been quantitative constraints on the angle
%for which double radio relic can be observed, even though that many studies
%have suggested that the detection of radio relic should imply that $\alpha$
%should be small. 
%From this simulation we have shown that it is possible to detect
%the double radio relics with $\alpha$ being as big as $61.14\degree$. 
%
%\item James' paper did mention how the mass estimation depends on $\alpha$,
%with the estimated mass being a lot smaller if $\alpha \ge 65\degree$. 
%However, since we did use the larger mass estimate as the input of this
%simulation, we can only say that the inferred $\alpha$ is consistent with
%the mass estimation. 
%% angle dependence of observation of radio relic,  this is the first estimate 
%% that provides an upper bound on the angle
%% for which we can observe double radio relic  
%\item discussion of the different scenarios mention in M12:
%1) we are viewing after core passage, but before first turn around, and
%the merger speed is low"\\
%2) the merger speed is high, but we are viewing after the first turn
%around as the two components come together for a second core passage
%\item discuss the inclination angle estimate from M12
%\item Dave: explain where the limits of the projection angle comes
%from. what observational evidence contradicts the low velocity
%scenario the most
%\end{itemize}


\section{DISCUSSION}
%-----------------------------------------------------------------------
\subsection{Comparison of our study with other studies of El Gordo}
We outline the qualitative agreement and disagreement between our
simulations and other hydrodynamical simulations of El Gordo such as
\cite{Donnert13} and \cite{Molnar14}. Our simulation focuses on giving PDF
estimates of particular dynamical and kinematic variables, whereas the
hydrodynamical simulations focused on understanding the detailed gas dynamics
required to reproduce the X-ray observables and Sunyaev-Zeldovich
observables of El Gordo. The goals,
assumptions, and initial conditions of \cite{Donnert13} and \cite{Molnar14}
are different from our simulations such that it is hard to come up with a
fair comparison. \par 
Both hydrodynamical simulations were based on a few sets of initial
conditions, instead of a thorough sampling of the inputs. For example, both
simulations made use of the mass estimates of from the dynamics analysis
of \citetalias{M12} at $m_{NW} = 1.9 \times
10^{15} M_{\sun}$,
which is larger than the upper 95\% CI of the mass that we used based on
the weak lensing estimate.
Furthermore, \cite{Molnar14} initialized the relative infall velocity
(velocity when the separation of subclusters equals the sum of the two virial
radii) to be $> 2250~\kilo \meter~ \second^{-1}$. This corresponds to
$v_{3D}(t_{col}) \gtrsim 4700~\kilo \meter~\second^{-1}$, which is close to
the escape velocity of the subclusters. From our simulation, there is a
negligible number of realizations with $v_{3D}(t_{col}) >
3000~\kilo\meter~\second^{-1}$. The range of projection angles suggested by
\cite{Molnar14} of $\alpha \gtrsim 45\degree$ is also excluded by our
polarization prior, whereas, we are unable to find information concerning
the projection angle of the simulation from \cite{Donnert13}, and we note
that \cite{Donnert13}
might have assumed $\alpha = 0$. \par 
With a time resolution of 0.25 $\giga$yr,
\cite{Donnert13} gave an estimate of  $T\approx 2~\giga$yr between the
first and second core-passage in Fig. 6 of their work, while our estimate gives $T
= 1.4\pm {0.2}~\giga$yr. 
By making use of the simulated X-ray luminosity and the projected separation
of $0.69$ Mpc, \cite{Donnert13} also reported their simulated work 
to best match  observation at $\sim 0.15~\giga$yr after collision. 
The $TSC_0$ from \cite{Donnert13} is below the estimated 95\% CI of
$TSC_0$ from our work and this might be due to their assumption of a zero
projection angle, it would take subclusters in our simulation with a
non-zero projection angle a longer time to travel the same projected distance. 
On the other hand, \cite{Donnert13} obtained a
relative collisional velocity between the subclusters at $\sim
2600~\kilo\meter~\second^{-1}$, which is compatible with our estimate of
$2400^{400}_{200}~\kilo\meter~\second^{-1}$.  
This agreement might be due to the similar assumptions of a low
energy orbit and a small impact parameter as the initial conditions in the
work of \cite{Donnert13} and our work.  


\subsection{Comparison to the merger scenarios of other merging clusters of galaxies}

\begin{figure}
	\includegraphics[width=\linewidth]{merger_scenario.png}
	\caption{Illustration of the different proposed merger scenarios, i.e. projected displacement
		($s_{proj}$) from the center of mass of different components along the
		merger axis, of El
		Gordo and the Bullet Cluster, at different stages of the mergers.  
		Shortly after the core-passage
		phase of the merger, the cool core had to travel through a dense gas region.
		Ram pressure ($=\rho v^2$) would strip the gas component from a subcluster and cause the
		cool core to lag behind the DM peak (This corresponds to the scenario
		for the Bullet Cluster indicated by the grey dotted line). As time went by, the DM and the cool core of
		a subcluster would propagate outwards while slowing down, they would
		encounter a region with much lower gas density. At this later stage, 
		the ram pressure from the lower density gas in the
		outskirt region may decrease drastically, so the cool core would seem
		to have received a kick by a slingshot. By this
		later stage, the DM component would have also reached the
		apocenter and started returning to the center of mass for a second
		core-passage. The cool core would then be able to travel further away from
		the center of mass than the DM component (El Gordo scenario indicated
		by the black dashed line).
	\label{fig:merger_scenario}}
\end{figure}
The hypothesis of El Gordo being in the outbound phase is more plausible when
we compare the details of the observables of El Gordo to the Bullet
Cluster (\citealt{Bradac2006b}, \citealt{Springel2007},
\citealt{Mastropietro2008a}).
A lot of the inferred properties are similar between the two clusters and
both clusters were observed in similar wavelengths. Both clusters are
considered as bimodal major mergers of subclusters of significant masses. The inferred
merger velocities are comparable at around $2600~\kilo\meter~\second^{-1}$
and $\alpha$ of both clusters are around $20 \degree$. 
In particular, the inferred outbound $TSC_0 / T$ of the Bullet Cluster and El Gordo
are similar. If instead, the El Gordo
is in the outbound phase of the merger (i.e. $TSC_0$ for El Gordo is
invalid) while the Bullet Cluster is in the inbound phase, the differences
in the observables of El Gordo and the Bullet Cluster can be explained.\par
First, the merger shock front of the Bullet Cluster is
observed only in the X-ray, meaning that the shock may not have the
time to propagate to the outskirt of the cluster (\citealt{Bruggen2011},
\citealt{Markevitch2007}), and this bow shock is indeed observed to closely
lead the corresponding less massive subcluster by $\sim 0.08~\mega$pc,
assuming they are propagating outwards. On the other hand, indirect observables of the merger
shocks of El Gordo can only be detected through the radio relic, and the shock is
further offset from the corresponding subcluster ($\sim 0.5~\mega$pc) and
the cool core ($\sim 0.4~\mega$pc). \par
%(Should discuss the Mach number,
%the speed of sounds in each cluster, and the DM velocities)  
Second, for the Bullet Cluster, the cool core (or the bullet) is closer to the
interior of the cluster than the corresponding less massive DM subcluster mass
peak, whereas, the cool core of El Gordo is further offset from the center of
mass than the corresponding SE subcluster (See Fig.~\ref{fig:config} for
the observed positions). 
Both configurations of cool core relative to the subcluster mass peak are
mentioned in \cite{Markevitch2007}, with the case of the Bullet Cluster
explained by the ram pressure stripping effect , and the case of El Gordo
explained by the ram pressure slingshot effect, which only occurs at a
later stage of a merger (See Fig. \ref{fig:merger_scenario} for an
illustration depicting this conjectured scenario).\par 
Simulation of a major merger by
\cite{Mathis05} with comparable mass ($1.4 \times 10^{15} M_{\sun}$) and
mass ratio (1:1) as El Gordo supports
our proposed scenario: it shows the turn-around of the cool core can occur after the
apocenter of the DM component, resulting in the cool core being further
away from the center of mass than the dark matter by as much as $\sim
0.2~\mega$pc.  The gas northwest of the cool core of El Gordo shows a comet-like
morphology with two tails that suggests outbound motion of the cool core, which may seem
contradictory to the returning scenario. However, from our proposed merger
scenario of El Gordo in Fig. \ref{fig:merger_scenario},
it is possible that the cool core and the DM are observed to be moving in
opposite directions, with the DM subcluster started returning for a second core-passage. If the returning scenario is true, El Gordo would be one of the first
clusters to be observed at a late stage of the merger, after another
bimodal cluster merger A168 with a cool front leading the corresponding DM
subcluster \citep{Hallman04}.

\subsection{El Gordo as a probe of dark matter self-interaction}
El Gordo possesses a range of special properties that makes it a promising
probe of self-interaction of DM. Its high mass ensure high DM
particle density for interactions during the high-speed core-passage. Its bimodal configuration makes it
relatively simple to interpret the offset and dynamics of the different
components. The observation of the radio relic has enabled us to
constrain the projection angle and reduce uncertainties of other dynamical
parameters. Furthermore, El Gordo is likely to be a late-stage merger
unlike other well studied clusters such as the Bullet Cluster. This gives
us a better picture of how a bimodal merger would behave at a later stage of a merger. \par 
This special merger scenario of El Gordo also raises a question: what phase
of a merger or what type of mergers would allow the most stringent
constraints on the self-interaction cross section of DM ($\sigma_{\text{SIDM}}$)? 
The use of merging clusters as probes of $\sigma_{\text{SIDM}}$. 
has been proposed and used in various literature.
(\citealt{Markevitch2004}, \citealt{Randall2008d}, \citealt{Merten2011},
\citealt{Dawson12}). One common theme among such work is
to make use of the observed offsets of the different components of the
merging clusters for the estimation. One of the most popular methods proposed by
\citealt{Markevitch2004} (i.e. method 1 in the paper) assumes the
corresponding the gas component would lag behind the corresponding DM
subcluster along the direction of motion due to ram pressure stripping.   
%and make use of the scattering depth $\tau_s = 1$ as an upper
%limit of the scattering depth of DM for estimating
%$\sigma_{\text{SIDM}}$. 
For El Gordo, since the cool core is further away from the
center of mass than the SE DM centroid, it is apparent that this particular
method does not apply. However, from a preliminary analysis of
the galaxy number density map of El Gordo \citepalias{Jee13}, there is a
noteworthy $\sim0.2~\mega$pc offset between the SE galaxy number density
peak and the SE DM centroid. This galaxy-DM offset may provide an even
better constraint than the gas-DM offset method. Before there is a
conclusion about how the galaxy-DM offset would vary over
the different phase of the merger Fig.~\ref{fig:merger_scenario}, it is
helpful to have some understanding about clusters at a late-stage of the
merger. 
\par 
%For example, another method for constraining $\sigma_{\text{SIDM}}$ is to
%compare the magnitude of the observed offset of galaxy and the corresponding DM
%subcluster peak using simulations with different assumed
%$\sigma_{\text{SIDM}}$ \citep{Randall2008d}. 
%the Musketball Cluster (\citealt{Dawson12}, \citetalias{D13}). 
%
%%In addition to the default prior, \citetalias{D13} applied a temporal prior
%%based on the observed X-ray luminosity boost for the Bullet Cluster, while
%%only the default priors were suitable to be applied for the Musketball.
%%Therefore, with the additional priors, both the Bullet Cluster and El Gordo
%%show a more constrained parameter space.
%%\begin{equation}
%%	\text{scattering rate}\propto v 
%%\end{equation}
%
%The three output variables that we compare among the listed clusters include $TSC$,
%$v_{3D}$ and the masses. These variables can affect
%how strongly the self-interaction of dark matter would manifest itself in
%case the DM self-interaction cross section is not zero. A higher TSC might
%allow more time for DM to interact with itself; a higher $v_{3D}$ might
% and a higher mass would correspond to
%higher central density for a higher self-interaction rate (surface mass
%density $\Sigma$). Although El
%Gordo is the most massive among the listed clusters, the higher mass did not result in a higher $v_{3D}(t_{col})$.
%The estimate of $v_{3D}(t_{col}) = 2400~\kilo\meter~\second^{-1}$ is
%comparable to the inferred $v_{3D}(t_{col}) =
%2800~\kilo\meter~\second^{-1}$ of the Bullet Cluster. 
%Both the phase $TSC_0  \approx 40\%~T$ and the $TSC_0 = 0.6~\giga$yr are
%similar between El Gordo and the Bullet Cluster
%It is remained to be investigated if our hypothesis about the relationship
%between how these variables would affect our chance of seeing effects from
%self-interaction of dark matter. 
%
%
%Merging clusters of galaxies are unique probes of the self-interaction of
%dark matter from the separation of different components. In particular, it
%is interesting to examine the scattering probability of dark matter by other dark matter. 
%From \cite{Randall2008d}, the scattering probability of each dark matter
%would be proportional to the local dark matter density and the relative
%velocities. 
%
%
%El Gordo is one of the most massive merging galaxy clusters. 
%
%This is further complicated by the likely returning scenario of El Gordo.
%time since collision 
%
%It is interesting to compare 
%We compare El Gordo to both the Bullet Cluster and the Musketball
%Cluster, which were studied by \citetalias{D13} using the Monte Carlo simulation.

%
\section{SUMMARY} 
We provide estimates of the dynamical parameters of El Gordo using Dawson's
Monte Carlo simulation, in particular, we 
\begin{enumerate}
	\item demonstrated the first use of polarization fraction information from
		the radio relics to reduce our estimates of the projection angle from
		$43\degree^{26}_{24}$ to $21 \degree \pm^{9}_{11}$ (See	Fig.~\ref{fig:geom_geom}). By performing sensitivity analysis, we showed that this prior helps reduce uncertainty for the dynamical variables without changing the estimates drastically\\ 
	\item inferred a {\it relative}\/collisional velocity between the subclusters of ElGordo as $2400\pm^{400}_{200}~\kilo\meter~\second^{-1}$ \\ 
	\item showed that a returning scenario is favored if the $\langle v_{NW relic}\rangle \leq
		1000~\kilo
		\meter~\second^{-1}$ and $\langle v_{SE relic}\rangle \leq
		1800~\kilo\meter~\second^{-1}$between the collision and the observation in the
		center-of-mass frame. It takes
		an unlikely high speed of $\langle v_{relic} \rangle \gg 1.5~v_{3D,
		sub}(t_{col})$ for the outgoing scenario to be favored.  
	\item made used of a simplified illustration to show how our inferred
		returning scenario may explain the unexpected location of the cool
		core and still be consistent with the wake / gas-tail morphology of the cool core. 
\end{enumerate}
As large scale sky surveys come online, more cluster mergers at late
stages of their merger would be discoverd. El Gordo will serve as one of
the best studied example of a late-stage cluster merger for comparison.  
In addition, our work has allowed us to examine
what information would be needed to better understand the merger
dynamics and scenario. Important questions concerning merging galaxy
clusters pending for answers include:  
\begin{itemize}
\item What are the typical propagation velocities of the shockwave that
corresponds to the radio relic in the	center of mass frame of the cluster?
\item What physical properties of the DM subclusters would correlate the
	best with the time-evolution of the propagation velocity of the shockwave?  
\item What is the typical duration
after the merger for which radio relics are observable in terms of the merger
core-passage time-scales? 
\item how generalizable is the merger scenario in Fig.
	\ref{fig:merger_scenario} is?  
\item how would the galaxy-DM offset evolve if we were to add that information
	to Fig. \ref{fig:merger_scenario}?
\end{itemize}
We urge simulators to help come up with answers that would be directly
comparable with data. Metastudies of merging clusters in multiple
wavelengths supported with simulations such as this work would allow us to
understand the merger clusters well enough to push the bounds on
$\sigma_{\text{SIDM}}$. \par 
\section{ACKNOWLEDGEMENTS}
We thank Franco Vazza, Marcus Br\"{u}ggen and Surajit Paul for sharing
their knowledge on the simulated properties of radio relic and merger
shocks. We extend our gratitude to Reinout Van Weeren for first proposing the use of
radio relic as prior. We appreciate the comments from Maru\v{s}a
Brada\v{c} about using the position of the relic to break degeneracy
of the merger scenario. KN is grateful to Paul Baines and Tom Loredo for
discussion of the use of Bayesian priors and sensitivity tests. 
Part of this work was performed under the auspices of the U.S. DOE by LLNL
under Contract DE-AC52-07NA27344. We would also like to thank 
GitHub for providing free repository for version control for our data
analysis code, and open source
packages developers of Ipython notebook, Ipython, astroML  and astropy.
(should provide proper citations)
\bibliographystyle{mn2e}
\bibliography{bib}
\appendix
\section{DEFAULT PRIORS USED FOR DAWSON'S MONTE CARLO SIMULATION}
\label{app:priors}
The default prior probabilities that we employed can be summarized as
follows for most of the output variables: 
\begin{equation}
	P(v_{3D}(t_{col}) | \alpha, v_{proj}(t_{col})) = 0\text{ if }v_{3D}(t_{col}) >
	v_{\text{free fall}}. 
\end{equation}
\begin{equation}
	P(TSC_0) = 
	\begin{cases}
		& \text{const}~\text{if }TSC_0 < \text{age of universe at } z=0.87	\\
		& 0~\text{otherwise}.
	\end{cases}
\end{equation}
In addition, we apply the following prior on $TSC_1$ only when evaluating the
statistics of $TSC_1$, thus allowing realiziations with a valid
outgoing TSC but an invalid returning $TSC_1$. 
\begin{equation}
	P(TSC_1) = 
	\begin{cases}
		& \text{const}~\text{if }TSC_1 < \text{age of universe at } z=0.87	\\
		& 0~\text{otherwise} \label{eqn:TSM_1}.
	\end{cases}
\end{equation}
To correct for observational limitations, we further convolve the
posterior probabilities of the different realizations with 
\begin{equation}
	P(TSC_0 | T) = 2 \frac{TSC_0}{T},
\end{equation}
to account for how the subclusters move faster at lower $TSC$ and thus it
is more probable to observe the subclusters at a stage with a larger $TSC$.
\par 
\section{PLOTS OF OUTPUTS OF THE MONTE CARLO SIMULATION}
We present the PDFs of the inputs of the dynamical simulation and the
marginalized PDFs of the outputs after applying the polarization prior in
addition to the default priors. See Fig. \ref{fig:plot_config} for explanations of
the order that we present the figures containing the PDFs . 
\begin{figure}
	\begin{center}
	\includegraphics[width=\linewidth]{ElGordo_plot_config.png}
	\end{center}
	\caption{Matrix of variables used in the simulations. We present them in
	4 categories, including, inputs, geometry, time and velocity. Regions of
	the same color represent one plot and the number
indicates the corresponding figure number in this appendix.
\label{fig:plot_config}
}
\end{figure}
\label{app:results}
%%%%%%%%%%%%% TASK --- 
\clearpage
\begin{figure*}
	\begin{minipage}{180mm}
	\begin{center}
	\includegraphics[width=0.65\linewidth]{TwoMnWBSG_inputsVsinput.png}
	\caption{Marginalized PDFs of original inputs (vertical axis) and the inputs after
applying polarization prior and default priors (horizontal axis). The inner and outer contour
denote the central 68\% and 95\% credible regions respectively.
The circular contours show that the application of priors did not introduce
uneven sampling of inputs. }
	\end{center}
	\end{minipage}
\end{figure*}
\begin{figure*}
\begin{minipage}{180mm}
	\begin{center}
	%\vspace{200px}
	\includegraphics[width=0.5\linewidth]{TwoMnWBSG_tri_geo.png}
	\caption{One-dimensional marginalized PDFs (panels on the diagonal) and
		two-dimensional marginalized PDFs of variables
		denoting characteristic distances and projection angle of the mergers.
	\label{fig:geom_geom}
	}
	\end{center}
	\end{minipage}
\end{figure*}
\begin{figure*}
\begin{minipage}{180mm}
	\begin{center}
	%\vspace{200px}
	\includegraphics[width=0.7\linewidth]{TwoMnWBSG_geoVsinputs.png}
	\caption{Marginalized PDFs of characteristic distances and projection
		angle of the merger and the inputs of the simulation.}
	\end{center}
	\end{minipage}
\end{figure*}
\begin{figure*}
\begin{minipage}{180mm}
	\begin{center}
	%\vspace{200px}
	\includegraphics[width=0.5\linewidth]{TwoMnWBSG_tri_time.png}
	\caption{One-dimensional PDFs of characteristic timescales of the simulation
(panels on the diagonal) and the marginalized PDFs of different
timescales. Note that there is a default prior for constraining $TSC_0$ but
not for $TSC_1$ and $T$, so $TSC_0$ spans a smaller range.}
	\end{center}
\end{minipage}
\end{figure*}
\begin{figure*}
\begin{minipage}{180mm}
	\begin{center}
	%\vspace{200px}
	\includegraphics[width=0.5\linewidth]{TwoMnWBSG_timeVsgeo.png}
	\caption{Marginalized PDFs of characteristic timescales of the simulation
and the characteristic distances and the projection angle of the merger. }
	\end{center}
\end{minipage}
\end{figure*}
\begin{figure*}
\begin{minipage}{180mm}
	\begin{center}
	%\vspace{200px}
	\includegraphics[width=0.7\linewidth]{TwoMnWBSG_timeVsinput.png}
	\caption{Marginalized PDFs of characteristic timescales of the simulation
and the inputs.}
	\end{center}
\end{minipage}
\end{figure*}
\begin{figure*}
\begin{minipage}{180mm}
	\begin{center}
	%\vspace{200px}
	\includegraphics[width=0.5\linewidth]{TwoMnWBSG_tri_vel.png}
	\caption{One-dimensional marginalized PDFs of velocities at
	characteristic times (panels on the diagonal) and marginalized PDFs of
different velocities.}
	\end{center}
\end{minipage}
\end{figure*}
\begin{figure*}
\begin{minipage}{180mm}
	\begin{center}
	%\vspace{200px}
	\includegraphics[width=0.5\linewidth]{TwoMnWBSG_velVStime.png}
	\caption{Marginalized PDFs velocities and the characteristic timescales
	of the simulation against the inputs.}
	\end{center}
\end{minipage}
\end{figure*}
\begin{figure*}
\begin{minipage}{180mm}
	\begin{center}
	%\vspace{200px}
	\includegraphics[width=0.5\linewidth]{TwoMnWBSG_velVSgeo.png}
	\caption{Marginalized PDFs of the velocities at characteristic timescales
		and the characteristic distances and the projection angle of the merger. }
	\end{center}
\end{minipage}
\end{figure*}
\begin{figure*}
\begin{minipage}{180mm}
	\begin{center}
	%\vspace{200px}
	\includegraphics[width=0.7\linewidth]{TwoMnWBSG_velVsinputs.png}
	\caption{Marginalized PDFs of relative velocities characteristic
	timescales of the simulation and the inputs.}
	\end{center}
\end{minipage}
\end{figure*}
\section{Model comparison for the outgoing and returning scenario}
\label{app:Bayes_factor}
\begin{figure}
	\includegraphics[width=\linewidth]{polar_prior_bounds.png}
	\caption{Comparison of the PDFs of the observed position of the NW relic (red bar
		includes 95\% confidence interval of location of relic in the center of
		mass frame) with the	predicted position from the two simulated merger scenarios (blue for
	outgoing and green for the returning scenario). 
	For the most likely value of $\beta < 1.1$, the returning scenario is preferred. 
	For comparison purpose, we also show that the shock velocity has to
	be as extreme as $1.5~v_{3D, NW}(t_{col})$ (top panel) for the outgoing
	scenario to be favored. 
	Note that we made use of the	polarization prior for producing this figure. 
	\label{fig: positionprior}}
\end{figure}
\begin{figure}
	\includegraphics[width=\linewidth]{polar_prior_bounds_SE.png}
	\caption{Comparison of the PDFs of the observed position of the SE relic (red bar
		includes 95\% confidence interval of location of relic in the center of
		mass frame) with the	predicted position from the two simulated merger scenarios (blue for
	outgoing and green for the returning scenario). 
	For the most likely value of $\beta < 1.1$, the returning scenario is preferred. 
	We obtained similar conclusion about the merger scenario as for the NW
	relic calculation.
	\label{fig:positionprior_SE}}
\end{figure}
We can write down our model for the different scenarios to be $M_{out}$ and
$M_{ret}$ for the outgoing and returning scenarios respectively. Both
models are computed using the same variables except for $TSC$. The models
are parametrized by $\beta$, which represents our uncertainty of $\langle v_{relic} \rangle$
(See equation~\ref{eq:proj_s_model}). Our data in this
case is the observed projected location of the radio relic $s_{proj}$. 
The Bayes factor thus can be computed as: 
\begin{align}
	K &= \frac{P(s_{proj} | M_{ret})}{P(s_{proj}| M_{out}}\\
	 &= \frac
	 {\int P(\beta | M_{ret}) P(s_{proj} | \beta, M_{ret}) d\beta}
	 {\int P(\beta | M_{out}) P(s_{proj} | \beta, M_{out})
 d\beta}\\
 &\approx 
 \begin{cases}
 & 1.6 \text{ for the NW relic}\\
 & 460 \text{ for the SE relic}
 \end{cases}
\end{align}
where $P(s_{proj} | \beta, M_{ret})$ and $P(s_{proj} | \beta, M_{out})$ can be
viewed as the green and blue curves respectively for each given $\beta$ value and we
have used priors on $\beta$ as: 
\begin{equation}
	\label{eq:beta_prior}
	P(\beta | M_{ret}) = P(\beta | M_{out}) =  
	\begin{cases}
		& \text{const}~\text{if } 0.7 \leq \beta \leq 1.5 \\
		& 0~\text{otherwise}.
	\end{cases}
\end{equation}
This conservative prior is chosen to reflect our belief that the
NW shockfront should propagate at approximately $v_{3D, NW}(t_{col})$ in the
center-of-mass frame, and likewise for the SE shockfront, although we do
not think that $\beta_{SE} = \beta_{NW}$ has to be necessarily true. 
%When we assume $\langle v_{relic} \rangle / v_{3D, 1}(t_{col}) = 1.5$ (uppermost
%panel of Figure \ref{fig: positionprior}), $\langle v_{relic} \rangle$
%corresponds to an time-averaged velocity of $\sim3800~\kilo
%\meter~\second^{-1}$ relative to SE subcluster, or, in the center of
%mass frame $\sim1400~\kilo\meter~\second^{-1}$. 
As more simulations of merger shocks become available, one can update equation~\ref{eq:beta_prior} accordingly.


\bsp 
\label{lastpage} 
\end{document}
