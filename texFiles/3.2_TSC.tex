\textbf{The simulation gives two plausible estimates for
the time-since-collision, with $TSC_0 = \giga \text{yr}$ and $TSC_1 = \giga
\text{yr}$}. The presence of the radio relic, in conjunction with a
depression in the X-ray surface brightness shown in M11, strongly suggest
that El Gordo is a post-collision system. 
Based on \ref{sec: positionprior}, we have come up with an estimate for the
likely position based on the two inferred TSC. 

In this calculation, the dominant uncertainty comes from the
reported location of the radio relic, 
After taking into this uncertainty, the possible locations of the relic still favors the
outgoing scenario. (See figure \ref{fig: positionprior}) Other
uncertainties arise from how we define the reference frame for the calculation. The
uncertainty associated with the two centroids are of the order
of $\sim 0.1 \mega$pc \citep{Jee13} and are relatively unimportant???.  

%Base on the time evolution of radio relic, we speculate that El Gordo has already reached its apoapsis and the two subclusters are heading for another merger.
%%This degeneracy between $TSC_0$ and $TSC_1$ that
%is not resolved by taking the separation constrain from the radio relic. 

%the post-collision estimate of the
%time-since-collision ($TSC_0$), the pre-collision estimate of the
%time-since-collision ($TSC_1$) and the time between collisions ($T$).
%We pick the post-collision time-since-collision estimate ($TSC_0$) of Gyr to be representative of the observed status of El Gordo, instead of the pre-collision time-since-collision estimate ($TSC_1$). While the simulation models both scenarios either the subclusters are approaching each other (incoming) or they have already passed through each
%other and are moving apart (outgoing), the presence of the radio relic rules out the possibility that the two subclusters still have not encountered each other. 
%However, this does not exclude the possibility of having the subcluster
%approaching each other again after reaching apoapsis.  
%%\citet{b9} have reported that the simultaneous optical and near-IR data of
%%AC Her can be fitted by a combination of two blackbodies at 5680 and
%1800\,K, representing, respectively, the stellar and

