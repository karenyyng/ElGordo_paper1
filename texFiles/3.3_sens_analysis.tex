\label{sec: sensitivityTests}
We performed tests of how each of the output variables vary according to the
choice of the cutoff of the polarization prior between
$\alpha_{\text{cutoff}} =
29 \degree$ to $49\degree$ instead of $35 \degree$.  
We found that in the most extreme case, choosing the cutoff values as $29
\degree$ ($-6 \degree$), the location of the $v_{3D}(t_{obs})$, is
increased by $ 16 \%$. While the upper $95\%$ CI of $d_{max}$ is
the most sensitive to the prior and it changes by
$\sim 20 \%$ when $\alpha_{\text{cutoff}} = 49 \degree$. 
This shows that the exact choice of the cut off value for $\alpha$ does
not affect our estimates significantly.

%Should also talk about angle uncertainty between choosing different models
%%and how the sensitivity test show the percentage tests for those. 
%Even if we made use of the least conservative model, assuming weak field
%and a radio spectral index of 1.5 from \cite{E98}, which
%corresponds to a cutoff of $\alpha = 33\degree$, the output variable would
%be changed by X\% at most. 

%how does this affect the conclusion of our merger scenario? 
% talk about the sensitivity tests for TSM_1 and TSM_0
% talk about the sensitivity tests for  




