We performed tests of how each of the output variables vary according to the
choice of the cutoff of the polarization prior.
We found that in the most extreme cases, shifting of the cutoff values by
$2.5\degree$ ($\sim 6.4\%$ of the choosen cut off), the location of the output
variable, $v_{3D}(t_{obs})$, is changed by $\%$. The CI of BLAH variable is
the most sensitive to the prior and changed by $X\%$ from CI_LOWER, CI_UPPER to
CI_LOWER to CI_UPPER for the $68\%$ credible interval. 

The overall effect of varying the cutoff value does not favor the
returning scenario significantly more. 

%Should also talk about angle uncertainty between choosing different models
%and how the sensitivity test show the percentage tests for those. 
Even if we made use of the least conservative model, assuming weak field
and a radio spectral index of 1.5 from \cite{E98}, which
corresponds to a cutoff of $\alpha = 33\degree$, the output variable would
be changed by X\% at most. 

%how does this affect the conclusion of our merger scenario? 
% talk about the sensitivity tests for TSM_1 and TSM_0
% talk about the sensitivity tests for  




