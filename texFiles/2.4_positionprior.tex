One of the biggest question involving the merger is if El Gordo was
observed to be in a returning or outgoing phase.
We perform a qualitative comparison of the observed and the
simulated position of the NW radio relic in each scenario.
%This method depends on the time evolution of the relic velocity, which gives the upper and lower bounds on the possible position of the relic for each scenario. 
%%The velocity depends on a number of physical quantities, including the
%local gravitational potential, matter density, temperature, pressure among
%others (\citealt{E98}, Shu .F., more citations?).  The exact time evolution
%of the shock velocity requires detailed numerical simulation similar to
%\citet{Springel2007}, \citet{Vazza11}, \citet{Kang2007}, etc.  
We considered different possibilities of the time evolution of the shock
velocity due to lack of knowledge of the details of the evolution. We drew physical insight from the simulations of the merger shock of the
staged numerical simulation of the Bullet cluster from \citet{Springel2007}
and the cosmological simulation from \citet{Paul2011b}. Right after
the collision of the subclusters, \citet{Springel2007} shows that the shock speed is
comparable to the merger speed of the two subclusters; the shock speed
dropped only by $\sim 14\%$ in the $300~\mega$yr period while the speed of
the main subcluster in the simulation dropped by $\sim65\%$ in the center
of mass frame. On the other hand, \citet{Paul2011b} showed that the shock
speed was $\sim1.5$ times the relative collisional speed of the subcluster
shortly after the collision and the shock speed decreases only
slightly as it propagates away from the center of mass. \par  
We approximated the upper and lower bounds of the NW relic speed with the
simulated speeds of the NW subcluster.  We worked in the center of mass frame where the shock speed is expected to
drop slightly with time. 
The projected separation of the shock is approximated as:
\begin{equation}
	s_{proj} = \langle v_{relic} \rangle (\hat{t}_{obs} - \hat{t}_{col}) \cos(\hat{\alpha})
	\label{eqn: projectedsep}
\end{equation}
where the quantities with hats on the right hand side of the equations were
inferred from the simulation, and $s_{proj}$ is the estimated projected separation and we estimated the
upper and lower bounds of the time-averaged velocity
$\langle v_{relic} \rangle$ of the shock between
the collision of the subclusters and the observed time as:  
\begin{equation}
	\langle v_{relic} \rangle = \beta~v_{3D,1}(t_{col})  
\end{equation}
where $0.8 \leq \beta \leq 1.2$ is a factor that we introduce to represent the
uncertainty of the speed of the relic and $v_{3D,1}(t_{col})$ refers to the collisional velocity of
the NW relic in the center-of-mass frame. The upper bound can be
approximated as the collisional speed of the NW subclusters due to how the
shock is powered by the collision. After the collision, it is unlikely that
there would be significant energy injected into the shock to speed up the
shock such that the shock travels much faster than the collision speed of the subcluster. While the shock does not experience gravitational deceleration as a
pressure wave, some dissipative processes may have slowed down the shock
wave slightly as it propagated. By making use of a
range of $\beta$ values, we examine how the rate of slow down would
give a different lower bound of the projected separation of the relic.   
\par      
