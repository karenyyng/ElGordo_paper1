% this line is included when individual section needs to be compiled and
% reviewed  
%% mn2esample.tex
%
% v2.1 released 22nd May 2002 (G. Hutton)
%
% The mnsample.tex file has been amended to highlight the proper use of
% LaTeX2e code with the class file and using natbib cross-referencing.
% These changes do not reflect the original paper by A. V. Raveendran.
%
% Previous versions of this sample document were compatible with the LaTeX
% 2.09 style file mn.sty v1.2 released 5th September 1994 (M. Reed) v1.1
% released 18th July 1994 v1.0 released 28th January 1994

\documentclass[letterpaper,useAMS,usenatbib]{"mn2e"}
%\documentclass[letterpaper,useAMS]{"mn2e"}
%LINUX version of the path
%\documentclass[useAMS,usenatbib,letterpaper]{"/media/blank/Macintosh
%HD/Users/karenyng/Library/texmf/tex/latex/commonstuff/mn2e"}

% If your system does not have the AMS fonts version 2.0 installed, then
% remove the useAMS option.
%
% useAMS allows you to obtain upright Greek characters.  e.g. \umu, \upi
% etc.  See the section on "Upright Greek characters" in this guide for
% further information.
%
% If you are using AMS 2.0 fonts, bold math letters/symbols are available
% at a larger range of sizes for NFSS release 1 and 2 (using \boldmath or
% preferably \bmath).
%
% The usenatbib command allows the use of Patrick Daly's natbib.sty for
% cross-referencing.
%
% If you wish to typeset the paper in Times font (if you do not have the
% PostScript Type 1 Computer Modern fonts you will need to do this to get
% smoother fonts in a PDF file) then uncomment the next line 
%\usepackage{Times}

%%%%% AUTHORS - PLACE YOUR OWN MACROS HERE %%%%%
\usepackage{hyperref}
\usepackage{amssymb}
\usepackage{graphicx}
\usepackage{amsmath}
\usepackage[amssymb]{SIunits} 
\usepackage{booktabs}
\usepackage{hhline}
\usepackage{breqn}
\usepackage{standalone}
\usepackage{dcolumn}
	\newcolumntype{d}[1]{D{.}{.}{#1}}
\usepackage{tabularx}
\usepackage{booktabs}
\graphicspath{{graphics/}}
\newcommand{\mc}[1]{\multicolumn{1}{c}{#1}} % handy shortcut macro
%-----------------------------------------------------------------------

%\begin{document}

% draft: 
% explain that 4300 km /s from L13 
% explain how the position of the radio relic is calculated 
% explain the physics / concepts behind this calculation 
% explain the assumptions
% explain the data 
% decide on which frame of reference we want to do the calculation in 
% corresponding code: position_prior_elgordo.ipnb in
% TSM/mercury_elGo/Feb_data folder 
% incorporate the uncertainty of the centroid position???????
% paragraph 1: explain where the data comes from and how they are used 
% why only the NW relic is used but not the SE one ?
%Even though there is a list of standard required data as denoted in section
%\ref{sec: inputs} for the simulation, it is straight forward to incorporate
%new data variables. 
%With additional of data from the radio relic
%\citep{L13}, this simulation is capable of providing a quantitative view of
%how likely each of  the two possible merger scenarios mentioned in section
%\ref{sec: outputs} is true. 
%We make reference to the radio relic data from \cite{L13} in this
%subsection for extending the simulation. Three sources of radio relic
%were identified - including the NW, SE and the E relic. The NW radio
%relic possesses the most extended geometry among all the identified relic source. 
%
%% make sure of Springel and Farrar 
%% the frame of the shock given by Lindner et al. is complicated 
%% there is no trivial translation between  
%We do not refer to the The SE nor the E radio relic in this calculation
%since we do not have an estimation of the shock speed of the SE relic
%nor the E relic from \citealt{L13}.    
%We incorporate the calculated shock speed of the radio relic as part of
%$\vec{D}$ for the Monte Carlo simulation. 
%We draw  $v_{relic} \sim N(4300~\kilo\meter~\second^{-1},
%800~\kilo\meter~\second^{-1})$ \citep{L13} from each realization and
%compute how far the NW relic would have traveled given a certain TSC in the
%frame of the SE subcluster. Finally we compare the distance traveled
%predicted from the simulation to the reported position of the NW relic at RA = 01:02:46, DEC = -49:14:43\citep{L13}.
%
% this paragraph should really discuss why we are assuming constant speed 
% paragraph 2: estimate the upper and lower bound of the speeds and how
% that is going to impact the conclusion drawn 
% Ambiguity about the speed:
% do we need to talk about that we are only referring to the NW relic not
% the SE relic? Yes! 
% do we need to explain why we are only using the NW relic not the SE
% relic?
We demonstrate that despite substantial uncertainty in the time
evolution of the shock, we can give constraints on the likelihood of the
outgoing and incoming merger scenarios by comparing the observed and the
simulated position of the radio relic.
The key to the success of this method depends on our ability to estimate the
time evolution of the relic velocity, which gives the upper and lower
bounds on the possible position of the relic for each scenario. The
uncertainty of the time evolution of the velocity of the shock stem from
how the velocity depends on a number of physical quantities, including the
local gravitational potential, matter density, temperature, pressure among
others (\citealt{E98}, Shu .F., more citations?).  The exact time evolution
of the shock velocity requires detailed numerical simulation similar to
\citet{Springel2007}, \citet{Vazza11}, \citet{Kang2007}, etc. 
We draw physical insight from the simulations of the merger shock of the
Bullet cluster from \citet{Springel2007} and \citet{Paul2011b}, which both show
that, right after the collision of the subclusters, the shock speed is
comparable to the merger speed of the two subclusters in the center of mass
frame.  While \citet{Paul2011b} reports the shock speed decreases only
slightly as it propagates away from the center of mass, the Bullet shock
speed mildly drops by $\sim 14\%$ in the 300~\mega yr period after the
formation of the Bullet shock versus a $\sim65\%$ drop for the main
subcluster. \par  
We approximated the upper and lower bounds of the NW relic speed with the
simulated speeds of the NW subcluster.  We simplified the calculation by
working in the center of mass frame where the shock speed is expected to
drop slightly with time. 
The projected separation of the shock is approximated as:
\begin{equation}
	s_{proj} = \langle v_{relic} \rangle (\hat{t}_{obs} - \hat{t}_{col}) \cos(\hat{\alpha})
\end{equation}
where $s_{proj}$ is the estimated projected separation and we estimated the
upper and lower bounds of the time-averaged velocity
$\langle v_{relic} \rangle$ of the shock between
the collision of the subclusters and the observed time as:  
\begin{equation}
	\beta~v_{3D,1}(t_{col}) < \langle v_{relic} \rangle \lesssim v_{3D,1}(t_{col}) 
\end{equation}
where $0 < \beta < 1.0$ is a factor that we introduce to denote the slow-down
of the relic and $v_{3D,1}(t_{col})$ refers to the collisional velocity of
the NW relic in the center-of-mass frame. The upper bound can be
approximated as the collisional speed of the NW subclusters due to how the
shock is powered by the collision. After the collision, it is unlikely that
there would be significant energy injected into the shock to speed up the shock such that
the shock travels much faster than the collision speed of the subcluster. 
While the shock does not experience gravitational deceleration as a
pressure wave, some dissipative processes may have slowed down the shock
wave slightly as it propagates. By making use of a
range of $\beta$ values, we examine the how the rate of slow down would
give a different lower bound of the projected separation of the relic.   

We compare the bounds with the observed position of the NW relic  at RA = 01:02:46, DEC = -49:14:43 \citep{L13}.
Three sources of radio relic
were identified - including the NW, SE and the E relic. The NW radio
relic possesses the most extended geometry among all the identified relic source. 
We do not refer to the The SE nor the E radio relic in this calculation
since we do not have an estimation of the shock speed of the SE relic
nor the E relic from \citealt{L13} for comparison.    
From \citet{L13}, the estimated speed of the NW relic is $4300 \pm
^{800}_{500}\kilo\meter~\second^{-1}$. This speed is compatible with our
simulated collisional speed but the conversion of the frame of reference is
non-trivial since the speed is measured with respect to on the turbulent
intercluster gas as a Mach number.


%We incorporate the calculated shock speed of the radio relic as part of
%$\vec{D}$ for the Monte Carlo simulation. 
%We draw  $v_{relic} \sim N(4300~\kilo\meter~\second^{-1},
%800~\kilo\meter~\second^{-1})$ \citep{L13} from each realization and
%compute how far the NW relic would have traveled given a certain TSC in the
%frame of the SE subcluster. Finally we compare the distance traveled
%predicted from the simulation to the reported position of the NW relic 

% make sure of Springel and Farrar 
% the frame of the shock given by Lindner et al. is complicated 
% there is no trivial translation between  

%Observationally, studies from Lindner et al. also provide some
%comparison of the estimated shock speed relative to the gas medium as $4300
%\pm^{800}_{500} \kilo \meter~\second^{-1}$. 

%We simplified the computation of the position of the relic by estimating
%the time-averaged velocity of the relic.
%We treated the lower bound of the time-average
%relic velocity $\hat{v}_{relic} \approx  \hat{v}_{3D}(t_{col})$ due to how
%shock speed evolves over time.  

%We performed two sets of calculation to bracket the possible positions of
%the radio relic.
%
%%* shock speed from Lindner et al. is in the frame of  
%
%In order to capture the major uncertainty in the time evolution of the
%shock, we approximated the time-averaged velocity of the shock with a mean
%as the relative merger speed, and assumed a very wide support of $\kilo
%\meter~\second^{-1}$.
%(See Figure to denote how the distribution compares with the
%relative velocity, the observed shock velocity in the unknown frame)
%\begin{itemize}
%\item ``In previous work, it has been assumed that the shock
%			velocity is equal to
%			the subclustre's relative velocity with respect to the parent
%			cluster
%			(Markevitch et al. 2002, Hayashi \& White 2006, Markevitch
%			2006, among
%			others')'' all velocities are quoted in the CM frame
%
%\item Shock speed related to the local density / temperature / pressure of
%	the medium.
%\end{itemize}

%We denoted the speed in the SE subcluster frame, we take into consideration
%that after the collision, the SE subcluster move away from the NW
%shock(relic) at a deceleration due to the gravitational pull of the NW subcluster.
%Therefore, in the SE subcluster frame, the shock would also slow down. 
%(See figure \ref{fig:shock_evo} for the evolution of the shock wave speed
%in different reference frames)
%The study of the radio relic from \citealt{L13} also shows that the shock
%speed can be consistent with the merger speed. 
%However, since the shock speed was measured with respect to the 
%(See figure X for a distribution
%of the merger speed and how that compares to the observed relic speed) 
%
%The greatest uncertainty remains in the modeling of the time evolution of
%the speed of the shock wave.
\par      
%\bibliographystyle{mn2e}
%\bibliography{bib}
%
%\end{document}
