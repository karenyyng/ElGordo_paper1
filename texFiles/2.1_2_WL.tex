%The input 
We obtained the PDFs of the masses of the subclusters by doing a Monte
Carlo Markov Chain (MCMC) analysis of the reduced shear from the
weakly lensed background galaxies similar to \citet{Dawson12}. We computed the reduced shear signal
generated by two NFW halos according to \citet{Umetsu10} (See Appendix
\ref{app:MCMC} for
details of implementation and output diagnostics).
At each step we followed the procedure of a
Metropolis algorithm.  The transition kernel was set to
be the log likelihood of fit of the model shear to the reduced shear of the
data (\ref{eqn:jointposterior}).
In total, eight MCMC chains were used. After every 5000 MCMC steps for all
the chains, we computed the R coefficient \citep{Gelman92}  to
check for convergence. We performed more MCMC steps as long as convergence
was not achieved. After convergence was achieved, we removed the
burn-in portions of the MCMC chains and used the resulting MCMC chains as
samples of the PDFs of the masses. \par 
% this is from Jee13 section 3.5
We make use of an effective redshift of $z_{\text{eff}} = 1.37$ or $D_{LS}
/ D_L = 0.276$ \citepalias{Jee13} and $g'\approx(1 + 0.79 \kappa)g$
(\citetalias{Jee13}, \citealt{Seitz97}).    
%\textbf{The inputs of our MCMC mass inference are from proposal blah,
%which is similar to \citepalias{Jee13}, however, this
%separate implementation of the MCMC analysis code is different than
%\citepalias{Jee13}.} 
On the other hand, we fixed the
position of the centers of the NFW halos to be  the luminosity peaks of the
respective galaxy populations of  the two subclusters, which are at R.A. $=
01$:02:51.68, Decl. = $-49$:15:04.40 and R.A. = $01$:02:38.38, Decl. =
$-49$:16:37.64 for the NW and SE subclusters
respectively \citepalias{Jee13}. The separations between the luminosity
peaks and the estimated mass centroids of the subclusters are $X kpc$ and
$X kpc$ respectively for the NW and SE subcluster.  (! Lori and Nick actually asked why we do not free
the centroids like Jee 13)  The agreement between our analysis and \citepalias{Jee13} to within
the 68\% credible interval serves as a sanity check on the estimated masses. 
%Other sanity check including the control of acceptance rate between 20\%
%and 50\%. 

The mass estimates from \citealt{Jee13} are $M_{200c} = 13.8~\pm~2.2\times
10^{14}~h_{70}^{-1} M_{\sun}$ for the NW subcluster and $M_{200c} = 7.8~\pm
2.0\times10^{14}~h_{70}^{-1} M_{\sun}$ for the SE subcluster. 
