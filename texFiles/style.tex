% mn2esample.tex
%
% v2.1 released 22nd May 2002 (G. Hutton)
%
% The mnsample.tex file has been amended to highlight the proper use of
% LaTeX2e code with the class file and using natbib cross-referencing.
% These changes do not reflect the original paper by A. V. Raveendran.
%
% Previous versions of this sample document were compatible with the LaTeX
% 2.09 style file mn.sty v1.2 released 5th September 1994 (M. Reed) v1.1
% released 18th July 1994 v1.0 released 28th January 1994

\documentclass[letterpaper,useAMS,usenatbib]{"mn2e"}
%\documentclass[letterpaper,useAMS]{"mn2e"}
%LINUX version of the path
%\documentclass[useAMS,usenatbib,letterpaper]{"/media/blank/Macintosh
%HD/Users/karenyng/Library/texmf/tex/latex/commonstuff/mn2e"}

% If your system does not have the AMS fonts version 2.0 installed, then
% remove the useAMS option.
%
% useAMS allows you to obtain upright Greek characters.  e.g. \umu, \upi
% etc.  See the section on "Upright Greek characters" in this guide for
% further information.
%
% If you are using AMS 2.0 fonts, bold math letters/symbols are available
% at a larger range of sizes for NFSS release 1 and 2 (using \boldmath or
% preferably \bmath).
%
% The usenatbib command allows the use of Patrick Daly's natbib.sty for
% cross-referencing.
%
% If you wish to typeset the paper in Times font (if you do not have the
% PostScript Type 1 Computer Modern fonts you will need to do this to get
% smoother fonts in a PDF file) then uncomment the next line 
%\usepackage{Times}

%%%%% AUTHORS - PLACE YOUR OWN MACROS HERE %%%%%
%\usepackage{hyperref}
\usepackage[colorlinks=true,
            linkcolor=blue,
            urlcolor=blue,
					citecolor=blue]{hyperref}
\usepackage{amssymb}
\usepackage{graphicx}
\usepackage{amsmath}
\usepackage[amssymb]{SIunits} 
\usepackage{booktabs}
\usepackage{hhline}
\usepackage{breqn}
\usepackage{standalone}
\usepackage{dcolumn}
	\newcolumntype{d}[1]{D{.}{.}{#1}}
\usepackage{tabularx}
\usepackage{booktabs}
\usepackage{microtype}
\graphicspath{{graphics/}}
\newcommand{\mc}[1]{\multicolumn{1}{c}{#1}} % handy shortcut macro
%-----------------------------------------------------------------------
