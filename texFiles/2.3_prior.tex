One of the biggest strengths of the Monte Carlo simulation by \citetalias{D13} is its ability
to detect and rule out extreme input values that would result in
unphysical realizations via the application of prior probability. 
Our default priors are described in D13 and we include them in Appendix
\ref{app: results} for the convenience of the readers. 
In addition, we have come up with a prior on the projection angle $\alpha$
based on the polarization fraction of the radio relic.
%Radio relics have been suggested to be able to constrain the mass ratios,
%the projection and the merger configuration \citep{vanWeeren10}. 

%%%%%%%%%% WAIT WHAT AM I TRYING TO SAY????
%Ever since the first detection of radio relic, cosmological hydrodynamical simulations of
%merging clusters have been used to model their emission spectrum and
%geometry. (\citealt{Vazza11}, \citealt{VanWeerenRJ2011}, Bonafede
%et al. 2013, \citealt{E98}, Br\"{u}ggen et al., Skillman et al.
%2013) While such cosmological simulations have provided valuable insights
%to verifying the physical models, they are expensive in terms of
%computational power and novel techniques have to be invented in order to
%analyze the large amount of simulated data so progress has been slow. 
%Our Monte Carlo simulation can make use of known physics combined with the
%preliminary results from such cosmological simulations to use properties of
%the radio relic to constrain merger dynamics. 

%Compared to hydrodynamical simulations or cosmological simulations, this
%    Monte Carlo simulation is not demanding in terms of CPU time, therefore, we
%    can run many realizations in order to probe how the input variables
%affect the output variables. 

%\begin{itemize}
%\item talks about the observable, which is the comoving kinetic power through each shock surface
%\item refer to diffusive shock acceleration (DSA) mechanism?
%\item Kang \& Jones treatment of Mach number-dependent efficiency
%considering the possibility of having an non-isotropic magnetic field  
%$KJ_BparallelRadial$ model
%\end{itemize}
%\begin{figure}
%	\includegraphics[width=\linewidth]{d_3d_prior1.png}
%	\caption{The marginalized output PDFs of the observed 3D separation
%		($d_{3D}$) 
%		with and without the radio prior applied. 
%		(maybe I should replot this more nicely without too many
%		distracting lines but only the lines showing the location)
%		%The vertical
%		%lines denote, dashed line: biweight location, dash-dot
%		%line: 68\% credible limit, dotted line: 95\% credible
%		%limit.
%	\label{fig:radioprior}}
%\end{figure}


%\subsubsection{Weighting function based on the observed position of the radio relic}\label{sec:relic} 
%-----------------------------------------------------------------------
%Among the known galaxy cluster mergers that are associated with radio
%relics, \cite{Vazza12} noted that most of them have radio relic located
%more than 800 \kilo pc away from the merger center. \cite{Vazza12} then
%conducted hydrodynamical simulations of twenty of known galaxy mergers with 
%radio relic to investigate this observed trend. They found a radial
%trend of kinetic power dissipation increasing up to around half the virial
%radius (r$_{vir}$) of the cluster. Summarizing the results from the proposed model for energy dissipation of the radio relic, \cite{Vazza12} gives the range of highest kinetic power emission in a range of 
%$.2 ~r_{\mathrm{vir}} < d_{\mathrm{3D}} < .5~r_{\mathrm{vir}}$.
%
%
%% In particular, Vazza et al. (2011) showed dependence of observed
%%location of radio relic: when the clusters are at small separation, the
%%Mach number is too high for a radio shock to form and the steep fall off of
%%the emission power of the radio relic as a function of separation makes it
%%difficult to observe a radio relic when it has propagated beyond a certain
%%separation.  
%
%%\textbf{We take into account the uncertainties of their modeling and 
%% construct prior probability on a range of 3D separation for which the kinetic
%%power dissipation of the radio relic is more than 10\% of the peak value.} 
%Since we do not have information on how the probability of
%being able to observe the relic would fall off as a function of emission power, we adopt a conservative approach and designed a uniform prior. 
%%and contrast that to a flat prior to test the effect of the prior on the output variables. 
%We also take into account the uncertainties of the different proposed power
%emission model and come up with a prior of:
%\[
% \text{P}({d_{3D}}) = 
%\begin{cases} 
%\text{constant,} & \text{for 1.0  Mpc} < d_{3D}(t_{\mathrm{obs}}) < 3.0 \text{ Mpc}\\
%0, & \text{otherwise}
%\end{cases}
%\]
% for El Gordo.\par 
%%\begin{equation}  
%%P(d_{3D}(t_{\mathrm{obs}})) = 
%%\begin{cases}
%%1/C, \text{ if }1.0 \text{ Mpc } < d_{3D}(t_{\mathrm{obs}}) < 3.0 \text{ Mpc} \\ 0, \text{ otherwise}
%%\end{cases}
%%\end{equation}
%
%\begin{figure}
%	\includegraphics[width=\linewidth]{alpha_pdf_prior_diff.png}
%	\caption{The projection angle with and without the radio relic
%prior applied. (Needs to update and label the figure better)} 
%\end{figure}

