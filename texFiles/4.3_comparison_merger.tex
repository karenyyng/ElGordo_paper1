
Talks about how El Gordo is more massive and collided at higher speed than
both the Bullet and the Musketball, so El Gordo is probably a better probe of SIDM properties.
%This estimated value has
%several implications for the analysis of the mass estimation and our
%knowledge of the physics underlying the creation of radio relic.   
%
%Compare with other literature???? 
%address concerns from James' paper that talks about how the mass would
%remain high as long as the viewing angle is $\le$ 65\degree ? 
%
% defy expectation 
%brings up that the large projection angle is unexpected
%merger axis has to be close to the sky for the radio relic to be observed 
%From simulation, it has been shown that the size of radio relic can be
%used to constrain with the mass ratio of the subclusters, with the more massive subcluster being further away from the larger relic. However, this is not the case for 
%El Gordo. The more prominent relic is located further away from the less massive
%SE cluster. This can be explained if the following holds: 
%\begin{enumerate}
%\item the less massive SE subcluster shows a higher redshift than the NW subcluster. This suggests that the SE radio relic could be moving away from us and
%hence appears to be less bright than its NW counterpart\\ 
%\item the extend to which (i) is true depends on the projection angle. With
%$\alpha \sim 41.7\degree$, one can naively calculate that the observed relic
%would be only $\cos(41.7 \text{ deg}) \sim .75 $ the size of the original relic. %\end{enumerate}

With this new piece of evidence, we find that the absence of an
X-ray shock feature from El Gordo, may not be due to the merger speed being
low, as suggested by \citetalias{Jee13}. 
In particular, taking into account that the estimated projection angle of 
$\sim 41.7\degree$, we estimate the projected relative velocity to be 597
\kilo \meter~\second$^{-1}$, which is consistent with the estimated
line-of-sight velocity differences of $586 \pm  96~\kilo \meter
~\second^{-1}$ in \citetalias{M11}. 

Furthermore, the study from \cite{L13} Lindner et al. has come up
with an estimation of the shock velocity of the radio relic of El Gordo as 
$\sim 4000~\kilo \meter~\second^{-1}$. While this shock velocity is not the
same as the merger velocity, they should be of similar magnitude. Indeed
our simulation found that a merger velocity of $4000~\kilo
\meter~\second^{-1}$ is within the 95\% credible interval. 
