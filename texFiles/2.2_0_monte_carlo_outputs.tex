We summarize the output of the simulation here and leave detailed
plots and descriptions in Appendix \ref{app: plots}. The simulation
provides PDF estimates for many of the output variables. Variables
of the most interest include the time dependence and $\alpha$, which is
defined to be the projection angle between the plane of the sky and the merger axis. Other output variables are dependent on $\alpha$ and the time
dependence. Specifically, the simulation denotes the time dependence by
providing several characteristic time-scales, including the time
elapsed between the collision and when the subclusters first reach apoapsis
($T$) and the time-since-collision.  

\textbf{The two version of the time-since-collision variables $TSC_0$ and
$TSC_1$ denotes different possible merger scenarios.} 1) We call the scenario for which the subclusters are
moving apart after collision to be ``outgoing" and it corresponds to the
smaller $TSC_0$ value, and 2) we call the alternative scenario 
``incoming" for which the subclusters are approaching each other after turning
around from the apoapsis for the first time and it corresponds to $TSC_1$.
We describe how we use to break the degeneracies of the two scenarios in
section \ref{sec: positionprior}. 
 
\textbf{The simulation also output estimates of variables that characterize
the dynamics of the merger.} The 3D velocities, both at the time of the
collision ($v_{3D}(t_{col})$) and at the time of observation
($v_{3D}(t_{obs})$) are provided. The maximum 3D separation ($d_{max}$)
which is defined to be the distance between the position of collision to the apoapsis is also part of
the outputs. (See the lower half of Table \ref{tab:outputs} for all the outputs).
%Here we present results based on:\\  %1) a flat radio prior\\
%2) a uniform prior over a range of most likely 3D separations\\
%3) a Gaussian prior  
%We discuss in subsection \ref{sec:priors}  on the use of the default filters
%and two new filters designed according to the observed data and the physics of the radio relic.
%
%\textbf{While the underlying formalism of the Monte Carlo simulation is
%    based on the Bayes theorem, we caution the reader that this simulation
%    does not correspond to a conventional Bayesian parameter estimation but
%    more similar to the Bayesian uncertainty estimation method mentioned in Saltelli 
%    et al. (2004). (See appendix \ref{} for a more in-depth discussion)}

