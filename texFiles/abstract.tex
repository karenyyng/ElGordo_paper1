
Merging galaxy clusters with radio relics provide rare insights to the merger
dynamics as the relics are created by the
violent merger process. 

From the double radio relic observation and X-ray wake morphology, 
it is believed that El Gordo is observed shortly after the first passage
before reaching apo

We demonstrate one of the first uses of the
properties of the radio relic
to reduce the uncertainties of the dynamical variables 
and 3D configurations of a cluster merger, ACT-CL J0102-4915, El Gordo. 
At a redshift of 0.87, El Gordo (M$_{200c} = 
2.75\times10^{15} \pm^{7.4}_{1.5}$ M$_{\sun}$) is one of the most massive
clusters discovered in the early universe. The two subclusters of El
Gordo has a mass ratio of around 2:1. 
The X-ray and weak-lensing data of El Gordo show an offset of X kpc between
the intercluster gas and the dark matter (DM) at $\sim$4 $\sigma$ level.
All these features of El Gordo make it part of a valuable class of
dissociative mergers that can probe the self-interaction of dark matter.
%As more and more cluster mergers are being discovered using
%the radio relic emission detected in upcoming large scale radio surveys,
We employ a Monte Carlo simulation to investigate the three-dimensional (3D)
configuration and dynamics of El Gordo. 
We give a summary of the inferred
dynamical variables. By making use the polarization, velocity and position
of the radio relic, we are able to confirm at X $\sigma$ that the subclusters of El Gordo are moving away from each other. We find that
the 3D merger speed of El Gordo to be $\sim3000~\kilo\meter~
\second^{-1}$ (or in projected velocity = ), which is still consistent with the low line-of-sight
velocity of $\sim600~\kilo\meter~\second^{-1}$ based on the inferred time-since-collision ($TSC$ = Gyrs) and
the projection angle (\(\alpha = 41^{\circ}\pm \)). We put our estimates of $TSC$ and $\alpha$ into context by relating them to existing observations of El Gordo. 
Finally, we compare our simulation result of El Gordo to the simulation
result of the Bullet Cluster, and show that
El Gordo is a very promising candidate for giving tigher constraint than
the Bullet Cluster on the self-interaction of dark matter. 
(200 words)
(check against astro-ph word limit)
%\textbf{Findings}\\ 
%We found that the merger speed at the collision of El Gordo
%($\mathrm{km/s}$) is higher than those of the Bullet Cluster.  

%We estimate El Gordo to have a slightly higher TSC than the bullet cluster
%using the same analysis by Dawson (2012).  
%To be continued.  (250 words) 
