In particular, \cite{L13} reports an integrated
polarization fraction of $\sim33\%$ for the two identified relics. The
high integrated polarization fraction can be explained by uniformly aligned
magnetic field. (Synchrotron emission from unorganized magnetic field are
randomly polarized) We refer to a model from \citet{E98} with
the following physical picture: during a merger, the intracluster medium is compressed, this aligns the unordered
magnetic field perpendicular to the line joining the cluster center to the
radio relic. (\citealt{E98}, \citealt{vanWeeren10}, \citealt{Feretti12})
Thus, the synchrotron emission emitted from the electrons near this aligned
magnetic field is strongly polarized perpendicular to this magnetic field. \par
\textbf{The major assumption behind the design of our filter
is that the integrated polarization fraction is a monotonically
decreasing function of $\alpha$.} 
This assumption is inspired by the class of models given by \cite{E98}, 
which, despite various inputs for spectral indices and magnetic field strength, each predicts a monotonically decreasing integrated
polarization fraction as a function of $\alpha$. 
%This assumption is yet to be verified by cosmological simulations of radio relic. 
In particular, we refer to a model from \cite{E98} that would give the most
conservative estimate on the upper bound of $\alpha$. 
This model predicts a maximum integrated polarization fraction of
$\sim75\%$ when $\alpha = 0$ . From this model, the observed integrated
polarization fraction of 33\% corresponds to $\mu_\alpha =  39\degree$. 
%We consider 39\degree as an upper bound on the projection angle since this idealized model assume isotropic distribution of magnetic field and
%electrons. 
This  polarization fraction of $\sim 75\%$ predicted by \citep{E98} is
consistent with the upper bound of relic polarization fraction in cosmological
simulations \citep{S13}. No other model of the magnetic field should predict a higher polarization fraction, thus it is highly unlikely that we see 33\%
integrated polarization at $\alpha > 39\degree$. \par 
\textbf{We cannot rule out $\alpha \leq 39\degree$ as a result of possible
variations in the magnetic field.} \cite{E98} assumes an isotropic
distribution of electrons in an isotropic magnetic field. Cosmological
simulations of radio relics from \cite{S13} show varying polarization
fraction across and along the relic assuming $\alpha = 0$, resulting in a
lower integrated polarization fraction. 
%Due to these likely variations in the true magnetic field, the true observable integrated polarization values at a given $\alpha$ can be lower than what is predicted by \cite{E98}. 
%For example, it is possible that the radio relic of El Gordo has a lower maximum face-on polarization fraction than 75\%, but if we are viewing the relic at a smaller $\alpha$, the integrated
%polarization fraction can still comes out to be 33\%.
For example, it is possible to see a edge-on radio relic ($\alpha = 0$) with integrated polarization fraction of 33\%. 
\par 
%With simplifying assumptions, \cite{E98} have derived the integrated polarization fraction of a radio relic as a function of the viewing
%angle ($\delta = 90\degree - \alpha$).
% and the compression
%\~{R} of the magnetized region where the relic is generated. 
%. The simplifying assumptions, such as having an
%isotropic distribution of unshocked magnetic fields and electrons etc.,
%represents an idealized case showing maximum possible polarization fraction at a given $\alpha$.  
%% Cosmological simulations of radio
%relic \citep{S13} show a maximum integrated polarization fraction $\sim75\%$ at
%$\alpha = 0$ as predicted by \cite{E98}. 
%After accounting for different spectral
%indices and magnetic field strength, 
%The simplifying
%assumptions, such as having an isotropic distribution of unshocked fields
%and an isotropic distribution of electrons etc. \citep{E98}, gives
%polarization fraction as high as $\sim$ 75\% when $\alpha = 0$. 

%For an
%actual merger, the magnetic field can be less isotropic,  and the resulting polarization fraction at a given $\alpha$ would be lower. This postulate is backed up by the edge-on view of polarization fraction of simulated relics, such as the top left hand panel of figure 9 from Skillman et al. 2013.
%%This model, however, assumes an isotropic distribution of electrons in an isotropic magnetic field. \cite{E98}
%%These mathematical relationships underlies the design of this prior based on observed polarization fraction. The different cases that \cite{E98} considered have different magnetic field strengths and various spectral indices.
%We note that power of polarized synchrotron emission from relativistic electrons has a ratio of 7:1 between parallel polarization and perpendicular polarization. 
%Therefore,   
%
%\par
\textbf{Observation also introduces uncertainties that we have to take into account}. \cite{S13} shows that after convolving the
simulated polarization signal with a Gaussian kernel of 4\arcmin to match
observable resolution, the polarization fraction drops to between 30\% to
65\% even when $\alpha = 0$. 
Other uncertainties come from the fact that the inferred spectral indices
differ between the two observed frequencies and vary between the three
identified relic sources \citep{L13}. 
%\textbf{We pick a form of uniform prior, to represent
%the uncertainties in both the modeling (\citealt{E98}, \citealt{S13}) and the interpretation of the data from \cite{L13}.} 
Following previous discussion, we pick a value of $\mu_\alpha =39\degree +
2 \degree$ to filter realizations, i.e. we do not draw values of $\alpha >
41\degree$. The extra $2 \degree$ in the prior is included to account for the uncertainty of the integrated polarization fraction reported by \cite{L13}. 

%For the width of fall off of the sigmoidal function, we pick
%$\sigma_\alpha = 1\degree$ that corresponds to the uncertainty of the
%integrated polarization fraction reported by \cite{L13}.    

%\begin{itemize}
%\item spectral index of ...
%\item During the merger process, the hot intracluster is cluster merger compresses the magnetic field and orders the polarization.    
%\item \cite{L13} reported that the polarization can constrain viewing angle to be $> 18 \degree $-- check if this viewing angle is defined the same way 
%\item Ensslin 's work which is an application of the theory of
%plane-parallel shock acceleration, which can be justified by the large
%radius of the shock sphere
%\item we consider the most conservative constraint that can be recovered
%from this model, which is strong/weak field case with a spectral index of
%$\alpha_{\text{spectral}}\sim 2$ combined with the observed mean
%polarization fraction of $P \sim 33.3\%$, we recover a  
%\item
%\end{itemize}
%\begin{equation}
%P(\alpha) = 
%\frac{1}{2} - \frac{1}{2} \text{erf}\left(\frac{1}{\sqrt{2}}\frac{\alpha -
%(\mu_\alpha+3\degree)}{\sigma_\alpha}\right)  
%\label{eqn:prior}
%\end{equation}
%
%\noindent See Appendix \ref{app:priors} for a plot of (\ref{eqn:prior}).

%The polarization information has larger constraining power than the .   
%To test the effects of applying the prior on the aforementioned range of
%separation,  we have come up two priors and applied them separately
%%Therefore, the distance between the subclusters, which has to be less than twice the 3D distance between the radio relic from the center of the cluster, is taken conservatively to be $1.0~\mega$pc $<$ d$_{\mathrm{3D}}
%%(t_{\mathrm{obs}}) < 3.0~\mega$pc. 
% to the 3D separation of the subclusters at the time of observation 
%($d_{\mathrm{3D}}(t_{\mathrm{obs}})$):
%
%The effect of the uniform prior is shown in Figure \ref{fig:radioprior}.
%
%%\textbf{description of the radio observation} 
%
%\textbf{how the distances were determined - overview of previous work}
%
