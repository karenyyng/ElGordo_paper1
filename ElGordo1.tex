% mn2esample.tex
%
% v2.1 released 22nd May 2002 (G. Hutton)
%
% The mnsample.tex file has been amended to highlight the proper use of
% LaTeX2e code with the class file and using natbib cross-referencing.
% These changes do not reflect the original paper by A. V. Raveendran.
%
% Previous versions of this sample document were compatible with the LaTeX
% 2.09 style file mn.sty v1.2 released 5th September 1994 (M. Reed) v1.1
% released 18th July 1994 v1.0 released 28th January 1994

\documentclass[letterpaper,useAMS,usenatbib]{"mn2e"}
%\documentclass[letterpaper,useAMS]{"mn2e"}
%LINUX version of the path
%\documentclass[useAMS,usenatbib,letterpaper]{"/media/blank/Macintosh
%HD/Users/karenyng/Library/texmf/tex/latex/commonstuff/mn2e"}

% If your system does not have the AMS fonts version 2.0 installed, then
% remove the useAMS option.
%
% useAMS allows you to obtain upright Greek characters.  e.g. \umu, \upi
% etc.  See the section on "Upright Greek characters" in this guide for
% further information.
%
% If you are using AMS 2.0 fonts, bold math letters/symbols are available
% at a larger range of sizes for NFSS release 1 and 2 (using \boldmath or
% preferably \bmath).
%
% The usenatbib command allows the use of Patrick Daly's natbib.sty for
% cross-referencing.
%
% If you wish to typeset the paper in Times font (if you do not have the
% PostScript Type 1 Computer Modern fonts you will need to do this to get
% smoother fonts in a PDF file) then uncomment the next line 
%\usepackage{Times}

%%%%% AUTHORS - PLACE YOUR OWN MACROS HERE %%%%%
\usepackage{hyperref}
\usepackage{graphicx}
\usepackage{amsmath}
\usepackage[amssymb]{SIunits} 
\usepackage{booktabs}
\usepackage{breqn}
\usepackage{standalone}
\graphicspath{{graphics/}}
%-----------------------------------------------------------------------
\defcitealias{D13}{D13}
\defcitealias{Jee13}{J13}
\defcitealias{M12}{M12}
\defcitealias{Sifon13}{Sif\'{o}n 2013}
\def\apjl{ApJL }
\def\aj{AJ }
\def\apj{ApJ }
\def\pasp{PASP }
\def\spie{SPIE }
\def\apjs{ApJS }
\def\araa{ARAA }
\def\aap{A\&A }
\def\nat{Nature }
\def\mnras{MNRAS }
\def\mnrasl{MNRASL }
\providecommand{\eprint}[1]{\href{http://arxiv.org/abs/#1}{#1}}
\providecommand{\adsurl}[1]{\href{#1}{ADS}}
\providecommand{\ISBN}[1]{\href{http://cosmologist.info/ISBN/#1}{ISBN: #1}} 


%-----------------------------------------------------------------------
\title[El Gordo]{The dynamics and the merging scenario of the galaxy cluster 
ACT-CL J0102-4915, 
El Gordo}
\author[author]{K. Y. Ng$^{1}$, W. A. Dawson$^{2}$, D. Wittman$^{1}$, J.
Jee$^{1}$, J. Hughes$^{3}$, F. Menanteau$^{3}$, C. Sif\'{o}n$^{4}$\\
(temporary order)\\
$^{1}$Department of Physics, University of California Davis, One Shields
Avenue, Davis, CA 95616, USA\\ 
$^{2}$Lawrence Livermore National Laboratory, P.O. Box 808, Livermore, CA
94551-0808, USA \\
$^3$Department of Physics \& Astronomy,
Rutgers University, 136 Frelinghysen Rd., Piscataway, NJ 08854, USA\\
$^{4}$Leiden Observatory, Leiden University, PO Box 9513, NL-2300 RA
Leiden, Netherlands\\}

\begin{document}
%%---------to adjust for the weird bottom margin-----------
%\voffset=-.8in \hoffset=.15in

\date{arXiV 666} \pagerange{\pageref{firstpage}--\pageref{lastpage}}
\pubyear{1988} \maketitle \label{firstpage}
%-----------------------------------------------------------------------
\begin{abstract} 
    
Merging galaxy clusters with radio relics provide rare insights to the merger
dynamics as the relics are created by the
violent merger process. 

From the double radio relic observation and X-ray wake morphology, 
it is believed that El Gordo is observed shortly after the first passage
before reaching apo

We demonstrate one of the first uses of the
properties of the radio relic
to reduce the uncertainties of the dynamical variables 
and 3D configurations of a cluster merger, ACT-CL J0102-4915, El Gordo. 
At a redshift of 0.87, El Gordo (M$_{200c} = 
2.75\times10^{15} \pm^{7.4}_{1.5}$ M$_{\sun}$) is one of the most massive
clusters discovered in the early universe. The two subclusters of El
Gordo has a mass ratio of around 2:1. 
The X-ray and weak-lensing data of El Gordo show an offset of X kpc between
the intercluster gas and the dark matter (DM) at $\sim$4 $\sigma$ level.
All these features of El Gordo make it part of a valuable class of
dissociative mergers that can probe the self-interaction of dark matter.
%As more and more cluster mergers are being discovered using
%the radio relic emission detected in upcoming large scale radio surveys,
We employ a Monte Carlo simulation to investigate the three-dimensional (3D)
configuration and dynamics of El Gordo. 
We give a summary of the inferred
dynamical variables. By making use the polarization, velocity and position
of the radio relic, we are able to confirm at X $\sigma$ that the subclusters of El Gordo are moving away from each other. We find that
the 3D merger speed of El Gordo to be $\sim3000~\kilo\meter~
\second^{-1}$ (or in projected velocity = ), which is still consistent with the low line-of-sight
velocity of $\sim600~\kilo\meter~\second^{-1}$ based on the inferred time-since-collision ($TSC$ = Gyrs) and
the projection angle (\(\alpha = 41^{\circ}\pm \)). We put our estimates of $TSC$ and $\alpha$ into context by relating them to existing observations of El Gordo. 
Finally, we compare our simulation result of El Gordo to the simulation
result of the Bullet Cluster, and show that
El Gordo is a very promising candidate for giving tigher constraint than
the Bullet Cluster on the self-interaction of dark matter. 
(200 words)
(check against astro-ph word limit)
%\textbf{Findings}\\ 
%We found that the merger speed at the collision of El Gordo
%($\mathrm{km/s}$) is higher than those of the Bullet Cluster.  

%We estimate El Gordo to have a slightly higher TSC than the bullet cluster
%using the same analysis by Dawson (2012).  
%To be continued.  (250 words) 

\end{abstract}
%-----------------------------------------------------------------------

\begin{keywords}
gravitational lensing -- dark matter -- cosmology: observations -- X-rays:
galaxies: clusters -- galaxies: clusters: individual (ACT-CL J0102-4915) --
galaxies: high redshift 
\end{keywords}

%-----------------------------------------------------------------------
\section{Introduction} 
%\textbf{background - why do merging clusters of
%galaxies are worthy of investigating}
%Clusters of galaxies are some of the most interesting astrophysical
%laboratories. The environment of the cluster affects the evolution of the
%galaxy members. Redder galaxies are located closer to the dynamical
%centers of galaxies while bluer star-forming galaxies are located at the
%edge of clusters.

%----------------------------------------------------------------------
% To do:
% - make sure that when introduce the 2 subclusters - talk about the 
%   NW and SE notation
% - add Williamson 's paper as one of the reference for the first mention
%   of El Gordo
%----------------------------------------------------------------------

%\textbf{to motivate why we want to do this study} 
Mergers of dark-matter-dominated galaxy clusters probes properties
of the cluster components like no other systems. 
Clusters of galaxies are made up of 80\% of dark matter in mass content, 
with a smaller  portion of intercluster gas($\sim 15\%$ in mass content), and
sparsely spaced galaxies ($\sim 2\%$ in mass content) (REF). During a merger of
clusters, the subclusters are accelerated to high speeds of several
thousand \kilo \meter~\second$^{-1}$. The offsets of different components
of the subclusters dissociate show how various interactions of the different
components are at work. Observables such as offset between dark
matter and the other components may suggest dark matter self-interaction
(REF).  (The following sentence does not actually fit in this paragraph and
I have to put it somewhere else) difference of the galaxy colors in a merging cluster from relaxed cluster can also verify effects of environment on galaxy evolution.\par
%\textbf{background of El Gordo}
%\textbf{What literature exists for El Gordo.}

%van Weeren 2011a suggests that the double radio relic can provide clue to
%collisional parameters ???
Ever since the discovery of El Gordo in the Atacama Camera Telescope (ACT)
survey (REF), there is an ongoing effort for collecting comprehensive data
for El Gordo.
The presence of the radio relic, in
conjunction with a depression in the X-ray surface brightness shown in M11,
strongly suggest that El Gordo is a post-collision system so we limit our
discussion to inferring the time-since-collision. 

From the spectroscopy and Dressler-Schecter test for the member galaxies  in Sif\'{o}n et al. (2013), El Gordo is confirmed to be a binary merger 
without significant substructures. This picture is further supported by the
weak lensing analysis by Jee et al. (2013). The weak lensing analysis shows
a mass ratio of $\sim$2:1  between the two main subclusters, named according to their location as the northeast (NW) and southeast (SE) subclusters respectively. 
(See Figure \ref{fig:config}). El Gordo has interesting intracluster medium morphology as shown in the X-ray. In the northwest, it shows a wake feature, i.e.,
depression in the X-ray emissivity, while in the southeast, it shows
highest X-ray emissivity indicative of a cold gas core southeast of the
wake. The cold gas core may have passed from the northwest to the southwest
to have caused this morphology (Menanteau et al. 2011, hereafter M11). 
The extended mass distribution of El Gordo also makes it a good
gravitational lens. Zitrin et al. (2013) have found multiple strong
gravitationally lensed images around the center region of El Gordo. 
On the outer skirt of El Gordo, strong radio emission is detected in
the NW and the SE respectively. These radio emission has steep spectral
index gradient and are identified as radio relic created from a merger.\par 
%\begin{itemize}
%\item explains the 2D configuration of El Gordo - bimodal
%\item evidence that it is a merger 
%goes along a line joining northwest to southeast. 
%\item explains the spectroscopic surveys which confirms that that there are
%not a lot of line of sight substructures and that the line-of-sight
%velocity difference of the  
%\item strong lensing - many many arcs and elongated lens weak-lensing -
%bimodal distribution of matter  (describes what data has been available for 
%El Gordo) Since the
%discovery, an abundance of multi-wavelength data of El Gordo has been
%analyzed according to the two-dimensional projected view. \par 
%\item explain the wake feature (depression in X-ray) and the cold core in
%the SE which has high X-ray emission and point out it shows a recent merger 
%unlike the bullet cluster, El Gordo lacks a prominent X-ray shock
%\end{itemize}\par
%\textbf{Explains the importance of radio relic in constraining the merger
%physics.} Radio relics are direct products of the merger process itself.
%
%\begin{itemize}
%
%\item explains the double radio relic and their formation
%synchrontron emission in the radio wavelength with steep falloff of
%spectral indices.
%\item explains compression of magnetic field would cause polarization
%signal perpendicular to the merger axis 
%\item explains the current status of simulations in understanding radio
%relic
%\item  
%\end{itemize}
\begin{figure}
	\includegraphics[width=\linewidth]{ElGordo.png}
	\caption{Configuration of El Gordo (to decide which figure to use,
	this one is from Lindner et al.) \label{fig:config}}
\end{figure}
El Gordo is one of small sample of galaxy clusters ($\sim 50$) that have
been associated with a radio relic. (This paragraph needs a lot more
organization) Even fewer of them have been studied in
great details, making El Gordo a valuable candidate for further analysis. 
%Cosmological simulations have largely complement the studies of the
%observed radio relics.  
%The study of radio relics is just taking off. 
%Cosmological
%simulations have also been employed to study the properties of the radio
%relics. (explain how both observations and simulations confirm radio relics
%are due to a cluster merger and that double relics are proposed to
%correspond to binary mergers)
%Radio relic, also known as radio shockwaves, are created during the violent
%merger of clusters of galaxies (See Ensslin's paper for a review of the
%physics). 
%Such double radio relics are also exhibited in other confirmed merging clusters of galaxies, such as REF missing etc. 
%
%(more description of how the paper comes up with emission in their simulations)
%
%\textbf{(Need to add transition from background to this paragraph to
%motivate why we have to study El Gordo) 
Furthermore, El Gordo satisfies the four criteria for being a dissociative merger which are proposed to be excellent
probes of self-interacting dark matter (Dawson et al. 2012) . (1) The subclusters
of El Gordo has a small ratio of mass, i.e. $\sim 2:1$ (Jee et al. 2013,
hereafter J13). (2) The merger axis, the line joining the two subclusters,
coincides with the alignment of the double radio relic propagating outward at the periphery of the cluster (Menauteau et al. 2012,
hereafter M12). This suggests a simple merger configuration with small
impact variables.  (3) The X-ray luminosity peak is shown to be offset
from the weak-lensing peak by X kpc at X $\sigma$ level (J13). (4) The
observation of the double radio relic suggests that the angle between the
merger axis and the plane of the sky has to be reasonably small (M11,
Lindner et al. 2013), or else
the relic may appear as a halo instead. \citep{S13} \par 


%\textbf{motivation} 
%\textbf{summarizes what is missing for the understanding of the merger of El
%Gordo, which is the projection angle and the time-since-collision.}
In this paper, we perform results of simulations for modeling the time
evolution of the mergers. 
Determining the time-since-collision of mergers of similar clusters helps
us reconstruct different stages of a cluster merger.
Mergers of clusters proceed on the time-scale of millions of year,
observations of each cluster only provides a snapshot of a particular type
of merger. In order to understand the merger process observationally, 
we need to capture
and identify different stages of similar dissociative mergers. \par 

Another crucial piece of missing information is the 3D
configuration, i.e. the projection angle $\alpha$, which contributes the
largest amount of uncertainties to the dynamical variables \citep{D13}.
With a large projection angle $\alpha$, the radio emission may appear as a
radio halo instead.  \citep{S13}\par 
%\textbf{motivation2}
This work is particularly important since it is forbiddingly
expensive to simulate clusters similar to El Gordo in high resolution. 
The probability for finding an analog of El Gordo in a cosmological
simulation is as low as \% \citepalias{M11}. A realistic cosmological simulation of
El Gordo is thus computationally expensive. Under the hierarchical picture
of structure formation in the $\Lambda$CDM model, there is a rare
chance for massive clusters like El Gordo to have formed at a redshift of
$z = 0.87$.  Staged simulation would not be able to probe the angular
dependence. 
Both weak lensing analysis and BLAH DATA of El Gordo \citep{Jee13} has revealed a
relatively simple bimodal mass distribution.  The lack of complex
substructures makes modeling of El Gordo with only two subclusters possible.

\par

In this paper, we adopt the following conventions: (1) we
assume the standard $\Lambda$CDM cosmology with $\Omega_{m} = 0.3$, $\Omega_{\Lambda} = 0.7$. (2) All confidence intervals are quoted at the 68\% level unless otherwise stated. 
(3) All credible intervals (a.k.a. Bayesian confidence intervals that also
takes into account prior probability) are also
quoted at the 68\% level unless otherwise stated and are central credible
intervals. (4) All quoted masses ($M_{200c}$) are based on mass contained
within $r_{200}$ where the mass density is 200 times the critical density
of the universe ($\rho_{crit}$) at the redshift of $z = 0.87$. 
%(5) We
%demonstrate that the application of a uniform sampling PDFs derived due to the
%integrated polarization fraction of the radio relic does not introduce
%large biases into our estimators in section \ref{sec:relic}, and unless
%otherwise stated, the simulation results we quote are those after applying
%the uniform radio relic filter.  \par 


%-----------------------------------------------------------------------
\section[]{METHOD -- Monte Carlo simulation} 
For this analysis, we made use of the collisionless 
dark-matter-only Monte Carlo modeling code written by Dawson (2013),
hereafter \citepalias{D13}.  
In the code, the time evolution of the head-on merger was computed
based on an analytical model assuming that the only dominant force is the gravitational attraction from
the masses of two truncated Naverro-Frenk-White (hereafter NFW) DM halos.
Other major assumptions for modeling systems with this code include
negligible impact parameter and no self-interaction of dark matter.\par

In the Monte Carlo simulation, many realizations of the collision is
computed from the inputs of each realization, including
the data ($\vec{D}$) and the model variable ($\alpha$). In particular,
the standard required data, which were in the form of samples of the probability density
functions (PDFs), included the masses ($M_{200_{NW}},M_{200_{SE}}$) the
redshifts ($z_{NW}, z_{SE}$) and the projected separation of the two
subclusters ($d_{proj}$).  
%For the $j-$th realization, w
In each realization, we randomly drew samples of the PDFs.
%
%\begin{equation}
%	D_i^{j} \sim \mathcal{L}(\vec{\theta}|D_i) =  P(D_i | \vec{\theta})
%\end{equation}
%and we also draw the model variable $\alpha$ from the prior:
%\begin{equation}
%	\alpha^j \sim P(\alpha)
%\end{equation}
These inputs are then used for computing the output variables
($\vec{\theta}^\prime$) by making use of conservation of energy to describe
their collision due to the mutual gravitational attraction.
%\begin{equation}
%(\vec{\theta}^\prime)^{(j)} = f(\vec{D_i}^j, \alpha^j) 
%\end{equation}
%where $f$ are some suitable functions expressing the conservation of energy.
(See Table \ref{tab:inputs}
for quantitative descriptions of the sample PDFs and we outline how those
PDFs are obtained in the following subsections.) 
To ensure convergence of the output PDFs, in total, 2 million (to be
confirmed) realizations were computed. The results, however, are
consistent up to a fraction of a percent just from 20 000 runs
\citepalias{D13}.\par    
We note that the Monte Carlo simulation is described from a Bayesian
point of view but the analysis differs from conventional Bayesian inference. The Bayes
chain rule underlies the simulation can be written as:
\begin{equation}
    P(\vec{\theta}|\vec{D}) \propto P(\vec{D}|\vec{\theta})P(\vec{\theta})
\end{equation}
where the likelihood is defined to be the PDF of $\vec{D}$ given $\vec{\theta}$,
i.e. the input variables, not statistical parameters, and the priors are
defined to be the probabilities due to prior knowledge of the estimated values of
$\vec{\theta}$. The output variables $\vec{\theta}^\prime$, on the other
hand, were computed according to the conservation of energy, which is
represented by a suitable functional form $f$ below. For example,the
calculation of the output variables of the $j$-th realization can be denoted as: 
\begin{equation}
    (\vec{\theta}^\prime)^{(j)} = f(\vec{\theta}^{(j)}, \vec{D}) 
\end{equation}    
and computed over all $j$ realizations. Finally, we took the physical
constraints on $\vec{\theta}$ and $\vec{\theta}^\prime$ into account by
examining the resulting physical variables against the physical limits and
excluding realizations that would produce impossible values. We refer to this
process of excluding realizations as ``applying prior probability''. 

%To model projection effects, we randomly draw a projection
%angle in each realization. We throw out realizations with unphysical
%outputs.  \textbf{what are the inputs}


\subsection{Inputs of the Monte Carlo simulation}
\label{sec: inputs}
\setcounter{table}{0} 
\begin{table} 
\caption{Properties of the sampling PDFs of the Monte Carlo simulation }  
\begin{center} 
\begin{tabular}{@{}lcccc}
\hline Data & Units & $\mu$ & $\sigma$ & Ref\\ \hline
$M_{200c_{\mathrm{NW}}}$ & $10^{14}$ M$_{\odot}$ & & & \citetalias{Jee13}\\ 
c$_{\mathrm{NW}}$ & / & & & \citetalias{Jee13} \\ 
$M_{200c_{\mathrm{SE}}}$ & $10^{14}$ M$_{\odot}$ & & & \citetalias{Jee13}\\
$c_{\mathrm{SE}}$ & / & & & \citetalias{Jee13}\\ 
$z_{\mathrm{NW}}$ & / & 0.86901 & 0.00017$^b$  & \citetalias{M11},
\citetalias{Sifon13}\\
$z_{\mathrm{SE}}$ & / & 0.87175 & 0.00019$^b$  & \citetalias{M11},
\citetalias{Sifon13}\\ 
d$_{\mathrm{proj}}$ & Mpc & & & \citetalias{Jee13} \\ 
\hline
\end{tabular} 
\end{center} 
\label{tab:inputs} 
\footnotesize{
$^a$This $\sigma$ corresponds to the $68\%$ central Bayesian
credible interval computed from the posterior probability of our MCMC
analysis.\\
$^b$This $\sigma$ corresponds to the biweight scale. \\
$^c$We use the full PDFs as the inputs of our simulation so
different ways of denoting the uncertainties do not affect the simulation.\\ 
%\textbf{References:} (1) Jee et al. 2013 (2) Menanteau et al. 2011}
}
\end{table}


%-----------------------------------------------------------------------
\subsubsection{Membership selection and redshift estimation of subclusters}
% \textbf{how do we determine the overall membership}
% relevant info of the membership is in M11 table 1
% cannot determine the membership of two galaxies 
% spatial cut is defined in M11 p.9 first paragraph 

% paragraphs needs to be reorganized 
The overall membership of the galaxies
of El Gordo was first determined using a shifting gapper method
\citep{Fadda96} after applying a rest frame cut of
4000~\kilo\meter~\second$^{-1}$ \citep{Sifon13}. This method gives a
total count of 89 galaxy members of El Gordo.
To further distinguish member galaxies of each subcluster, we adopt a spatial cut approximately perpendicular to the 2D merger axis
from \citepalias{M11}. Using this cut, we determined that there are 54
members in the NW subclusters and 35 members in the SE subclusters (See Figure
\ref{fig:membership}). The spatial cut indicated by the green line was done after
mapping the world coordinates to pixel coordinates to avoid anamorphic distortion. 
Then, we bootstrapped the biweight locations of the redshifts of the
respective members in order to obtain the PDFs of the redshifts of each
subcluster 
\citep{Sifon13}. Using this method of determining subcluster members, the spectroscopic
redshift of the subclusters were determined to be $z_{NW} = 0.86901 \pm
0.00017$ and $z_{SE} = 0.87175 \pm 0.00019$, where the quoted numbers represent the biweight location and
biweight scale respectively \citep{Beers90}. 
These biweight location
estimators are less susceptible to outliers than the mean and standard
deviations. These redshifts of the subclusters correspond to a relative
velocity difference of $476$ km/s. 

\begin{figure}
	\includegraphics[width = \linewidth]{confirmed_member_divide.png}
	\caption{\label{fig:membership} The division of
the member galaxies among the two subclusters of El Gordo by a spatial cut
(green line). The color bar shows the color mapping of the spectroscopic
redshift of the member galaxies, with the redder end indicating higher
redshift.} 
\end{figure}

%We confirm the existence of the subclusters of El Gordo from the 2D spatial
%location of the galaxies, with the more massive cluster lying in the
%northwest (NW) and the less massive subcluster lying in the southeast(SE)  (REF - also have to reference the other literature
%for confirmation in the other wavelengths)
%
%Specifically, the membership of the NW and SE subclusters are
%determined based on a combination of the redshift and the spatial location. 


\begin{figure}
	\includegraphics[width = \linewidth]{confirmed_member_divide.png}
	\caption{\label{fig:membership} The division of
the member galaxies among the two subclusters of El Gordo by a spatial cut
(green line) that is approximately perpendicular to the 2D merger axis.
The color bar shows the color mapping of the spectroscopic redshift of the
member galaxies. The spatial cut is done after mapping the world coordinates to
pixel coordinates to avoid anamorphic distortion.}
\end{figure}

%-----------------------------------------------------------------------
\subsubsection{Weak lensing mass estimation} 
%The input 
We obtained the PDFs of the masses of the subclusters by doing a Monte
Carlo Markov Chain (MCMC) analysis of the reduced shear from the
weakly lensed background galaxies similar to \citet{Dawson12}. We computed the reduced shear signal
generated by two NFW halos according to \citet{Umetsu10} (See Appendix
\ref{app:MCMC} for
details of implementation and output diagnostics).
At each step we followed the procedure of a
Metropolis algorithm.  The transition kernel was set to
be the log likelihood of fit of the model shear to the reduced shear of the
data (\ref{eqn:jointposterior}).
In total, eight MCMC chains were used. After every 5000 MCMC steps for all
the chains, we computed the R coefficient \citep{Gelman92}  to
check for convergence. We performed more MCMC steps as long as convergence
was not achieved. After convergence was achieved, we removed the
burn-in portions of the MCMC chains and used the resulting MCMC chains as
samples of the PDFs of the masses. \par 
% this is from Jee13 section 3.5
We make use of an effective redshift of $z_{\text{eff}} = 1.37$ or $D_{LS}
/ D_L = 0.276$ \citepalias{Jee13} and $g'\approx(1 + 0.79 \kappa)g$
(\citetalias{Jee13}, \citealt{Seitz97}).    
%\textbf{The inputs of our MCMC mass inference are from proposal blah,
%which is similar to \citepalias{Jee13}, however, this
%separate implementation of the MCMC analysis code is different than
%\citepalias{Jee13}.} 
We used a catalog of reduced and bias-corrected background 
galaxy shapes from Hubble Space Telescope PROP 12755, and these galaxies
were discussed as the population in Region A in
\citetalias{Jee13}. (! \citealt{Jee13} actually used
additional data) On the other hand, we fixed the
position of the centers of the NFW halos to be  the luminosity peaks of the
respective galaxy populations of  the two subclusters, which are at R.A. $=
01$:02:51.68, Decl. = $-49$:15:04.40 and R.A. = $01$:02:38.38, Decl. =
$-49$:16:37.64 for the NW and SE subclusters
respectively \citepalias{Jee13}. The separations between the luminosity
peaks and the estimated mass centroids of the subclusters are $X kpc$ and
$X kpc$ respectively for the NW and SE subcluster.  (! Lori and Nick actually asked why we do not free
the centroids like Jee 13)  The agreement between our analysis and \citepalias{Jee13} to within
the 68\% credible interval serves as a sanity check on the estimated masses. 
%Other sanity check including the control of acceptance rate between 20\%
%and 50\%. 

The mass estimates from \citealt{Jee13} are $M_{200c} = 13.8~\pm~2.2\times
10^{14}~h_{70}^{-1} M_{\sun}$ for the NW subcluster and $M_{200c} = 7.8~\pm
2.0\times10^{14}~h_{70}^{-1} M_{\sun}$ for the SE subcluster. 


%----------------------------------------------------------------------

\subsubsection{Estimation of projected separation ($d_{proj}$) } 
To be consistent with our MCMC mass inference, our Monte Carlo simulation takes 
the projected separation of the NFW halos to be those of the two aforementioned 
luminosity peaks.
%A iterative centroiding method is used to find the centers of the two
%subclusters of El Gordo.

\subsection{Outputs of the Monte Carlo simulation}
\label{sec: outputs}
We outline the outputs of the simulation here to facilitate the discussion
of the design of the priors used in the simulation. The simulation
provides PDF estimates for many of the output variables. Variables
of the most interest include the time dependence and $\alpha$, which is
defined to be the projection angle between the plane of the sky and the merger axis. Other output variables are dependent on $\alpha$ and the time
dependence. Specifically, the simulation denotes the time dependence by
providing several characteristic time-scales, including the time
elapsed between the collision and when the subclusters first reach apoapsis
($T$) and the time-since-collision.  

The two versions of the time-since-collision variables $TSC_0$ and
$TSC_1$ denotes different possible merger scenarios. 1) We call the scenario for which the subclusters are
moving apart after collision to be ``outgoing" and it corresponds to the
smaller $TSC_0$ value, and 2) we call the alternative scenario 
``incoming" for which the subclusters are approaching each other after turning
around from the apoapsis for the first time and it corresponds to $TSC_1$.
We describe how we use to break the degeneracies of the two scenarios in
section \ref{sec: positionprior}. 
 
The simulation also output estimates of variables that characterize
the dynamics of the merger. The 3D velocities, both at the time of the
collision ($v_{3D}(t_{col})$) and at the time of observation
($v_{3D}(t_{obs})$) are provided. The maximum 3D separation ($d_{max}$),
which is defined to be the distance between the position of collision to
the apoapsis, is also part of
the outputs. (See the lower half of Table \ref{tab:outputs} for all the outputs).
%Here we present results based on:\\  %1) a flat radio prior\\
%2) a uniform prior over a range of most likely 3D separations\\
%3) a Gaussian prior  
%We discuss in subsection \ref{sec:priors}  on the use of the default filters
%and two new filters designed according to the observed data and the physics of the radio relic.
%
%\textbf{While the underlying formalism of the Monte Carlo simulation is
%    based on the Bayes theorem, we caution the reader that this simulation
%    does not correspond to a conventional Bayesian parameter estimation but
%    more similar to the Bayesian uncertainty estimation method mentioned in Saltelli 
%    et al. (2004). (See appendix \ref{} for a more in-depth discussion)}



%-----------------------------------------------------------------------
\subsection{Design and application of priors} 
\label{sec:priors}
The strength of the Monte Carlo simulation by \citetalias{D13} is its ability
to detect and rule out extreme input values that would result in
unphysical realizations via the application of prior probability. 
Our default Monte Carlo priors are described in D13 and in Appendix
\ref{app: results}. We also examine the effects of applying 
two priors derived based on the position and the integrated polarization
fraction of the radio relic of El Gordo respectively. 
We considered other properties of the radio relic to be used as prior
information, such as the physical location. Due to large uncertainty in the
projection angle that affects the prior information derived from the prior
information, we have not included the corresponding numerical results.
\par 
%(Not sure if the following fits best here but it is definitely an intro to
%the priors) 
El Gordo shows radio relics on the periphery of both subclusters
\citepalias{M11}. The
radio relic  of El Gordo was first mentioned in the Sydney University
Molonglo Sky Survey (SUMSS) data in low resolution at 843 MHz
\citep{Mauch03} as shown in M11. The higher resolution radio observation
conducted by \cite{L13} at 610 \mega Hz and 2.1 \giga Hz confirms that the identity of the radio relic
after removing effects of radio point sources. 
Three main sources of radio relic were identified, including the NW, SE and the
E relic. The NW radio relic possesses the most extended geometry among all
the identified relic source. We do not refer to the The SE nor the E radio
relic in our calculation since we do not have an estimation of the shock
speed of the SE relic nor the E relic from \citet{L13} for comparison.    
%Radio relics have been suggested to be able to constrain the mass ratios,
%the projection and the merger configuration \citep{vanWeeren10}. 

%%%%%%%%%% WAIT WHAT AM I TRYING TO SAY????
%Ever since the first detection of radio relic, cosmological hydrodynamical simulations of
%merging clusters have been used to model their emission spectrum and
%geometry. (\citealt{Vazza11}, \citealt{VanWeerenRJ2011}, Bonafede
%et al. 2013, \citealt{E98}, Br\"{u}ggen et al., Skillman et al.
%2013) While such cosmological simulations have provided valuable insights
%to verifying the physical models, they are expensive in terms of
%computational power and novel techniques have to be invented in order to
%analyze the large amount of simulated data so progress has been slow. 
%Our Monte Carlo simulation can make use of known physics combined with the
%preliminary results from such cosmological simulations to use properties of
%the radio relic to constrain merger dynamics. 

%Compared to hydrodynamical simulations or cosmological simulations, this
%    Monte Carlo simulation is not demanding in terms of CPU time, therefore, we
%    can run many realizations in order to probe how the input variables
%affect the output variables. 

%\begin{itemize}
%\item talks about the observable, which is the comoving kinetic power through each shock surface
%\item refer to diffusive shock acceleration (DSA) mechanism?
%\item Kang \& Jones treatment of Mach number-dependent efficiency
%considering the possibility of having an non-isotropic magnetic field  
%$KJ_BparallelRadial$ model
%\end{itemize}
%\begin{figure}
%	\includegraphics[width=\linewidth]{d_3d_prior1.png}
%	\caption{The marginalized output PDFs of the observed 3D separation
%		($d_{3D}$) 
%		with and without the radio prior applied. 
%		(maybe I should replot this more nicely without too many
%		distracting lines but only the lines showing the location)
%		%The vertical
%		%lines denote, dashed line: biweight location, dash-dot
%		%line: 68\% credible limit, dotted line: 95\% credible
%		%limit.
%	\label{fig:radioprior}}
%\end{figure}


%\subsubsection{Weighting function based on the observed position of the radio relic}\label{sec:relic} 
%-----------------------------------------------------------------------
%Among the known galaxy cluster mergers that are associated with radio
%relics, \cite{Vazza12} noted that most of them have radio relic located
%more than 800 \kilo pc away from the merger center. \cite{Vazza12} then
%conducted hydrodynamical simulations of twenty of known galaxy mergers with 
%radio relic to investigate this observed trend. They found a radial
%trend of kinetic power dissipation increasing up to around half the virial
%radius (r$_{vir}$) of the cluster. Summarizing the results from the proposed model for energy dissipation of the radio relic, \cite{Vazza12} gives the range of highest kinetic power emission in a range of 
%$.2 ~r_{\mathrm{vir}} < d_{\mathrm{3D}} < .5~r_{\mathrm{vir}}$.
%
%
%% In particular, Vazza et al. (2011) showed dependence of observed
%%location of radio relic: when the clusters are at small separation, the
%%Mach number is too high for a radio shock to form and the steep fall off of
%%the emission power of the radio relic as a function of separation makes it
%%difficult to observe a radio relic when it has propagated beyond a certain
%%separation.  
%
%%\textbf{We take into account the uncertainties of their modeling and 
%% construct prior probability on a range of 3D separation for which the kinetic
%%power dissipation of the radio relic is more than 10\% of the peak value.} 
%Since we do not have information on how the probability of
%being able to observe the relic would fall off as a function of emission power, we adopt a conservative approach and designed a uniform prior. 
%%and contrast that to a flat prior to test the effect of the prior on the output variables. 
%We also take into account the uncertainties of the different proposed power
%emission model and come up with a prior of:
%\[
% \text{P}({d_{3D}}) = 
%\begin{cases} 
%\text{constant,} & \text{for 1.0  Mpc} < d_{3D}(t_{\mathrm{obs}}) < 3.0 \text{ Mpc}\\
%0, & \text{otherwise}
%\end{cases}
%\]
% for El Gordo.\par 
%%\begin{equation}  
%%P(d_{3D}(t_{\mathrm{obs}})) = 
%%\begin{cases}
%%1/C, \text{ if }1.0 \text{ Mpc } < d_{3D}(t_{\mathrm{obs}}) < 3.0 \text{ Mpc} \\ 0, \text{ otherwise}
%%\end{cases}
%%\end{equation}
%
%\begin{figure}
%	\includegraphics[width=\linewidth]{alpha_pdf_prior_diff.png}
%	\caption{The projection angle with and without the radio relic
%prior applied. (Needs to update and label the figure better)} 
%\end{figure}



%---------------------------------------------------------------------------
\subsubsection{Monte Carlo filters based on the integrated polarization fraction of the radio relic}
We can relate the polarization fraction of the radio relic to the
projection angle by examining the
generating mechanism of the radio relic.
The observed radio relic is due to synchrotron emission of free electrons in a
magnetic field. If the magnetic field is uniform, the observed
polarization fraction of the synchrotron emission of the electrons depends on the
viewing angle (or equivalently the projection angle) with respect to the alignment of the magnetic field. 
Synchrotron emission from electrons inside unorganized magnetic field are
randomly polarized. The high reported integrated polarization fraction from
\citet{L13} can be explained by a highly aligned magnetic field,
created by the compressed intracluster medium during a merger
(\citealt{E98}, \citealt{vanWeeren10}, \citealt{Feretti12}).
This picture is consistent with a high polarization fraction perpendicular
to this magnetic field along the relic. 
\par
We designed our prior to reflect how $\alpha$ decreases monotonically as the
maximum observable integrated polarization fraction. 
\begin{figure}
	\includegraphics[width=\linewidth]{Ensslin_polar_fig.pdf}
	\caption{Predictions of polarization percentage of the radio relic at a
		given projection angle from different models, reproduced from
		\citealt{E98}. Each model assumes electrons producing the radio emission
		to be accelerated inside
		uniform magnetic field of various strengths ({\it strong} or {\it weak}). The curves are plotted with spectral index of the radio emission
		($\alpha_{radio}$), spectral index of the electrons ($\gamma$) and
		compression ratio of the magnetic field ($R$) corresponding to the
		estimated values from \citet{L13}.
		We highlight the observed polarization percentage of the main NW radio relic
		of El Gordo by the dotted vertical line with the greyed out region
		indicating the uncertainty \citep{L13}.\label{fig:Ensslin_fig}}
\end{figure}
This assumption is based on the class of models given by \cite{E98}(See
Figure~\ref{fig:Ensslin_fig}). In particular, we refer to a model from \cite{E98} that would give the most
conservative estimate on the upper bound of $\alpha$:
\begin{align}
\alpha &= 90 \degree -
\arcsin
\left(
\sqrt{
\frac{
	\frac{2}{15} \frac{13R - 7}{R - 1} \frac{\gamma + 7/3}{\gamma + 1}
	\langle P_{strong} \rangle}{
	1 + \frac{\gamma + 7/3}{ \gamma +1} \langle P_{strong} \rangle }}\right)
\end{align}


This model
corresponds to the strong field case with the relic being supported by
magnetic pressure only, with $\alpha_{radio} = 0.86$, compression ratio
$R=2.7$ and $\gamma = 2.7$. 
This model predicts a maximum integrated polarization fraction of
$\sim60\%$ when $\alpha \rightarrow 0$. From this model, the observed integrated
polarization fraction of $33\%\pm1\%$ corresponds to an estimated value
of $\hat{\alpha}
 = 35\degree$. 
%We consider 39\degree as an upper bound on the projection angle since this idealized model assume isotropic distribution of magnetic field and
%electrons. 
This  polarization fraction of $\sim 60\%$ predicted by \citep{E98} is
consistent with the upper bound of relic polarization fraction in cosmological
simulations \citep{S13}. No other model of the magnetic field should predict a higher polarization fraction, thus it is highly unlikely that we see 33\%
integrated polarization at $\alpha > 35\degree$.  
\par

We cannot rule out $\alpha \leq 35\degree$ as a result of possible
variations in the magnetic field. 
\cite{E98} assumes an isotropic distribution of electrons in an isotropic magnetic field. Cosmological
simulations of radio relics from \cite{S13} show varying polarization
fraction across and along the relic assuming $\alpha = 0$, resulting in a
lower integrated polarization fraction. For example, it is possible to see a edge-on radio relic ($\alpha = 0$) with integrated polarization fraction of 33\%. 
Furthermore, \cite{S13} shows that after convolving the
simulated polarization signal with a Gaussian kernel of 4\arcmin~to
illustrate effects of non-zero beam size, the polarization fraction drops to between 30\% to
65\% even when $\alpha = 0$. 
Other uncertainties come from the fact that the inferred spectral indices
differ between the two observed frequencies and vary between the three
identified relic sources \citep{L13}. We examine the effects  of changing
the cutoff value of this prior to ensure the uncertainties do not
introduce significant bias in the estimated output variables and we
present the results in Appendix \ref{app: results}.
To summarize, we adopt a conservative uniform prior to encapsulate the
information from the polarization fraction of the radio relic as:
\begin{equation}
P(\alpha) = 
	\begin{cases}
	& \text{const. $>$ 0 for  }\alpha < 35 \degree \\ 
	& 0 \text{ otherwise}
	\end{cases}
\end{equation}


%Due to these likely variations in the true magnetic field, the true observable integrated polarization values at a given $\alpha$ can be lower than what is predicted by \cite{E98}. 
%For example, it is possible that the radio relic of El Gordo has a lower maximum face-on polarization fraction than 75\%, but if we are viewing the relic at a smaller $\alpha$, the integrated
%polarization fraction can still comes out to be 33\%.

%With simplifying assumptions, \cite{E98} have derived the integrated polarization fraction of a radio relic as a function of the viewing
%angle ($\delta = 90\degree - \alpha$).
% and the compression
%\~{R} of the magnetized region where the relic is generated. 
%. The simplifying assumptions, such as having an
%isotropic distribution of unshocked magnetic fields and electrons etc.,
%represents an idealized case showing maximum possible polarization fraction at a given $\alpha$.  
%% Cosmological simulations of radio
%relic \citep{S13} show a maximum integrated polarization fraction $\sim75\%$ at
%$\alpha = 0$ as predicted by \cite{E98}. 
%After accounting for different spectral
%indices and magnetic field strength, 
%The simplifying
%assumptions, such as having an isotropic distribution of unshocked fields
%and an isotropic distribution of electrons etc. \citep{E98}, gives
%polarization fraction as high as $\sim$ 75\% when $\alpha = 0$. 

%For an
%actual merger, the magnetic field can be less isotropic,  and the resulting polarization fraction at a given $\alpha$ would be lower. This postulate is backed up by the edge-on view of polarization fraction of simulated relics, such as the top left hand panel of figure 9 from Skillman et al. 2013.
%%This model, however, assumes an isotropic distribution of electrons in an isotropic magnetic field. \cite{E98}
%%These mathematical relationships underlies the design of this prior based on observed polarization fraction. The different cases that \cite{E98} considered have different magnetic field strengths and various spectral indices.
%We note that power of polarized synchrotron emission from relativistic electrons has a ratio of 7:1 between parallel polarization and perpendicular polarization. 
%Therefore,   
%
%\par
%\textbf{We pick a form of uniform prior, to represent
%the uncertainties in both the modeling (\citealt{E98}, \citealt{S13}) and the interpretation of the data from \cite{L13}.} 
%Following previous discussion, we pick a value of $\mu_\alpha =39\degree +
%2 \degree$ to filter realizations, i.e. we do not draw values of $\alpha >
%41\degree$. The extra $2 \degree$ in the prior is included to account for the uncertainty of the integrated polarization fraction reported by \cite{L13}. 
%
%For the width of fall off of the sigmoidal function, we pick
%$\sigma_\alpha = 1\degree$ that corresponds to the uncertainty of the
%integrated polarization fraction reported by \cite{L13}.    

%\begin{itemize}
%\item spectral index of ...
%\item During the merger process, the hot intracluster is cluster merger compresses the magnetic field and orders the polarization.    
%\item \cite{L13} reported that the polarization can constrain viewing angle to be $> 18 \degree $-- check if this viewing angle is defined the same way 
%\item Ensslin 's work which is an application of the theory of
%plane-parallel shock acceleration, which can be justified by the large
%radius of the shock sphere
%\item we consider the most conservative constraint that can be recovered
%from this model, which is strong/weak field case with a spectral index of
%$\alpha_{\text{spectral}}\sim 2$ combined with the observed mean
%polarization fraction of $P \sim 33.3\%$, we recover a  
%\item
%\end{itemize}
%\begin{equation}
%P(\alpha) = 
%\frac{1}{2} - \frac{1}{2} \text{erf}\left(\frac{1}{\sqrt{2}}\frac{\alpha -
%(\mu_\alpha+3\degree)}{\sigma_\alpha}\right)  
%\label{eqn:prior}
%\end{equation}
%
%\noindent See Appendix \ref{app:priors} for a plot of (\ref{eqn:prior}).

%The polarization information has larger constraining power than the .   
%To test the effects of applying the prior on the aforementioned range of
%separation,  we have come up two priors and applied them separately
%%Therefore, the distance between the subclusters, which has to be less than twice the 3D distance between the radio relic from the center of the cluster, is taken conservatively to be $1.0~\mega$pc $<$ d$_{\mathrm{3D}}
%%(t_{\mathrm{obs}}) < 3.0~\mega$pc. 
% to the 3D separation of the subclusters at the time of observation 
%($d_{\mathrm{3D}}(t_{\mathrm{obs}})$):
%
%The effect of the uniform prior is shown in Figure \ref{fig:radioprior}.
%
%%\textbf{description of the radio observation} 
%
%\textbf{how the distances were determined - overview of previous work}
%


%---------------------------------------------------------------------------
\subsection{Extension to the Monte Carlo simulation - Determining merger scenario with radio relic position}
\label{sec: positionprior}
\label{sec: positionprior}
Even though there are list of standard required data as denoted in section
\ref{sec: inputs}, it is straight forward to incorporate new data variables.
With additional of data from the radio relic \citep{L13}, this simulation is
capable of providing a quantitative view of how likely each of 
 the two possible merger scenarios mentioned in \ref{sec: outputs} are true. 

We incorporate the speed of the radio relic as part of $\vec{D}$ for the
Monte Carlo simulation.
We draw  $v_{relic} \sim N(4300~\kilo\meter~\second^{-1}, 800
~\kilo\meter~\second^{-1})$ \citep{L13}. 
From Springel and Farrar we know that the shock speed is comparable to the
merger speed of the two subclusters.
\par      

% Assuming the conditions of a simple head-on collision hold, this simulation is a relatively cost effective way of
%inferring $\alpha$ and the time dependence.% without resorting to expensive hydrodynamical simulations or an educated guess.


%\begin{table*} 
%\begin{minipage}{180mm} 
%\caption{Table of comparison of PDFs of the data in the Monte Carlo
%simulation with different applied priors
%\label{tab:predictiveposteriorchecking}} 
%\begin{tabular}{@{}lccccccc@{}}
%\toprule 
%&&& Original PDF & & & PDF of data after applying polarization priors \\ 
%\hline
%%\multicolumn{3-5}{c}{Default Priors} \multicolumn{6-8}{c}{Default + radio prior} \\
%%\cmidrule(r){1-3} \cmidrule(r){4-6}
%Data & Units & Location & 68$\%$ CI$^{\dagger}$ & 95$\%$ 
%CI & Location & 68$\%$ CI  & 95$\%$ CI \\
%\hline 
%$M_{200c_{NW}}$ &$10^{14}$ M$_{\odot}$&&&&&&\\
%$M_{200c_{SE}}$ &$10^{14}$ M$_{\odot}$&&&&&&\\
%$z_{NW}$ &/&&&&&&\\
%$z_{SE}$ &/&&&&&&\\
%$d_{proj}$ &Mpc&&&&&&\\
%\bottomrule 
%\end{tabular} 
%\footnotesize{\\$\dagger$ CI stands for credible interval } \\ 
%\end{minipage} 
%\end{table*} 

%----------------------------------------------------------------------

\section{RESULTS} 
\begin{figure} 
	\includegraphics[width =\linewidth]{elGordoTSCwithBullet_revC.png}
	\caption{The marginalized time-since-collision (TSC) vs 3D
velocities ($v_{3D}$) of El Gordo and the Bullet Cluster. (to add
descriptions of the different filters used) }
\end{figure}

\begin{figure}
	\includegraphics[width=\linewidth]{TSC_0_vs_v_3d_col_histplot2d_combine2.png}
	\caption{The marginalized output PDF of the time-since-collision
(TSC$_0$) vs. the 3D velocity at the time of collision for El Gordo. (to
add more description about contours) }
	\label{fig:TSC_v3D}
\end{figure}

%-----------------------------------------------------------------------
\subsection{Relative merger speed}
\textbf{The relative merger speed of the two subclusters is estimated to be
3400~\kilo\meter~\second$^{-1}$ at the time of collision.} The two
subclusters are estimated to slow down to a 3D relative velocity of only
$\sim800$ \kilo\meter~\second$^{-1}$  at the time of observation. Based on
this time evolution of the $v_{3D}$, it is unlikely that El Gordo is a slow
merger as mentioned in \citet{M11}. 


%-----------------------------------------------------------------------
\subsection{Time-since-collision}  
\textbf{The simulation gives two plausible estimates for
the time-since-collision, with $TSC_0 = \giga \text{yr}$ and $TSC_1 = \giga
\text{yr}$}. Based on section \ref{sec: positionprior}, we have come up
with estimates for the position of the NW radio relic based on the two PDFs
of inferred TSC as shown in figure \ref{fig: positionprior}. We plotted
the PDF of the simulated positions of the relic for the outgoing scenario
(blue) and the incoming scenario (green) against
the observed position of the relic (red), with the width of the red
relic accounting for the reported width of the relic \citep{L13}. When we
assume $\langle v_{relic} \rangle / v_{3D,1}(t_{col}) = 1.0$ (uppermost
panel of Figure \ref{fig: positionprior}), which is approximately the upper
limit of how fast the shock can travel, the outgoing scenario is much more
favored. As we examine a decreased ratio of  $\langle v_{relic} \rangle /
v_{3D,1}(t_{col})$, we probe how much the shock could have slowed down
and still be consistent with the outgoing scenario. Figure \ref{fig:
positionprior} shows that if $\langle v_{relic} \rangle
\lesssim 0.6$, the outgoing scenario would be favored. \par 
	We intrepret Figure \ref{fig: positionprior} to be more favorable of
the outgoing scenario than the incoming scenario due to the following
reasons: 1) Both \citet{Springel2007} and \citet{Kang2007} showed a
more-or-less constant speed of the shock as the shock propagates out
from the merger center. 2) In order for the incoming scenario to be more
favorable, the shock has to travel significantly slower than $0.6~
v_{3D}(t_{col})$ for a significant period of time. With reasons listed
above, we present the rest of the results and figures based on the outgoing
scenario in the main text of this paper and leave those results for the
incoming scenario in Appendix \ref{app: results}. 

Other uncertainties arise from how we define the reference frame for the calculation. The
uncertainty associated with the two centroids are of the order
of $\sim 0.1 \mega$pc \citep{Jee13} and are relatively unimportant???.  

\begin{figure}
	\includegraphics[width=\linewidth]{default_prior_bounds.pdf}
	\caption{Comparison of the observed position of the relic (red) with the
	predicted position from the two simulated merger scenarios (blue for
	outgoing and green for the incoming scenario). The outgoing scenario
	is more favored than the incoming scenario since the shock speed is
	unlikely to travel at much less than $0.6 v_{3D,1}(t_{col}$ for a
	significant period of time. Default priors were
	used for this figure. Alternate version of this figure with the polarization prior applied can
	be found in Appendix \ref{app: results}. \label{fig: positionprior}}
\end{figure}

\begin{figure} 
	\includegraphics[width =\linewidth]{elGordoTSCwithBullet_revC.png}
	\caption{The marginalized outgoing time-since-collision (TSC_0) vs 3D
velocities ($v_{3D}$) of El Gordo and the Bullet Cluster. (to add
descriptions of the different filters used) }
\end{figure}

\begin{figure}
	\includegraphics[width=\linewidth]{TSC_0_vs_v_3d_col_histplot2d_combine2.png}
	\caption{The marginalized output PDF of the outgoing time-since-collision
(TSC$_0$) vs. the 3D velocity at the time of collision for El Gordo. (to
add more description about contours) }
	\label{fig:TSC_v3D}
\end{figure}

%%%%% list of stuff to talk about --- 
% - explain what the figure means 
% - how we intrepret that the outgoing scenario is more likely 
% - why we are leaving all the figures of incoming scenario in the appendix
%   / upon request   
% 
%Bse on the time evolution of radio relic, we speculate that El Gordo has already reached its apoapsis and the two subclusters are heading for another merger.
%%This degeneracy between $TSC_0$ and $TSC_1$ that
%is not resolved by taking the separation constrain from the radio relic. 

%the post-collision estimate of the
%time-since-collision ($TSC_0$), the pre-collision estimate of the
%time-since-collision ($TSC_1$) and the time between collisions ($T$).
%We pick the post-collision time-since-collision estimate ($TSC_0$) of Gyr to be representative of the observed status of El Gordo, instead of the pre-collision time-since-collision estimate ($TSC_1$). While the simulation models both scenarios either the subclusters are approaching each other (incoming) or they have already passed through each
%other and are moving apart (outgoing), the presence of the radio relic rules out the possibility that the two subclusters still have not encountered each other. 
%However, this does not exclude the possibility of having the subcluster
%approaching each other again after reaching apoapsis.  
%%\citet{b9} have reported that the simultaneous optical and near-IR data of
%%AC Her can be fitted by a combination of two blackbodies at 5680 and
%1800\,K, representing, respectively, the stellar and



\begin{figure} 
	\label{fig: positionprior}
	\includegraphics[width =\linewidth]{r_relic_4300.png}
	\caption{Posterior PDF of likely location of relic based on the
    outgoing and incoming scenario.}
\end{figure}
%-----------------------------------------------------------------------
%\subsection{Galaxy centroid from spectroscopic data} The galaxy
%distribution is bimodal (M12).


%-----------------------------------------------------------------------
\section{DISCUSSION}
%-----------------------------------------------------------------------

\subsection{Three-dimensional (3D) configuration of El Gordo}
\textbf{The Monte Carlo simulation estimates that the projection angle to
be 41.7\degree, with the CI = 22.7\degree, 61.14\degree.} 
\begin{itemize}
\item explains that there hasn't been quantitative constraints on the angle
for which double radio relic can be observed, even though that many studies
have suggested that the detection of radio relic should imply that $\alpha$
should be small. From this simulation we have shown that it is possible to detect
the double radio relics with $\alpha$ being as big as $61.14\degree$. 

\item Lindner et al. provided constraint of $\alpha > 7.8 \degree$ based 
on the dynamics.
\item James' paper did mention how the mass estimation depends on $\alpha$,
with the estimated mass being a lot smaller if $\alpha \ge 65\degree$. 
However, since we did use the larger mass estimate as the input of this
simulation, we can only say that the inferred $\alpha$ is consistent with
the mass estimation. 
% angle dependence of observation of radio relic,  this is the first estimate 
% that provides an upper bound on the angle
% for which we can observe double radio relic  
\item discussion of the different scenarios mention in M11:
1) we are viewing after core passage, but before first turn around, and
the merger speed is low"\\
2) the merger speed is high, but we are viewing after the first turn
around as the two components come together for a second core passage
\item discuss the inclination angle estimate from M11
\item Dave: explain where the limits of the projection angle comes
from. what observational evidence contradicts the low velocity
scenario the most
\end{itemize}

\textbf{With this new piece of evidence, we find that the absence of an
X-ray shock feature from El Gordo, may not be due to the merger speed being
low, as suggested by J13.} 
In particular, taking into account that the estimated projection angle of 
$\sim 41.7\degree$, we estimate the projected relative velocity to be 597 \kilo \meter~\second$^{-1}$, which is consistent with the estimated line-of-sight velocity differences of $586 \pm  96~\kilo \meter ~\second^{-1}$ in M11. 

Furthermore, the study from \cite{L13} Lindner et al. has come up
with an estimation of the shock velocity of the radio relic of El Gordo as 
$\sim 4000~\kilo \meter~\second^{-1}$. While this shock velocity is not the
same as the merger velocity, they should be of similar magnitude. Indeed
our simulation found that a merger velocity of $4000~\kilo
\meter~\second^{-1}$ is within the 95\% credible interval. 


\subsection{Our finding in the context of other studies of El Gordo}
Compare to \citet{L13}.
Compare to \citet{Donnert13} for their best fit scenario.







%See Menanteau 2012 v1 last section to incorporate discussion of sound
%crossing time of $\sim $ 1~Gyr. 

%\begin{itemize}
%\item I want to compare the radio relic speed $~4300 km~s^{-1}$ estimated in  Lindner et al. to our
%$v_{3D}{(t_{obs})}$ estimate but I am not sure if they are quoting the speed
%in the same reference frame. I need to double check
%%Also we note that from Lindner et al. in press, it is found that the
%%projected speed of the radio relic is estimated to be $4300 \pm ^{800}_{500} \kilo
%%\meter~\second{-1}$. Since the radio relic does not experience
%%gravitational effects to slow it down and is generated from the merger, it
%%can be used to approximate the collision speed.  
%\item I can also discuss the TSC constraint from the observation of the
%depression in X-ray (the wake) using the argument that sound-crossing time is $\sim 1$
%Gyr. This should set an upper limit to the $TSC$, but $TSC_0 = 0.62$ Gyr
%and $TSC_1 = 1.01$ Gyr, I do not think it helps break the degeneracy.   
%\end{itemize}
%


\subsection{Comparison to other merger clusters of galaxies}

Talks about how El Gordo is more massive and collided at higher speed than
both the Bullet and the Musketball, so El Gordo is probably a better probe of SIDM properties.
%This estimated value has
%several implications for the analysis of the mass estimation and our
%knowledge of the physics underlying the creation of radio relic.   
%
%Compare with other literature???? 
%address concerns from James' paper that talks about how the mass would
%remain high as long as the viewing angle is $\le$ 65\degree ? 
%
% defy expectation 
%brings up that the large projection angle is unexpected
%merger axis has to be close to the sky for the radio relic to be observed 
%From simulation, it has been shown that the size of radio relic can be
%used to constrain with the mass ratio of the subclusters, with the more massive subcluster being further away from the larger relic. However, this is not the case for 
%El Gordo. The more prominent relic is located further away from the less massive
%SE cluster. This can be explained if the following holds: 
%\begin{enumerate}
%\item the less massive SE subcluster shows a higher redshift than the NW subcluster. This suggests that the SE radio relic could be moving away from us and
%hence appears to be less bright than its NW counterpart\\ 
%\item the extend to which (i) is true depends on the projection angle. With
%$\alpha \sim 41.7\degree$, one can naively calculate that the observed relic
%would be only $\cos(41.7 \text{ deg}) \sim .75 $ the size of the original relic. %\end{enumerate}

\textbf{With this new piece of evidence, we find that the absence of an
X-ray shock feature from El Gordo, may not be due to the merger speed being
low, as suggested by J13.} 
In particular, taking into account that the estimated projection angle of 
$\sim 41.7\degree$, we estimate the projected relative velocity to be 597 \kilo \meter~\second$^{-1}$, which is consistent with the estimated line-of-sight velocity differences of $586 \pm  96~\kilo \meter ~\second^{-1}$ in M11. 

Furthermore, the study from \cite{L13} Lindner et al. has come up
with an estimation of the shock velocity of the radio relic of El Gordo as 
$\sim 4000~\kilo \meter~\second^{-1}$. While this shock velocity is not the
same as the merger velocity, they should be of similar magnitude. Indeed
our simulation found that a merger velocity of $4000~\kilo
\meter~\second^{-1}$ is within the 95\% credible interval. 


\subsection{Uncertainties of the radio relic prior}
needs better simulation to understand the physical properties of radio relic.

position information can be used better to constraint the angle / the TSC,
projection effects - can be degenerate according to \cite{S13}   


\subsection{Limitations of our model and future work} 
Impact parameter of El Gordo may not be negligible. 
Simulations from Ricker \& Sarazin (2001) showed that cool-core is not
disrupted  when the impact parameters of mergers are of the order of $\sim 500$ kpc.   




%-----------------------------------------------------------------------

\section{SUMMARY \& CONCLUSION}
This paper presents one of the first examples of using the observed radio
relic emission to constrain cluster merger properties.
While we have demonstrated how to use the physical properties of the radio
relic emission to constrain merger dynamics and configurations, many
improvements can still be made as more studies of radio relic are being
done from both cosmological simulations and observations.

Currently, there are only a few studies of
 radio relic available for a range of viewing angles (\citealt{S13}, one of
Bruggen's paper). As more cosmological simulations inform us  
if the relic is observable at certain viewing angles will help us 
come up with better Monte Carlo filters. 

\section{ACKNOWLEDGEMENTS}
We thank Franco Vazza and Marcus Br\"{u}ggen for sharing their knowledge on
the simulated properties of radio relic. We
extend our gratitude to Reinout Van Weeren for first proposing the use of
radio relic as prior. We appreciate the comments from Maru\v{s}a
Brada\v{c} about using the position of the relic to break degeneracy
of the merger scenario. 
%KN is grateful to Paul Baines for
%the discussion of the design of Monte Carlo filters and sensitivity tests.   


%-----------------------------------------------------------------------
\bibliographystyle{mn2e}
%\begin{thebibliography}{}
\bibliography{bib}

\appendix

\section{BAYESIAN FORMALISM OF DAWSON'S MONTE CARLO SIMULATION}
The default prior probabilities that we employed can be summarized as
follows for most of the output variables: 
\begin{equation}
	P(TSC_0) = 
	\begin{cases}
		& \text{const}~\text{if }TSC_0 < \text{age of universe at } z=0.87	\\
		& 0~\text{otherwise}
	\end{cases}
\end{equation}

In addition, we apply the following prior on $TSC_1$ only when evaluating the
statistics of $TSC_1$ and $T$, thus allowing realiziations with a valid
outgoing TSC but an invalid returning $TSC_1$.  

\begin{equation}
	P(TSC_1) = 
	\begin{cases}
		& \text{const}~\text{if }TSC_1 < \text{age of universe at } z=0.87	\\
		& 0~\text{otherwise}
	\end{cases}
\end{equation}

To correct for observational limitations, we further convolve the posterior probabilities of the different
realizations with 
\begin{equation}
	P(TSC_0 | T) = 2 \frac{TSC_0}{T}
\end{equation}
to account for how the subclusters move faster at lower $TSC$ and thus it
is more probable to observe the subclusters at a stage with a larger $TSC$.






%\begin{table*} 
%\begin{minipage}{180mm} 
%\caption{Table of comparison of PDFs of the data in the Monte Carlo
%simulation with different applied priors
%\label{tab:predictiveposteriorchecking}} 
%\begin{tabular}{@{}lccccccc@{}}
%\toprule 
%&&& Original PDF & & & PDF of data after applying polarization priors \\ 
%\hline
%%\multicolumn{3-5}{c}{Default Priors} \multicolumn{6-8}{c}{Default + radio prior} \\
%%\cmidrule(r){1-3} \cmidrule(r){4-6}
%Data & Units & Location & 68$\%$ CI$^{\dagger}$ & 95$\%$ 
%CI & Location & 68$\%$ CI  & 95$\%$ CI \\
%\hline 
%$M_{200c_{NW}}$ &$10^{14}$ M$_{\odot}$&&&&&&\\
%$M_{200c_{SE}}$ &$10^{14}$ M$_{\odot}$&&&&&&\\
%$z_{NW}$ &/&&&&&&\\
%$z_{SE}$ &/&&&&&&\\
%$d_{proj}$ &Mpc&&&&&&\\
%\bottomrule 
%\end{tabular} 
%\footnotesize{\\$\dagger$ CI stands for credible interval } \\ 
%\end{minipage} 
%\end{table*} 




\section{DETAILS AND TESTS FOR THE MCMC MASS INFERENCE}\label{app:MCMC}
The reduced shear generated by each NFW halo is determined by its
mass ($m_{200c}$) and the position of its center ($\vec{s}$). 
%This parametrization
%is possible since we make use of the mass-concentration relationship from 
%\citet{Duffy2008} to express the concentration of the halo in terms of the
%$m_{200c}$.
\par
% aligning multiline equations:
% http://tex.stackexchange.com/questions/44450/how-to-align-a-set-of-multiline-equations
%\begin{align}
%    &\begin{aligned}    
%    P(m_{a},  \vec{s_{a}}, &m_{2}, \vec{s_{2}} | \vec{e}) \propto \qquad\\
%    &P(\vec{e} | m_{1}, \vec{s_{1}}, m_{2}, \vec{s_{2}}) P(m_1) 
%    P(\vec{s_{1}}) P(m_2) P(\vec{s_{2}})  
%    \end{aligned}
%\end{align}
We consider the joint posterior as the fit to the ellipticity data:
%\begin{align}
%    &\begin{aligned}
%			&\log(P( m_{1}, m_{2}, \vec_{s_1}, \vec_{s_2} | \vec{e} )) \propto\\
%			&-\left[\frac{(\hat{e_1}(m_1, m_2, \vec{s_1}, \vec{s_2}) - e_1)^2
%    }{\sigma_{e_1}^2+\sigma_{SN}^2 }+ 
%    \frac{(\hat{e_2}(m_1, m_2) - e_2)^2
%    }{\sigma_{e_2}^2 + \sigma_{SN}^2 }\right] \label{eqn:jointposterior} 
%    \end{aligned}
%\end{align} 
%where we have fixed the centers of the halos so $\vec{s}_1$ and $\vec{s}_2$
%so $\vec{s}_1$ and $\vec{s}_2$ are left out of the joint posterior.

\begin{align}
    &\begin{aligned}
			&\log(P( m_{1}, \vec{s}_{1}, m_{2}, \vec{s}_{2} | \vec{e} )) \propto\\
%    &-\left[\frac{(\hat{e_1}(m_1, \vec{s_1}, m_2, \vec{s_2}) - e_1)^2
%    }{\sigma_{e_1}^2+\sigma_{SN}^2 }+ 
%    \frac{(\hat{e_2}(m_1, \vec{s_1}, m_2, \vec{s_2}) - e_2)^2
%    }{\sigma_{e_2}^2+\sigma_{SN}^2 }\right] \label{eqn:jointposterior} 
    \end{aligned}
\end{align} 

Gaussian shape noise of the background galaxies are represented by $\sigma_{SN} = 0.25$(to be checked) represents Gaussian shape noise
of the background galaxies, this is also the form of uncertainty made used
of by Umetsu et. al. on a similar analysis; 


The reduced shear due to the NFW halos can be
decomposed into two components, $\hat{e_1}$ and $\hat{e_2}$  
%The ellipticities generated by a NFW halo can be summarized as:
%\begin{align}
%    \hat{e}_1(m, \vec{s_1}) &=\\
%    \hat{e}_2(m, \vec{s_2}) &= 
%\end{align}
%\textbf{where did we use Duffy et al?}
%To reduce the number of model variables, we also made use of the
%mass-concentration relationship for NFW halos from Duffy et al. (2008). 

We apply a uniform (or log flat) prior and only drew starting mass values between
$10^{13} M_\odot$ and $10^{15} M_\odot$ for our MCMC
chains, as informed by previous published mass estimates. (M11, J13, Zitrin
et al. 2013). For each of the subsequent MCMC step, we 
draw a random pair of mass value with the values of
previous step as the means of the distributions, and two pairs of
coordinates for the centroids, i.e. we can write down the
proposal functions at each MCMC step as: 

\begin{align}
	&\Delta m = N(0, \sigma_m) \hspace{1pc} \text{for each halo} \\
	&\Delta s = U(s_{min}, s_{max}) \hspace{1pc} \text{for each RA, DEC for
	each halo}
\end{align}






We may use blocking. 
(THERE SHOULD BE A TRACE PLOT HERE REPORTING THE ACCEPTANCE RATE.)

We vary the step size of our MCMC  before the burn-in period such that an
optimal acceptance rate of the Metropolis algorithm of $\sim0.234$ \citep{Roberts97} is achieved.

In addition to using the Rubin-Gelman R statistic to check for convergence,
we remove autocorrelation of our chains by estimating the effective sample
size of our chains.
 


\section{RESULT PLOTS OF MONTE CARLO SIMULATION} \label{app: plots}
%make use of subplots from matplotlib in order to plot all the posteriors

\bsp 
\label{lastpage} 
\end{document}
