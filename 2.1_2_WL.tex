
%The input 
\textbf{We obtained the PDFs of the masses of the subclusters by doing a Monte
Carlo Markov Chain (MCMC) analysis of the reduced shear from the
weakly lensed background galaxies. } We computed the reduced shear signal
generated by two NFW halos according to \citet{Umetsu10} (See Appendix
\ref{app:MCMC} for
details of implementation and output diagnostics).
At each step we followed the procedure of a
Metropolis algorithm.  The transition kernel was set to
be the log likelihood of fit of the model shear to the reduced shear of the
data (\ref{eqn:jointposterior}).
In total, eight MCMC chains were used. After every 5000 MCMC steps for all
the chains, we computed the R coefficient \citep{Gelman92}  to
check for convergence. We performed more MCMC steps as long as convergence
was not achieved. After convergence was achieved, we removed the
burn-in portions of the MCMC chains and used the resulting MCMC chains as
samples of the PDFs of the masses. \par 
%\textbf{The inputs of our MCMC mass inference are from proposal blah,
%which is similar to \citepalias{Jee13}, however, this
%separate implementation of the MCMC analysis code is different than
%\citepalias{Jee13}.} 
We used an identical catalog of reduced and bias-corrected background 
galaxy shapes as in Hubble Space Telescope PROP 12755 from \citetalias{Jee13}. On the other hand, we fixed the
position of the centers of the NFW halos to be  the luminosity peaks of the
respective galaxy populations of  the two subclusters, which are at R.A. = 01:02:51.68, Decl. = -49:15:04.40 and R.A. = 01:02:38.38, Decl. = -49:16:37.64 for the NW and SE subclusters
respectively \citepalias{Jee13}.  The agreement between our analysis and \citepalias{Jee13} to within
the 68\% credible interval serves as a sanity check on the estimated masses. 
%Other sanity check including the control of acceptance rate between 20\%
%and 50\%. 
