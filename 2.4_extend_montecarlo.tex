\label{sec: positionprior}
Even though there are list of standard required data as denoted in section
\ref{sec: inputs}, it is straight forward to incorporate new data variables.
With additional of data from the radio relic \citep{L13}, this simulation is
capable of providing a quantitative view of how likely each of 
 the two possible merger scenarios mentioned in \ref{sec: outputs} are true. 

We incorporate the speed of the radio relic as part of $\vec{D}$ for the
Monte Carlo simulation.
We draw  $v_{relic} \sim N(4300~\kilo\meter~\second^{-1}, 800
~\kilo\meter~\second^{-1})$ \citep{L13}. 
From Springel and Farrar we know that the shock speed is comparable to the
merger speed of the two subclusters.
\par      

% Assuming the conditions of a simple head-on collision hold, this simulation is a relatively cost effective way of
%inferring $\alpha$ and the time dependence.% without resorting to expensive hydrodynamical simulations or an educated guess.
