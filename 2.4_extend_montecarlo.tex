\label{sec: positionprior}

% draft: 
% explain that 4300 km /s from L13 
% explain how the position of the radio relic is calculated 
% explain the physics / concepts behind this calculation 
% explain the assumptions
% explain the data 




Even though there is a list of standard required data as denoted in section
\ref{sec: inputs} for the simulation, it is straight forward to incorporate
new data variables. With additional of data from the radio relic
\citep{L13}, this simulation is capable of providing a quantitative view of
how likely each of  the two possible merger scenarios mentioned in section
\ref{sec: outputs} is true. We incorporate the speed of the radio relic as
part of $\vec{D}$ for the Monte Carlo simulation. We draw  $v_{relic} \sim
N(4300~\kilo\meter~\second^{-1}, 800~\kilo\meter~\second^{-1})$
\citep{L13}.  From numerical simulations of \citet{Springel2007} we know that the shock speed is comparable to the
merger speed of the two subclusters. The shock speed drops by $~14\%$
300~\Mega yr after the collision. 

We therefore estimated the location of the relic using:
\begin{equation}
s_{proj} = v_{relic} \times TSC \cos(\alpha) 
\end{equation}
where $s_{proj}$ represents an upper bound of the projected separation. We used a upper bound of the relic speed in this calculation.
The resulting $s_{proj}$ captures all the uncertainties associated with
the input. 



\par      

