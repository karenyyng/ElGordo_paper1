

% \textbf{how do we determine the overall membership}
% relevant info of the membership is in M11 table 1
% cannot determine the membership of two galaxies 
% spatial cut is defined in M11 p.9 first paragraph 
\textbf{We used a 2D spatial cut to determine members of the two
subclusters, then bootstrapped the biweight locations of the redshifts of the
respective members in order to obtain the PDFs of the redshifts of each
subcluster.} We made use of the spectroscopic data
obtained from the Very Large Telescope (VLT) and Germini South as described
in \citetalias{M11} and \citet{Sifon13}. The overall membership of the galaxies
of El Gordo was determined using a shifting gapper method after applying a
rest frame cut of 4000~\kilo\meter~\second$^{-1}$. This method gives a
total count of 89 galaxy members of El Gordo. From the 2D spatial cut of
the confirmed members, we determined that there are 54 members in the NW
subclusters and 35 members in the SE subclusters. (See Figure
\ref{fig:membership}) The spectroscopic redshift of the clusters were
determined to be $z_{NW} = 0.86901 \pm 0.00017$ and $z_{SE} = 0.87175 \pm
0.00019$, where the quoted numbers represent the biweight location and
biweight scale respectively \citep{Beers90}. The biweight location
estimators are less susceptible to outliers than the mean and standard
deviations. 

%We confirm the existence of the subclusters of El Gordo from the 2D spatial
%location of the galaxies, with the more massive cluster lying in the
%northwest (NW) and the less massive subcluster lying in the southeast(SE)  (REF - also have to reference the other literature
%for confirmation in the other wavelengths)
%
%Specifically, the membership of the NW and SE subclusters are
%determined based on a combination of the redshift and the spatial location. 
